\documentclass{article}
\usepackage{amsmath}
\usepackage{amssymb}
\usepackage{mathdots}
\usepackage[compact]{titlesec}
\usepackage{graphicx}
\usepackage{tikz}
\usetikzlibrary{calc,fadings,decorations.pathreplacing,decorations.markings,shapes,shapes.multipart,arrows,shapes.misc,intersections,positioning}
\usepackage{wrapfig}
\usepackage{bm}
\renewcommand{\vec}[1]{\boldsymbol{\mathbf{#1}}}

\textwidth = 6.5 in
\textheight = 9 in
\oddsidemargin = 0.0 in
\evensidemargin = 0.0 in
\topmargin = 0.0 in
\headheight = 0.0 in
\headsep = 0.0 in
\parskip = 0.2in
\parindent = 0.0in

\title{Localized (Point) Defect}
\author{Tianci Zhou}
\begin{document}
\maketitle
\large

{\bf Boson Point Defect}

\noindent\rule{\linewidth}{0.4pt}

\underline{Background}: Boson scale invariant defect. Specifically for a continuous free boson field, we impose
\begin{equation}
\partial_t \phi_1 = \frac{1}{\lambda} \partial_t \phi_2 \quad \partial_x \phi_1 = \lambda \partial_x \phi_2 
\end{equation}
where $\phi_1$ and $\phi_2$ are the boundary values of $\phi$ field on the two sides of the defect at $x = 0$. See bCFT notes for details. 

\underline{$S$ matrix}: We match the free solution on the two sides
\begin{equation}
\phi_1 = A_{-} e^{i \omega t - i k x } + A_{+} e^{i \omega t + ikx } \quad \phi_2 = B_{-} e^{i \omega t - i k x } + B_{+} e^{i \omega t + ikx }
\end{equation}
which gives
\begin{equation}
\begin{aligned}
- A_{-} + A_{+} =\lambda (  - B_{-} +  B_{+} ) \\
A_{-} + A_{+} = \frac{1}{\lambda} (B_{-} + B_{+}) \\
\end{aligned}
\end{equation}
The $S$ matrix converts the incoming wave to the outgoing wave $\begin{bmatrix} A_{+}\\ B_{-}\\ \end{bmatrix} = S \begin{bmatrix} A_{-}\\ B_{+}\\ \end{bmatrix}$  is 
\begin{equation}
S = 
\begin{bmatrix}
\frac{1 - \lambda^2}{1 + \lambda^2} & \frac{2\lambda }{1 + \lambda^2} \\
\frac{2\lambda }{1 + \lambda^2} & \frac{-1 + \lambda^2}{1 + \lambda^2} \\
\end{bmatrix}
 = 
\begin{bmatrix}
\cos 2\theta & \sin 2\theta \\
\sin 2\theta & - \cos2  \theta \\
\end{bmatrix}
\quad \text{where } \lambda = \tan \theta 
\end{equation}
\noindent\rule{\linewidth}{0.4pt}

{\bf Harmonic Chain} 

We construct the same defect in a discrete model. It corresponds to connecting two chains with different spring constant(rescaling for the compact case in the continuum limit.)

Consider a uniform infinite harmonic chain with a bond defect between site $0$ and $1$,
\begin{equation}
  H = \frac{1}{2} \sum_i \pi_i^2  +  \frac{1}{2} \sum_{i\ne 0 }  ( \phi_i - \phi_{i+1} )^2  +  \frac{1}{2} \begin{bmatrix}  \phi_0, \phi_1 \end{bmatrix}
\begin{bmatrix}
1 + \Sigma_{11}  & \Sigma_{12} \\
\Sigma_{21} &  1 + \Sigma_{22} \\
\end{bmatrix}
\begin{bmatrix}
\phi_0 \\
\phi_1 
\end{bmatrix}
\end{equation}
where we parameterize all possible interactions between size $0$ and $1$ with $\Sigma_{ij}$. 

The equation of motion is
\begin{equation}
-\frac{d^2}{dt^2} 
\begin{bmatrix}
\vdots\\
\phi_{-1}\\
\phi_0\\
\phi_1\\
\phi_2\\
\vdots 
\end{bmatrix}
 =  
\begin{bmatrix}
\ddots & & & & &\\
-1 & 2 & -1 & & &  \\
&-1 & 2 + \Sigma_{11} & \Sigma_{12} & & &\\ 
& & \Sigma_{21} & 2 + \Sigma_{22} & -1 & &\\
& & & -1 & 2 & -1 & \\
& & & & & \ddots &
\end{bmatrix}
\begin{bmatrix}
\vdots\\
\phi_{-1}\\
\phi_0\\
\phi_1\\
\phi_2\\
\vdots 
\end{bmatrix}
\end{equation}
Again we match the solution between site $\phi_0$ and $\phi_1$ (the $n -1$ in $B$ simplifies the calculation)
\begin{equation}
\phi_n
= \left\lbrace
  \begin{aligned}
    \phi_n^A &= A_{-} e^{i \omega t  - ikn}  + A_{+} e^{i \omega t  + ikn}  & \quad  n \le 0 \\
    \phi_n^B &= B_{-} e^{i \omega t  - ik(n-1)}  + B_{+} e^{i \omega t  + ik(n-1)} & \quad n \ge 1 \\
  \end{aligned} \right. 
 \quad 
\end{equation}
At the junction we have
\begin{equation}
\begin{aligned}
- \phi_{-1}^A + ( 2+ \Sigma_{11} ) \phi_{0}^A + \Sigma_{12} \phi_1^B =  \omega^2 \phi_0^A = - \phi_{-1}^A + 2 \phi_{0}^A - \phi_1^A\\
\Sigma_{21} \phi_0^A + ( 2 + \Sigma_{22}) \phi_1^B - \phi_2^B = \omega^2 \phi_1^B  = - \phi_0^B + 2 \phi_1^B - \phi_2^B 
\end{aligned}
\end{equation}
which simplifies to
\begin{equation}
\begin{aligned}
\Sigma_{11} \phi_0^A + \Sigma_{12} \phi_1^B + \phi_1^A  =0 \\
\Sigma_{21} \phi_0^A + \Sigma_{22} \phi_1^B + \phi_0^B = 0 \\
\end{aligned}
\end{equation}
From the relation of the amplitudes
\begin{equation}
\begin{bmatrix}
\Sigma_{11} + e^{-ik} & \Sigma_{12} \\
\Sigma_{21} & \Sigma_{22}  + e^{-ik}  \\
\end{bmatrix}
\begin{bmatrix}
A_{-} \\
B_{+}\\
\end{bmatrix}
=-
\begin{bmatrix}
\Sigma_{11} +e^{ik} & \Sigma_{12}\\
\Sigma_{21} & \Sigma_{22} + e^{ik}
\end{bmatrix}
\begin{bmatrix}
A_{+}\\
B_{-}\\
\end{bmatrix}
\end{equation}
we solve the $S$ matrix
\begin{equation}
  S = \frac{-1}{ \det \Sigma  + e^{-ik} \text{tr} \Sigma   + e^{-2ik}}
\begin{bmatrix}
\det \Sigma+ \Sigma_{11} e^{-ik} + \Sigma_{22} e^{ik}+1  & -2i \sin k \Sigma_{12}  \\
-2i \sin k \Sigma_{21} &  \det \Sigma+ \Sigma_{11} e^{ik} + \Sigma_{22} e^{-ik}+1\\
\end{bmatrix}
\end{equation}

The magnitudes of the $S$ matrix elements are refection and transmission coefficients. In order to have a scale invariant defect, they need to be $k$ independent. In particular there are two ways to generate $k$ independence of $|S_{12}|$:
\begin{enumerate}
\item $\Sigma_{12} = \Sigma_{21} = 0$, then the $\Sigma$ matrix is diagonal, and 
\begin{equation}
  -S_{11} = \frac{\det \Sigma + \Sigma_{11} e^{-ik} + \Sigma_{22} e^{ik} + 1}{\det \Sigma + \text{tr} \Sigma + e^{-ik} } = \frac{(\Sigma_{11} + e^{ik})(\Sigma_{22} + e^{-ik} ) }{(\Sigma_{11} + e^{-ik})(\Sigma_{22} + e^{-ik} ) }= \frac{\Sigma_{11} + e^{ik} }{\Sigma_{11} + e^{-ik} }
\end{equation}
it is not possible to make the modulus $k$ independent.
\item The determinant is cancels the $\sin k$
\begin{equation}
|e^{-ik} \det \Sigma + \text{tr} \Sigma + e^{-ik } | \sim \sin k \quad \implies \quad \det \Sigma = -1, \, \, \text{tr} \Sigma = 0
\end{equation}
\end{enumerate}
Consequently, the scale invariant $S$ matrix is
\begin{equation}
S = \frac{1}{1 - e^{-2ik } } ( -2i \sin k ) \Sigma
 = - e^{ik} \Sigma
\end{equation}
The phase $e^{ik}$ is the phase difference between neighboring site, which should really be $e^{ika}$ where $a$ is the lattice spacing. Hence in continuum limit, $e^{ik a} \rightarrow 1$,
\begin{equation}
\Sigma = -\lim_{a \rightarrow 0 } S = 
\begin{bmatrix}
\frac{\lambda^2- 1}{1 + \lambda^2} & \frac{-2\lambda }{1 + \lambda^2} \\
\frac{-2\lambda }{1 + \lambda^2} & \frac{1- \lambda^2}{1 + \lambda^2} \\
\end{bmatrix}
\end{equation}
It corresponds to the potential term $\frac{1}{1+ \lambda^2}( \lambda \phi_0 - \phi_1 ) ^2 $, which is derived by transfer matrix method in \cite{peschel_exact_2012}.

On the other hand, if one of the $\det$ and $\text{tr}$ conditions are not satisfied, then
\begin{equation}
\lim_{a \rightarrow 0 } S = \frac{-1}{\det \Sigma  + \text{tr} \Sigma   + 1}
\begin{bmatrix}
 \det \Sigma+ \Sigma_{11}  + \Sigma_{22} +1  & 0  \\
0 &   \det \Sigma+ \Sigma_{11} + \Sigma_{22} +1 \\
\end{bmatrix} = - I 
\end{equation}
which corresponds to Dirichlet-Dirichlet boundary condition. 

\underline{Boundary}

For a boundary at $x =0$, we restrict to the corner of the discrete Laplacian,
\begin{equation}
-\frac{d^2}{dt^2} 
\begin{bmatrix}
\vdots\\
\phi_{-1}\\
\phi_0\\
\end{bmatrix}
=
\begin{bmatrix}
\ddots & & \\
-1 & 2 & -1\\
  &-1 & 2+\Sigma\\
\end{bmatrix}
\begin{bmatrix}
\vdots\\
\phi_{-1}\\
\phi_0\\
\end{bmatrix}
\end{equation}
The discrete equation on the corner is
\begin{equation}
\begin{aligned}
-\phi_{-1}^A + ( 2 + \Sigma ) \phi_0^A  &= \omega^2 \phi_0^A = -\phi_{-1}^A + 2 \phi_0^A  - \phi_1^A \\
\implies \quad \Sigma \phi_0^A + \phi_1^A &= 0 \\
\implies \quad A_{+} &= - \frac{\Sigma + e^{ika}}{ \Sigma + e^{-ika} } A_{-} 
\end{aligned}
\end{equation}
In the long wavelength limit, the boundary conditions will flow to two fixed points
\begin{equation}
S = - \lim_{a\rightarrow 0 } \frac{\Sigma + e^{ika}}{ \Sigma + e^{-ika} } = 
\left\lbrace
\begin{aligned}
  -1 & \quad \Sigma \ne -1  \quad \text{Dirichlet}(\text{fixed point } \Sigma = \pm \infty)\\
   1&  \quad \Sigma =-1 \quad \text{Neumann}\\
\end{aligned} \right. 
\end{equation}

{\bf T-dual Boundary Condition}

There is another type of scale invariant defect, which is the T-dual of the original one, 
\begin{equation}
\partial_t \phi_1 = \lambda \partial_x \phi_2 \quad \partial_x \phi_1 = \frac{1}{\lambda} \partial_t \phi_2 
\end{equation}

\underline{$S$ matrix}: We match the free solution on the two sides
\begin{equation}
\phi_1 = A_{-} e^{i \omega t - i k x } + A_{+} e^{i \omega t + ikx } \quad \phi_2 = B_{-} e^{i \omega t - i k x } + B_{+} e^{i \omega t + ikx }
\end{equation}
which gives
\begin{equation}
\begin{aligned}
i \sigma( A_{-} + A_{+} )  =\lambda  ik (  - B_{-} +  B_{+} ) \\
i\omega( B_{-} + B_{+} ) = \lambda  ik( - A_{-} + A_{+}) \\
\end{aligned}
\end{equation}
Let $\omega = k$, the $S$ matrix converts the incoming wave to the outgoing wave $\begin{bmatrix} A_{+}\\ B_{-}\\ \end{bmatrix} = S \begin{bmatrix} A_{-}\\ B_{+}\\ \end{bmatrix}$  is 
\begin{equation}
S = 
\begin{bmatrix}
\frac{\lambda^2 - 1}{1 + \lambda^2} & \frac{2\lambda }{1 + \lambda^2} \\
\frac{-2\lambda }{1 + \lambda^2} & \frac{\lambda^2 - 1}{1 + \lambda^2} \\
\end{bmatrix}
 = 
\begin{bmatrix}
-\cos 2\theta & \sin 2\theta \\
-\sin 2\theta & - \cos2  \theta \\
\end{bmatrix}
\quad \text{where } \lambda = \tan \theta 
\end{equation}
which is the $S_2$ in the bCFT notes. \\
\noindent\rule{\linewidth}{0.4pt}

[tentative]
Since time derivative is involved, this is a dynamical boundary condition; we would expect that the Hamiltonian will contain $xp$ term. 

A guess would be
\begin{equation}
H = - \lambda \partial_x \phi_1 \pi_2 - - \lambda \partial_x \phi_2 \pi_1
\end{equation}
in this way, the resulting equation of motion will be
\begin{equation}
\partial_t \phi_2  = \lambda \partial_x \phi_1 \quad \partial_t \phi_1  = \lambda \partial_x \phi_2
\end{equation}
and the momentum part upon integration by part (which is not allowed, since this Hamiltonian is at the boundary), 
\begin{equation}
\partial_t \pi_2  = \lambda \partial_x \pi_1 \quad \partial_t \pi_1  = \lambda \partial_x \pi_2
\end{equation}


\bibliographystyle{unsrt}
\bibliography{defect}

\end{document}
%%% Local Variables: 
%%% TeX-PDF-mode: t
%%% End: