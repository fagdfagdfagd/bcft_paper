
In this paper, we analyzed a class of boson conformal interfaces by computing the Loschmidt echo and the bipartite fidelity. 

We began by classifying the boundary states by two types of $S$ matrices $S_1(\theta)$ and $S_2(\theta)$, where the parameter $\theta$ -- the scattering angle -- is determined by the transmission coefficient of the interface. The conventional `DD', `NN' boundary conditions are among the special choices of $\theta$ in $S_1$, and `DN' and P are among the special choices of $S_2$. Generic value of $\theta$ then interpolates between those conventional boundary conditions. A harmonic chain model allows us to realize part of these partially-transmitive boundary conditions in a concrete lattice setting. 

The dynamical behavior of the Loschmidt echo reflects the change of the conformal interfaces during the process described in Eq.~\eqref{eq:S_i_S_j}.
%, which we denote as
%\begin{equation}
%S_i( \theta_1 ) \rightarrow S_j( \theta_2 )  \quad i, j = 1, 2. 
%\end{equation}
Its power law decay exponent is related to the scaling dimension of the bcc operator that mediates the interfaces. Analytic computation shows that the exponent is always $\frac{1}{4}$ when the change of boundary conditions is made between different types of $S$ matrices ($i \ne j$), regardless of the choice of $\theta$. On the other hand, the exponent depends on the difference of angles $\theta_1 - \theta_2$ as a quadratic relation when the change is made between the same type of $S$ matrices ($i = j$).

These two features are tested in three typical processes in the numerical calculation of the harmonic chains. After using suitable regulators for the zero-mode problem, the numerical results agree with the analytic calculation within error. Although tangential to the non-equilibrium dynamics, the fidelity calculation is used as a diagnostic tool and shows better agreement of the exponent, providing more confidence about our analytic results. 

We proposed that the Loschmidt echo exponent in principle should be detectable in an X-ray edge singulrity type experiment on the quantum wire systems. 

%%% Local Variables:
%%% TeX-master: "bCFT_paper"
%%% TeX-PDF-mode: t
%%% End:
