
In this paper, we analyze a class of boson conformal interfaces by computing the fidelity and Loschmidt echo. 

We begin by classifying the boundary states by two types of $S$ matrices, where the conventional DD, NN boundary conditions belong to $S_1$ while DN and $P$ belongs to $S_2$. A harmonic chain model allows us to realize part of these boundary conditions, including the partially-transimitive ones on a concrete setting. 

The study of the Loschmidt echo are a generalization of the conventional "cut-and-join" protocol. Its power law decay exponent is related to the bcc operator scaling dimension that now can depend on the tune-able transmission coefficient. Analytic computation shows that the exponent is $\frac{1}{4}$ when the change of boundary condition is made between different types of $S$ matrices. While the exponent will depend on $\theta$ (scattering angle in the $S$ matrix) as a quadratic relation when the change is made between the same types of $S$ matrices of different $\theta$.

These two features are tested in three typical processes in the numerical calculation. After using suitable regulators for the zero-mode problem, the numerical results agree with the analytic calculation within error. Although irrelevant to the non-equilibrium dynamics, the fidelity calculation is used as a diagnostic tool and shows better agreement, providing more confidence about the analytic results. 

We proposed that the Loschmidt echo exponent in principle should be realizable in connecting two quantum wire with an X-ray type experiment. 



%%% Local Variables:
%%% TeX-master: "bCFT_paper"
%%% TeX-PDF-mode: t
%%% End:
