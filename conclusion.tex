
In this paper, we analyze a class of boson conformal interfaces by computing the Loschmidt echo and the bipartite fidelity. 

We began by classifying the boundary states by two types of $S$ matrices, where the conventional `DD', `NN' boundary conditions belong to $S_1$ and  `DN' and P belongs to $S_2$. A harmonic chain model allows us to realize part of these partially-transmitive boundary conditions a concrete lattice setting. 

The study of the Loschmidt echo is a generalization of the conventional ``cut-and-join" protocol. Its power law decay exponent is related to the bcc operator scaling dimension that can depend on the tune-able transmission coefficient. Analytic computation shows that the exponent is $\frac{1}{4}$ when the change of boundary condition is made between different types of $S$ matrices and depend on $\theta$ (scattering angle in the $S$ matrix) as a quadratic relation when the change is made between the same types of $S$ matrices of different $\theta$s.

These two features are tested in three typical processes in the numerical calculation of the harmonic chains. After using suitable regulators for the zero-mode problem, the numerical results agree with the analytic calculation within error. Although tangential to the non-equilibrium dynamics, the fidelity calculation is used as a diagnostic tool and shows better agreement of the exponent, providing more confidence about our analytic results. 

We proposed that the Loschmidt echo exponent in principle should be detectable in an X-ray edge singulrity type experiment on quantum wires systems. 



%%% Local Variables:
%%% TeX-master: "bCFT_paper"
%%% TeX-PDF-mode: t
%%% End:
