% flatex input: [comp_fid_echo.tex]

In this appendix, we provide the technical details on numerical calculation of bipartite fidelity and Loschmidt echo. We shall {\bf\color{red}1. lay out the plan for this long appendix. 2. Point out out method is different from Moore. 3. This is more general as Moore's does not apply to boson (say why?) but ours can be easily generalized to fermion 4. Say sth on Williamson. For fermion, Schur's decomposition replaces Williamson (probably in the end?). 5. Compare to the fermionic case in the echo section. } 

\subsection{Boson Bogoliubov transformation}
\label{Boson BdG}
We consider the following quadratic bosonic Hamiltonian
\begin{eqnarray}\begin{aligned}
\label{Quadratic boson H}
\hat{H}&=b_i^\dagger H_{ij}b_j+\frac{1}{2}b^\dagger_i\Delta_{ij}b_j^\dagger+\frac{1}{2}b_i\Delta^\dagger_{ij}b_j\\
&=\frac{1}{2}
\left(\begin{array}{ccc}
{\bf b}^\dagger & -{\bf b}
\end{array}\right)
\left(\begin{array}{ccc}
H & \Delta\\
-\Delta^\dagger & -H^T
\end{array}\right)
\left(\begin{array}{ccc}
{\bf b} \\
{\bf b}^\dagger
\end{array}\right)-\frac{1}{2}\text{tr}H,
\end{aligned}\end{eqnarray}
where ${\bf b}\equiv(b_{1},...,b_{N})$ is a vector of annihilation operators. Notice $H$ is a $N\times N$ single-particle Hamiltonian, and $\Delta$ is a symmetric paring matrix. The use of $({\bf b}^\dagger , -{\bf b})$ rather than $({\bf b}^\dagger , {\bf b})$ is due to the Sp(2N,{\bf C}) character of the bosonic Bogoliubov transformation, as we now demonstrate. 

The Bogoliubov transformation that diagonalizes the Hamiltonian
\begin{eqnarray}\begin{aligned}
\label{Def. of a}
({\bf a} , {\bf a}^\dagger)\equiv({\bf b} , {\bf b}^\dagger)S
\end{aligned}\end{eqnarray}
should preserve the commutation relation
\begin{eqnarray}\begin{aligned}
\label{Preserve commutator}
J\equiv\left(\begin{array}{ccc}
0 & {\bf I}\\
-{\bf I} & 0
\end{array}\right)
=\left[
\left(\begin{array}{ccc}
{\bf a} \\
{\bf a}^\dagger
\end{array}\right),
\left(\begin{array}{ccc}
{\bf a} & {\bf a}^\dagger
\end{array}\right)\right]
=S^T\left[
\left(\begin{array}{ccc}
{\bf b} \\
{\bf b}^\dagger
\end{array}\right),
\left(\begin{array}{ccc}
{\bf b} & {\bf b}^\dagger
\end{array}\right)\right]S
=S^T JS,
\end{aligned}\end{eqnarray}
where the commutator, for example, $\left[{\bf b},{\bf b}^\dagger\right]$ is understood as a matrix with component $\left[b_i,b_j^\dagger\right]=\delta_{ij}$. It is clear from Eq.(\ref{Preserve commutator}) that $S\in$Sp(2N,{\bf C}){\bf\color{red}Cite sth}. Further, the requirement that $a^\dagger$ is a complex conjugation of $a$ leads to the block structure of S
\begin{eqnarray}\begin{aligned}
\label{Block of S}
S=\left(\begin{array}{ccc}
u & v^*\\
v & u^*
\end{array}\right).
\end{aligned}\end{eqnarray}
The symplectic property of S implies the following constraints 
\begin{eqnarray}
u^\dagger u-v^\dagger v&={\bf I}\label{Constraint 1}\\
u^T u-v^T v&=0\label{Constraint 2}
\end{eqnarray}
which will be important for calculation of the ground state. 

For the ground state of the quadratic Hamiltonian in Eq.(\ref{Boson BdG}), it is the vacuum of the annihilation operators defined in Eq.(\ref{Def. of a}) and satisfies
\begin{eqnarray}\begin{aligned}
\label{a Vacuum condition}
(b_iu_{ij}+b_i^\dagger v_{ij})|0\rangle_{{\bf a}}=0.
\end{aligned}\end{eqnarray}
If the matrix $u$ is invertible, then we can introduce a matrix
\begin{eqnarray}\begin{aligned}
T=vu^{-1}
\end{aligned}\end{eqnarray}
such that Eq.(\ref{a Vacuum condition}) reduces to
\begin{eqnarray}\begin{aligned}
(b_i+b_j^\dagger T_{ji})|0\rangle_{{\bf a}}=0,
\end{aligned}\end{eqnarray}
One can use Eq.(\ref{Constraint 2}) to show that $T$ is symmetric, and observe that
\begin{eqnarray}\begin{aligned}
\exp\left\{-\frac{1}{2}b_j^\dagger T_{jk}b_k^\dagger\right\}b_i\exp\left\{\frac{1}{2}b_j^\dagger T_{jk}b_k^\dagger\right\}=b_i+T_{ij}b^\dagger_j.
\end{aligned}\end{eqnarray}
Thus we conclude that 
\begin{eqnarray}\begin{aligned}
\label{Boson BCS gnd}
|0\rangle_{{\bf a}}&=\text{det}^{\frac{1}{4}}(1-T^\dagger T)\exp\left\{-\frac{1}{2}b_j^\dagger T_{jk}b_k^\dagger\right\}|0\rangle_{\bf b}\\
&=|\text{det}(u)|^{-\frac{1}{2}}\exp\left\{-\frac{1}{2}b_j^\dagger T_{jk}b_k^\dagger\right\}|0\rangle_{\bf b}.
\end{aligned}\end{eqnarray}
where we have used Eq.(\ref{Constraint 1}) in the second line. From the following McMahon master theorem
\begin{eqnarray}\begin{aligned}\label{Bosonic McMahon}
\langle0|\exp\left\{\frac{1}{2}b_iX_{ij}b_j\right\}\exp\left\{\frac{1}{2}b^\dagger_iY_{ij}b^\dagger_j\right\}|0\rangle=\text{det}^{-\frac{1}{2}}(1-XY)
\end{aligned}\end{eqnarray}
one can easily check that the ground state in Eq.(\ref{Boson BCS gnd}) is correctly normalized. 

{\bf\color{red} 1. Comment of BCS; 2. Say sth of deferment of the discussion of how to find S explicitly; 3. }

\subsection{Boson fidelity} 
\label{Boson fidelity}

Fidelity is defined as the overlap of groundstates of two different bosonic Hamiltonians{\bf\color{red}cite sth}. We start with a quadratic bosonic Hamiltonian $\hat{H}_0$ in ${\bf b}$ basis, as in Eq.(\ref{Quadratic boson H}). From the discussion in App.{\bf\color{red}where}, we are able to diagonalize it in ${\bf a}$ basis. At $t=0$, the Hamiltonian becomes $\hat{H}_1$, which is still in ${\bf b}$ basis, but is diagonalized in a new ${\bf c}$ basis. The corresponding Bogoliubov transformations read
\begin{eqnarray}\begin{aligned}
\label{Two BdG}
({\bf b} , {\bf b}^\dagger)S_0=({\bf a} , {\bf a}^\dagger)\quad
({\bf b} , {\bf b}^\dagger)S_1=({\bf c} , {\bf c}^\dagger)
\end{aligned}\end{eqnarray}
such that
\begin{eqnarray}\begin{aligned}
\label{S0invS}
({\bf a} , {\bf a}^\dagger)S_0^{-1}S_1=({\bf c} , {\bf c}^\dagger)
\end{aligned}\end{eqnarray}
One immediately realizes Eq.(\ref{S0invS}) takes the same form as Eq.(\ref{Def. of a}) such that $S_0^{-1}S_1$ is in the same symplectic group, and takes the following block form
\begin{eqnarray}
S_0^{-1}S_1=\left(\begin{array}{ccc}
u_1 & v_1^*\\
v_1 & u_1^*
\end{array}\right).
\end{eqnarray}
Thus $|0\rangle_{\bf c}$ is related to the $|0\rangle_{{\bf a}}$ in the same way as in Eq.(\ref{Boson BCS gnd}), and their overlap, which is the fidelity, reads
\begin{eqnarray}\begin{aligned}
|_{\bf a}\langle0|0\rangle_{\bf c}|^2=\frac{1}{|\text{det}(u_1)|}
\end{aligned}\end{eqnarray}

\subsection{Boson Loschmidt echo}

The Loschmidt echo is defined as the overlap of the evolved state
\begin{eqnarray}\begin{aligned}
|0\rangle_{{\bf a}(t)}\equiv e^{-i\hat{H}_1t}|0\rangle_{\bf a}
\end{aligned}\end{eqnarray}
with the $\hat{H}_0$ ground state $|0\rangle_{\bf a}$. We introduce the third dynamical basis
\begin{eqnarray}\begin{aligned}
a_i(t)&=e^{-i\hat{H}_1t}a_ie^{i\hat{H}_1t}
\end{aligned}\end{eqnarray}
which annihilate the evolved state at time t: $a_i(t)|0\rangle_{{\bf a}(t)}=0$. Upon using the expression $\hat{H}_1=\sum_iE_ic_i^\dagger c_i$, the Bogoliubov transformation at time t can be represented as a chain of symplectic transformation
\begin{eqnarray}\begin{aligned}
\label{BdG transformation for echo}
({\bf a}(t),{\bf a}^\dagger(t))=e^{-iHt}
({\bf a} , {\bf a}^\dagger)e^{iHt}=({\bf a},{\bf a}^\dagger)S_0^{-1}S_1\text{diag}(e^{iEt},e^{-iEt})S_1^{-1}S_0
\end{aligned}\end{eqnarray}
It is evident that the evolved state $|0\rangle_{{\bf a}(t)}$ is related to the $|0\rangle_{{\bf a}}$ via the same way as in Eq.(\ref{Boson BCS gnd}). In order to calculate the Loschmidt echo, we simply take the left top corner of the Bogoliubov transformation in Eq.(\ref{BdG transformation for echo}), and it {\bf\color{red}reads}
\begin{eqnarray}\begin{aligned}
\mathcal{L}(t)=|_{\bf a}\langle0|0\rangle_{{\bf a}(t)}|^2=\frac{1}{|\text{det}(u_1^\dagger e^{iEt}u_1-v_1^\dagger e^{-iEt}v_1)|}
\end{aligned}\end{eqnarray}

\subsection{Description on how to get S}

As demonstrated in App.{\bf\color{red}where}, the calculations of fidelity and echo boil down to calculate the Bogoliubov transformations in Eq.(\ref{Two BdG}). Here, for a quadratic bosonic Hamiltonian
\begin{eqnarray}\begin{aligned}
\label{Quadratic boson H2}
\hat{H}&=\frac{1}{2}
\left(\begin{array}{ccc}
{\bf b}^\dagger & -{\bf b}
\end{array}\right)
\left(\begin{array}{ccc}
H & \Delta\\
-\Delta^\dagger & -H^T
\end{array}\right)
\left(\begin{array}{ccc}
{\bf b} \\
{\bf b}^\dagger
\end{array}\right)\\
&=\frac{1}{2}
\left(\begin{array}{ccc}
{\bf b}^\dagger & -{\bf b}
\end{array}\right)
M
\left(\begin{array}{ccc}
{\bf b} \\
{\bf b}^\dagger
\end{array}\right)
\end{aligned}\end{eqnarray}
we shall describe how to find such a Bogoliubov transformation S$\in$Sp(2N,{\bf C}) that diagonalizes M. The basic idea is {\bf\color{red}what}.

We introduce the real basis 
\begin{eqnarray}\begin{aligned}
\label{Real basis}
\left(\begin{array}{ccc}
{\bf b}\\
{\bf b}^\dagger
\end{array}\right)
=C\left(\begin{array}{ccc}
\phi\\
\pi
\end{array}\right)
=\frac{1}{\sqrt{2}}\left(\begin{array}{ccc}
1 & i \\
1 & -i
\end{array}\right)\left(\begin{array}{ccc}
\phi\\
\pi
\end{array}\right)
\end{aligned}\end{eqnarray}
in which the Hamiltonian reads
\begin{eqnarray}\begin{aligned}
\hat{H}
&=\frac{1}{2}
\left(\begin{array}{ccc}
\phi & \pi
\end{array}\right)
\left(\begin{array}{ccc}
\text{Re}(H+\Delta) & -\text{Im}(H-\Delta)\\
\text{Im}(H+\Delta)& \text{Re}(H-\Delta) \\
\end{array}\right)
\left(\begin{array}{ccc}
\phi\\
\pi
\end{array}\right)
=\frac{1}{2}
\left(\begin{array}{ccc}
\phi & \pi
\end{array}\right)
\mathcal{M}
\left(\begin{array}{ccc}
\phi\\
\pi
\end{array}\right)
\end{aligned}\end{eqnarray}
It is not hard to realize $\mathcal{M}$ is real and symmetric. We shall further assume that $\mathcal{M}$ is positive definite, and from Williamson's theorem{\bf\color{red}cite sth} we have
\begin{eqnarray}\begin{aligned}
\mathcal{M}=\tilde{S}\left(\begin{array}{ccc}
d\\
&d
\end{array}\right)\tilde{S}^T
\end{aligned}\end{eqnarray}
where $\tilde{S}\in$Sp(2N,{\bf R}) and the diagonal matrix $d$ are positive eigenvalues of $iJ\mathcal{M}$. Thus we have
\begin{eqnarray}\begin{aligned}
\label{Diagonalization of M}
M=J(C^{-1})^T\tilde{S}C^TJ^{-1}\left(\begin{array}{ccc}
d\\
&-d
\end{array}\right)C\tilde{S}^TC^{-1}.
\end{aligned}\end{eqnarray}
Using the fact that $\tilde{S}$ is an element of real symplectic group and the explicit form of C-matrix, one can show that Eq.(\ref{Diagonalization of M}) is a similarity transformation, and $C\tilde{S}^TC^{-1}$, which takes the block form as in Eq.(\ref{Block of S}), is an element of complex symplectic group. Thus we identify 
\begin{eqnarray*}\begin{aligned}
S\equiv C\tilde{S}^TC^{-1}
\end{aligned}\end{eqnarray*}
as the desired Bogoliubov transformation that diagonalizes the Hamiltonian in Eq.(\ref{Quadratic boson H2}). 

Notice in the above construction, we have assumed the Hamiltonian in the real basis to be positive definite. {\bf\color{red}This is not as restrictive as it may appear}, and in App.{\bf\color{red}where} we shall explicitly construct the matrix $\tilde{S}$ for the case of harmonic chain.
\subsection{Harmonic chain}

{\bf\color{red}Say sth on harmonic chain}

In the real basis defined in Eq.(\ref{Real basis}), the Hamiltonian for 1D harmonic chain reads
\begin{eqnarray}\begin{aligned}
\hat{H}
=\frac{1}{2}
\left(\begin{array}{ccc}
\phi & \pi
\end{array}\right)
\mathcal{M}
\left(\begin{array}{ccc}
\phi\\
\pi
\end{array}\right)
=\frac{1}{2}
\left(\begin{array}{ccc}
\phi & \pi
\end{array}\right)
\left(\begin{array}{ccc}
\mathcal{V} \\
& {\bf I}
\end{array}\right)
\left(\begin{array}{ccc}
\phi\\
\pi
\end{array}\right)
\end{aligned}\end{eqnarray}
where $\mathcal{V}$ is real symmetric matrix that can be diagonalized as $\mathcal{V}=\mathcal{O}D^2\mathcal{O}^T$. We construct the following real symplectic matrix
\begin{eqnarray}\begin{aligned}
\tilde{S}\equiv\left(\begin{array}{ccc}
\mathcal{O}D^{1/2} \\
& \mathcal{O}D^{-1/2}
\end{array}\right)
\end{aligned}\end{eqnarray}
such that 
\begin{eqnarray}\begin{aligned}
\mathcal{M}=\tilde{S}\left(\begin{array}{ccc}
D \\ 
& D
\end{array}\right)
\tilde{S}^T
\end{aligned}\end{eqnarray}
in accordance with Williamson's theorem. Thus the second quantized Hamiltonian will be diagonalized as
\begin{eqnarray}\begin{aligned}
M=S^{-1}\left(\begin{array}{ccc}
D \\
& -D
\end{array}\right)S
\end{aligned}\end{eqnarray}
where the Bogoliubov transformation take the desired block form
\begin{eqnarray}\begin{aligned}
S&=C\tilde{S}C^{-1}
&=\left(\begin{array}{ccc}
O(D^{1/2}+D^{-1/2}) & O(D^{1/2}-D^{-1/2}) \\
O(D^{1/2}-D^{-1/2}) & O(D^{1/2}+D^{-1/2}) 
\end{array}\right)
\end{aligned}\end{eqnarray}


