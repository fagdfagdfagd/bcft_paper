
In this appendix, we provide technical details about the numerical calculation of the bipartite fidelity and Loschmidt echo.  Our strategy takes advantage of the symplectic structure of the bosonic Bogoliubov transformation and explicitly construct the "BCS" like ground state. With slight modification\cite{blaizot_quantum_1986}, one can work out its fermionic version and apply to quadratic fermion models for example in Ref.~\onlinecite{vasseur_universal_2014,stephan_local_2011}.

During the course of derivation in this and other appendices, we will repeatedly use the combinatorial identity called McMahon Master theorem
\begin{equation}
\label{eq:bosonic_McMahon}
\langle0|\exp\left\{\frac{1}{2}b_iX_{ij}b_j\right\}\exp\left\{\frac{1}{2}b^\dagger_iY_{ij}b^\dagger_j\right\}|0\rangle=\text{det}^{-\frac{1}{2}}(1-XY)
\end{equation}
for symmetric matrix $X$ and $Y$ and set of independent Bosonic creation operators $b_i^{\dagger}$. One can prove it for (simultaneously) diagonalizable matrices and then claim its legitimacy for its combinatorial nature. 

\subsection{Boson Bogoliubov transformation}
\label{app_sub:boson_BdG}
We consider the following quadratic bosonic Hamiltonian
\begin{equation}
\begin{aligned}
\label{eq:quadratic_boson_H}
\hat{H} &= \frac{1}{2} (\vec{b}^{\dagger}, -\vec{b}) M 
\begin{pmatrix}
\vec{b}\\
\vec{b}^{\dagger} 
\end{pmatrix},  \qquad 
M = 
\begin{bmatrix}
A & -B^* \\
B & -A^* \\
\end{bmatrix}
\end{aligned}
\end{equation}
where ${\bf b}\equiv(b_{1},...,b_{n})^T$ is a vector of bosonic annihilation operators. The matrix $M$ consists of a $n\times n$ Hermitian block $A$ which plays the role of single particle Hamiltonian in the fermionic case and symmetric block $B$ of the pairing interaction. 

We want to do a Bogoliubov transformation, which uses a $2n \times 2n$ matrix $S$ to define a diagonal basis 
\begin{equation}\begin{aligned}
\label{eq:def_a}
({\bf a} , {\bf a}^\dagger)\equiv({\bf b} , {\bf b}^\dagger)S
\end{aligned}\end{equation}
of the Hamiltonian. The transformation is canonical, meaning that it preserves the commutation relation 
\begin{equation}
\begin{aligned}
\label{eq:preserve_commutator}
J&\equiv\begin{bmatrix}
0 & \mathbb{I}\\
-\mathbb{I} & 0
\end{bmatrix}
=[
\begin{pmatrix}
{\bf a} \\
{\bf a}^\dagger
\end{pmatrix},
\begin{pmatrix}
{\bf a} & {\bf a}^\dagger
\end{pmatrix}]
=S^T[
\begin{pmatrix}
{\bf b} \\
{\bf b}^\dagger
\end{pmatrix},
\begin{pmatrix}
{\bf b} & {\bf b}^\dagger
\end{pmatrix}]S \\
&=S^T JS.
\end{aligned}
\end{equation}
where we have used the compact notation of the sort $([\vec{b}, \vec{b}^{\dagger}])_{ij} =  [b_i, b_j^\dagger]$ to denote the commutator matrix. The appearance of $J$ makes the symplectic nature of the problem manifest and we find $S$ is in the symplectic group ${\rm Sp}( 2n, \mathbb{C} ) $\cite{blaizot_quantum_1986,fulton_representation_2004}. Furthermore, the requirement that $a^\dagger$ is a complex conjugation of $a$ leads to the block structure of $S$
\begin{equation}
\label{eq:block_S}
S=
\begin{bmatrix}
u & v^*\\
v & u^*
\end{bmatrix}.
\end{equation}
And the blocks are constrained by the symplectic property
\begin{eqnarray}
  u^\dagger u-v^\dagger v&=\mathbb{I}\label{eq:constraint_1}\\
  u^T u-v^T v&=0\label{eq:constraint_2}  
\end{eqnarray}
With these conditions, the Hamiltonian in basis $a$ becomes (the use of $({\bf b}^\dagger , -{\bf b})$ rather than $({\bf b}^\dagger , {\bf b})$ can be appreciated in this step)
\begin{equation}
H = \frac{1}{2} ( a^{\dagger}, -a )  (S^{\top} M (S^{\top})^{-1} )
\begin{pmatrix}
a\\
a^{\dagger} 
\end{pmatrix}.
\end{equation}
Quiet unusually, the diagonalization is performed by the symplectic group element.

To proceed, we introduce the real basis 
\begin{equation}\begin{aligned}
\label{eq:real_basis}
\begin{pmatrix}
{\bf b}\\
{\bf b}^\dagger
\end{pmatrix}
=C\begin{pmatrix}
\phi\\
\pi
\end{pmatrix}
=\frac{1}{\sqrt{2}}\begin{bmatrix}
1 & i \\
1 & -i
\end{bmatrix}\begin{pmatrix}
\phi\\
\pi
\end{pmatrix}
\end{aligned}\end{equation}
in which the Hamiltonian reads
\begin{equation}\begin{aligned}
\hat{H}
&=\frac{1}{2}
\begin{pmatrix}
\phi & \pi
\end{pmatrix}
\begin{bmatrix}
\text{Re}(A-B^*) & -\text{Im}(A)+\text{Im}(B)\\
\text{Im}(A)+\text{Im}(B)& \text{Re}(A+B) \\
\end{bmatrix}
\begin{pmatrix}
\phi\\
\pi
\end{pmatrix} \\
&=\frac{1}{2}
\begin{pmatrix}
\phi & \pi
\end{pmatrix}
\mathcal{M}
\begin{pmatrix}
\phi\\
\pi
\end{pmatrix}
\end{aligned}\end{equation}
It is not hard to check that $\mathcal{M}$ is real and symmetric. 

The general solution diagonalization problem is hard\cite{arnold_mathematical_2010}, however the positive definite $\mathcal{M}$ (and hence $M$) can be solved by Williamson's theorem\cite{arnold_mathematical_2010,xiao_theory_2009,pirandola_correlation_2009,gosson_symplectic_2006}, which states the existence, uniqueness (up to reordering of eigenvalues) and explicit construction of the matrix $\mathcal{S}\in {\rm Sp}(2n \mathbb{R}) $ such that 
\begin{equation}\begin{aligned}
\mathcal{M}=\mathcal{S}
\begin{bmatrix}
\, d \, & \\
 & \, d\, \\
\end{bmatrix}
\mathcal{S}^T
\end{aligned}\end{equation}
where the diagonal matrix $d$ are positive eigenvalues of $iJ\mathcal{M}$. After some algebra, we have
\begin{equation}\begin{aligned}
\label{eq:diagonalization_M}
M=J(C^{-1})^T\mathcal{S}C^TJ^{-1}
\begin{bmatrix}
d\\
&-d
\end{bmatrix}
C\mathcal{S}^TC^{-1}.
\end{aligned}\end{equation}
One can show that
\begin{equation*}\begin{aligned}
S\equiv C\mathcal{S}^TC^{-1}
\end{aligned}\end{equation*}
is the required symplectic matrix in the complex basis. 

We will not elaborate on Williamson's theorem and its proof (see proofs in Ref.~\onlinecite{xiao_theory_2009,pirandola_correlation_2009,gosson_symplectic_2006} and also a recent application in entanglement entropy context\cite{coser_contour_2017}). Instead we will show in App.~\ref{app_sub:harmonic_chain} that for the problem of harmonic chain we are interested in, the diagonalization can be easily done without using the general recipe in the Williamson theorem. 


\subsection{Groundstate in $b$ Basis}

Suppose we have obtained the required matrix $S$, the ground state will be the vacuum of the annihilation operators defined in Eq.~\eqref{eq:def_a} and in $b$ basis it satisfies
\begin{equation}\begin{aligned}
\label{eq:a_vacuum_condition}
(b_iu_{ij}+b_i^\dagger v_{ij})|0\rangle_{{\bf a}}=0.
\end{aligned}\end{equation}
If the matrix $u$ is invertible, then we can introduce a matrix $T=vu^{-1}$ to rewrite Eq.~\eqref{eq:a_vacuum_condition} as
\begin{equation}\begin{aligned}
(b_i+b_j^\dagger T_{ji})|0\rangle_{{\bf a}}=0. 
\end{aligned}\end{equation}
The constraint Eq.~\eqref{eq:constraint_2} on the blocks of $u$ and $v$ (followed by the symplectic constraint of $S$) implies that $T$ is a symmetric matrix. With the observation of 
\begin{equation}\begin{aligned}
\exp\left\{-\frac{1}{2}b_j^\dagger T_{jk}b_k^\dagger\right\}b_i\exp\left\{\frac{1}{2}b_j^\dagger T_{jk}b_k^\dagger\right\}=b_i+T_{ij}b^\dagger_j,
\end{aligned}\end{equation}
we solve the groundstate
\begin{equation}\begin{aligned}
\label{eq:boson_BCS_gnd}
|0\rangle_{{\bf a}}&=\text{det}^{\frac{1}{4}}(1-T^\dagger T)\exp\left\{-\frac{1}{2}b_j^\dagger T_{jk}b_k^\dagger\right\}|0\rangle_{\bf b}\\
\end{aligned}\end{equation}
where the normalization is given by the McMahon Master theorem Eq.~\ref{eq:bosonic_McMahon}. Apply constraint in Eq.~\eqref{eq:constraint_1}, it simplifies to the top left corner of the symplectic matrix
\begin{equation}
\text{det}^{\frac{1}{4}}(1-T^\dagger T) =|\text{det}(u)|^{-\frac{1}{2}}
\end{equation}

Eq.~\eqref{eq:boson_BCS_gnd} takes a similar form as the superconducting ground state, with the pairing wavefunction $T_{ij}$ determined by the Bogoliubov transformation. In the next section, we will see that the normalization factor gives the fidelity and Loschmidt echo.

\subsection{Boson fidelity} 
\label{app_sub:boson_fidelity}

Fidelity is defined as the (squared) overlap of groundstates of two different bosonic Hamiltonians. 

We start with a quadratic bosonic Hamiltonian $\hat{H}_0$ in ${\bf b}$ basis, as in Eq.~\eqref{eq:quadratic_boson_H}. From the discussion in App.~\ref{app_sub:boson_BdG}, we are able to diagonalize it in ${\bf a}$ basis for positive definite $M$. At $t=0$, the Hamiltonian becomes $\hat{H}_1$, which is still written in ${\bf b}$ basis, but is diagonalized in a new basis ${\bf c}$. The corresponding Bogoliubov transformations read
\begin{equation}\begin{aligned}
\label{eq:two_BdG}
({\bf b} , {\bf b}^\dagger)S_0=({\bf a} , {\bf a}^\dagger)\quad
({\bf b} , {\bf b}^\dagger)S_1=({\bf c} , {\bf c}^\dagger)
\end{aligned}\end{equation}
and so
\begin{equation}\begin{aligned}
\label{eq:S0invS}
({\bf a} , {\bf a}^\dagger)=({\bf c} , {\bf c}^\dagger)\left(S_0^{-1}S_1\right)^{-1}
\end{aligned}\end{equation}
One realizes that Eq.~\eqref{eq:S0invS} is another Bogoliubov transformation and so the corresponding matrix has the block structure
\begin{equation}
S_1^{-1}S_0=\begin{bmatrix}
u_1 & v_1^*\\
v_1 & u_1^*
\end{bmatrix}.
\end{equation}
Thus $|0\rangle_{\bf c}$ is related to the $|0\rangle_{{\bf a}}$ in the same way as in Eq.~\eqref{eq:boson_BCS_gnd}. Their overlap is therefore given by the normalization factor
\begin{equation}\begin{aligned}
|{}_{\bf a}\langle0|0\rangle_{\bf c}|^2=\frac{\Big|{}_{\bf a}\langle 0 | \exp( -\frac{1}{2} a_j^{\dagger} T^{jk} a_k^{\dagger} )|0   \rangle_{\bf a} \Big|^2}{|\text{det}(u_1)|} = \frac{1}{|\text{det}(u_1)|}
\end{aligned}\end{equation}

\subsection{Boson Loschmidt echo}
\label{app_sub:boson_Loschmidt_echo}

The Loschmidt echo is defined as the (squared) overlap of the evolved state
\begin{equation}\begin{aligned}
|0\rangle_{{\bf a}(t)}\equiv e^{-i\hat{H}_1t}|0\rangle_{\bf a}
\end{aligned}\end{equation}
with $|0\rangle_{\bf a}$ the ground state of the Hamiltonian $\hat{H}_0$ before the quench. We introduce a dynamical basis
\begin{equation}\begin{aligned}
a_i(t)&=e^{-i\hat{H}_1t}a_ie^{i\hat{H}_1t}
\end{aligned}\end{equation}
which annihilate the evolved state at time $t$: $a_i(t)|0\rangle_{{\bf a}(t)}=0$. Upon using the diagonal basis $\hat{H}_1=\sum_iE_ic_i^\dagger c_i$, the Bogoliubov transformation at time $t$ can be represented as a chain of symplectic transformation
\begin{equation}\begin{aligned}
\label{eq:BdG_transformation_echo}
&({\bf a}(t),{\bf a}^\dagger(t))=e^{-iHt}
({\bf a} , {\bf a}^\dagger)e^{iHt} \\
=&({\bf a},{\bf a}^\dagger)S_0^{-1}S_1\text{diag}(e^{iEt},e^{-iEt})S_1^{-1}S_0
\end{aligned}\end{equation}
It is evident that the evolved state $|0\rangle_{{\bf a}(t)}$ is related to the $|0\rangle_{{\bf a}}$ in the same way as in Eq.~\eqref{eq:boson_BCS_gnd}. The overlap, as we have seen in the fidelity case, is the normalization factor of the "BCS" ground state. It is related to the top left block of the Bogoliubov transformation in Eq.~\eqref{eq:BdG_transformation_echo},
\begin{equation}\begin{aligned}
\label{eq:echo_t}
\mathcal{L}(t)=|_{\bf a}\langle0|0\rangle_{{\bf a}(t)}|^2=|\text{det}(u_1^\dagger e^{iEt}u_1-v_1^\dagger e^{-iEt}v_1)|^{-1}
\end{aligned}\end{equation}

\subsection{Harmonic chain} 
\label{app_sub:harmonic_chain}

In this subsection, we explicitly construct the matrix $\mathcal{S}$ for the case of harmonic chain introduced in Sec.~\ref{sec_sub:free_boson_lattice}. In the basis defined in Eq.~\eqref{eq:real_basis}, the Hamiltonian for 1D harmonic chain reads
\begin{equation}\begin{aligned}
\hat{H}
=\frac{1}{2}
\begin{pmatrix}
\phi & \pi
\end{pmatrix}
\mathcal{M}
\begin{pmatrix}
\phi\\
\pi
\end{pmatrix}
=\frac{1}{2}
\begin{pmatrix}
\phi & \pi
\end{pmatrix}
\begin{bmatrix}
\mathcal{V} \\
& {\mathbb{ I}}
\end{bmatrix}
\begin{pmatrix}
\phi\\
\pi
\end{pmatrix}
\end{aligned}\end{equation}
where $\mathcal{V}$ is real symmetric matrix that can be diagonalized as $\mathcal{V}=\mathcal{O}D^2\mathcal{O}^T$. The matrix $\mathcal{V}$ depends on the boundary condition, but positive definiteness is the only requirement here. 

The matrix $\mathcal{S}$ that diagonalize $\mathcal{M}$
\begin{equation}
\begin{aligned}
\mathcal{M}=\mathcal{S}
\begin{bmatrix}
D \\ 
& D
\end{bmatrix}
\mathcal{S}^T
\end{aligned}
\end{equation}
is given by the following real symplectic matrix
\begin{equation}\begin{aligned}
\mathcal{S}\equiv
\begin{bmatrix}
\mathcal{O}D^{1/2} \\
& \mathcal{O}D^{-1/2}
\end{bmatrix}
\end{aligned}\end{equation}


Thus the Hamiltonian is diagonalized as
\begin{equation}\begin{aligned}
M=S^{-1}
\begin{bmatrix}
D \\
& -D
\end{bmatrix}
S
\end{aligned}\end{equation}
where the Bogoliubov transformation take the desired block form
\begin{equation}\begin{aligned}
S&=C\mathcal{S}C^{-1}
=\begin{bmatrix}
O(D^{1/2}+D^{-1/2}) & O(D^{1/2}-D^{-1/2}) \\
O(D^{1/2}-D^{-1/2}) & O(D^{1/2}+D^{-1/2}) 
\end{bmatrix}
\end{aligned}\end{equation}


%%% Local Variables:
%%% TeX-master: "bCFT_paper"
%%% TeX-PDF-mode: t
%%% End:
