In order to check our analytic results, we reconsider the lattice model introduced in Sec.~\ref{sec_sub:free_boson_lattice}. In particular, we calculate the Loschmidt echo and bipartite fidelity for various joining boundary conditions determined by $S_1(\theta)$ or $S_2(\theta)$. We refer the reader to App.~\ref{app:comp_fid_echo} for the technical details. 

\begin{figure}[h]
\includegraphics[width=1\columnwidth]{DDDD_fit.pdf}
\caption{The slope of the free energy for Loschmidt echo in the process $\text{DD}+\text{DD}\rightarrow\lambda+\text{DD}$, with gluing condition $S_1(\theta)$ where $\theta\in(0,\frac{\pi}{2})$. We work with total system size $N=30k$ sites, with parameters $m=10^{-8}$,$k=1$. The lattice constant is set to unity. The blue dots are the numerical results and the red line is the analytical result. As predicted, the slopes are equal for different values of $\theta$. Inset: An example of Loschmidt echo with $\theta=0.02\pi$ shown in log scale. The dashline denotes the power law with $t^{-0.25}$. The curve fitting method is described in the main text.}
\label{fig:DDDD}
\end{figure}

We start with the process $\text{DD}+\text{DD}\rightarrow\lambda+\text{DD}$ and the numerical simulation is performed for $\theta=0.02n\pi$, $n=1,...,25$. The results for Loschmidt echo is shown in Fig.~\ref{fig:DDDD}, and numerical parameters are given in the caption. The green dots correspond to numerical results for the slope of free energy, and the solid line shows the analytical result for echo in Eq.~\eqref{eq:result_DDDD}. The inset shows the double logarithmic plot of Loschmidt echo versus time for the case $\theta=0.02\pi$. The dashline denotes the expected power law decay $\mathcal{L}(t)\sim t^{-0.25}$. For the curve fitting, we randomly pick a data point near $t=10$ and fit the lower half of the data. This is repeated five times and we calculate the mean and standard deviation for the exponents. With the data processing method described, the error bar in the main figure is invisible on the data points. We have performed the identical simulation for the process $\text{NN}+\text{NN}\rightarrow\lambda+\text{NN}$ and obtained identical result shown in Fig.~\ref{fig:DDDD} (the initial decay of Loschmidt is different but has the same exponent at larger time). We note one caveat for this case with Neumann boundary conditions on both sides of the chain. Because of the zero mode, it is expected (indeed we saw) that the numerical simulation may not be stable. The problem can be avoided by adding a small mass regulator $m=10^{-8}$. Similar numerical analysis can be applied to calculate the fidelity which matches the result in Eq.~\eqref{eq:result_DDDD}. We shall not show the result here. 

Next we analyze the process $\text{DN}+\text{DN}\rightarrow\lambda+\text{DN}$ in which the joining boundary condition is determined by $S_1(\theta)$. There is one subtly which is that $S_1(0)$ corresponds to `DN' and the process reduces to $\text{DN}+\text{DN}\rightarrow\text{DN}+\text{DN}$. The zero mode will cause the numerical simulation unstable near $\theta=0$, and it turns out that the mass regulator cannot resolve this problem. Rather, we shall shift one of the `DN' slightly and consider the following process
\begin{eqnarray}\begin{aligned}
S_1(0)+S_1(\delta\theta)\rightarrow S_1(\theta)+S_1(\delta\theta)
\end{aligned}\end{eqnarray}
where we take $\delta\theta=0.003\pi$ and $\theta=0.02n\pi$, $n=1,...,25$. We shall discuss the introduction of extra bcc at $-\infty$ due to this shift in Sec.~\ref{sec:disc}. We show the numerical result for Loschmidt echo in Fig.~\ref{fig:DDNN}(a). The slope of the free energy follows a quadratic relation as predicted in Eq.~\eqref{eq:result_DNDN}. The deviation from the analytic result (without the shift) near $\theta=0$ has been greatly suppressed because of $\delta\theta$ introduced manually. The numerical results near $\theta=0.5\pi$ will match the analytical ones more precisely if we were to set $\delta\theta=0$. 

Next we perform the numerical simulation for $\text{DN}+\text{DN}\rightarrow\lambda+\text{DN}$ with $\theta\in(0,\frac{\pi}{2})$.  We recall that  

{\color{red}say why echo bad, say we shift it that is why dot below curve slightly near pi, say there is extra bcc, say why fidelity better, say to confirm we do fidelity as shown in Fig.~\ref{fig:DDNN}(b) and it is good, confirm our conjecture. }

We further consider the case $\text{P}+\text{P}\rightarrow\lambda+\text{P}$, and the numerical results for Loschmidt echo and fidelity are shown in Fig.~\ref{fig:PPPP}. 


\begin{figure}
  \centering
\includegraphics[width=1\columnwidth]{DDNN_fit.pdf}
\includegraphics[width=1\columnwidth]{DN_DN2tan.pdf}
    \caption{The slope of the free energy for Loschmidt echo (a) and bipartite fidelity (b) in the process $\text{DN}+\text{DN}\rightarrow\lambda+\text{DN}$. The total system size is $N=30k$ sites with the same parameters in Fig.~\ref{fig:DDDD}. The numerical value of slopes follow a quadratic relation as predicted in Eq.~\eqref{eq:result_DNDN}. For the deviation from the analytic results in (a), see the discussion in the main text. Inset in (a): From the top to bottom, we show the power law decay of Loschmidt echo with $\theta=0.02\pi, 0.12\pi,0.24\pi$ and $0.48\pi$. We use the same curve fitting method as described for Fig.~\ref{fig:DDDD}.}
      \label{fig:DDNN}
    \todo[inline]{Tianci: make the plots have the same style?}
\end{figure}


\begin{figure}
  \centering
  \includegraphics[width=1\columnwidth]{p_p2tan.pdf}
    \includegraphics[width=1\columnwidth]{p_p2tan.pdf}
    \caption{daf}
    \label{fig:PPPP}
    \todo[inline]{Mao: Put the PPPP for echo; Tianci: make the plots have the same style?}
\end{figure}

