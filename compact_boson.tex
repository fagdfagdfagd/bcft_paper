
In this appendix, we address the issue of winding modes of compact bosons. In the main text, we have exclusively worked with the oscillator modes of free bosons. Here, we shall show that winding modes for compactified bosons will have no contribution to the fidelity or Loschmidt echo in the leading order. Therefore, our results are ready to be applied in the case where two compactified bosons of different radii are connected by a conformal interface\cite{PhysRevLett.118.136801}. \todo[inline]{say intuition why not matter, mention application to Laughlin, etc.}

Our derivation follows the general multi-component Boson constraints in Ref.~\onlinecite{affleck_quantum_2001,oshikawa_boundary_2010}. A review of the detailed parameterization of the states can be found in Ref.~\onlinecite{sakai_entanglement_2008}. 

\subsection{Mode Expansion of Compact Boson}
Suppose the boson is compactified as $\phi =  \phi + 2\pi R$, then mode expansion will have zero mode

\begin{widetext}
\begin{equation}
\begin{aligned}
\phi( x, t ) &= \phi_0 - \frac{2\pi m R}{L}x + \frac{\pi_0}{gL} t  + \frac{i}{\sqrt{4 \pi g}}\sum_{n\neq0}\frac{1}{n}(a_ne^{2\pi i n(x-t)/L}+ \bar{a}_{n} e^{2\pi i n(x+t)/L})\\  
&= \phi_0 - \frac{2\pi m R}{L}x + \frac{n}{gLR} t  +  \frac{i}{\sqrt{4 \pi g}}\sum_{n\neq0}\frac{1}{n}(a_ne^{2\pi i n(x-t)/L}+ \bar{a}_{n} e^{2\pi i n(x+t)/L})\\ 
\end{aligned}
\end{equation}
where the conjugate momentum $\pi_0$ takes discrete eigenvalues $\frac{n}{R}$. In the holomorphic coordinates
\begin{equation}
\label{eq:zzbar}
z = \exp( 2 \pi i \frac{t - x}{L}) \qquad \bar{z} = \exp( 2 \pi i \frac{t + x}{L})
\end{equation}
the mode expansion becomes
\begin{equation}
\label{eq:boson-mode-exp}
\begin{aligned}
\phi( z, \bar{z}) = &\phi_0 -i \left( \frac{n}{4\pi g  R} + \frac{m R }{2} \right)  \ln z + \frac{i}{\sqrt{4\pi g}} \sum_{n\ne 0 } \frac{a_n}{n} z^{-n } -i \left( \frac{n}{4\pi g R} - \frac{m R }{2} \right)  \ln \bar{z} + \frac{i}{\sqrt{4\pi g}} \sum_{n\ne 0 } \frac{\bar{a}_n}{n} \bar{z}^{-n } 
\end{aligned}
\end{equation}
\end{widetext}
We further identify 
\begin{equation}
\begin{aligned}
  a_0 &= \sqrt{ 4 \pi g } \left( \frac{n}{4\pi g R} + \frac{m R }{2} \right) \\
   \bar{a}_0 &= \sqrt{ 4 \pi g } \left( \frac{n}{4\pi g R} - \frac{m R }{2} \right)
  \end{aligned}
\end{equation}
and then 
\begin{equation}
\begin{aligned}
i \partial_z \phi =  \sum_n \frac{1}{\sqrt{4\pi g}} a_n z^{-n-1} 
\end{aligned}
\end{equation}

\subsection{Gluing Condition for the Winding Modes}
\label{app_sub:compact_gluing_boundary}

Here we solve the boundary state on the $t = 0$ slice. Our actual boundary state used in the text corresponds to the $x = 0$ slice. There is a $x \rightarrow t$ exchange in the convention here. However, that amounts to the change of $\theta \rightarrow \frac{\pi}{2} - \theta$ in the $S$ matrix argument. 

\todo[inline]{Mao: this is not a T-dual transformation, it is DN -> ND. }
We recall that the gluing condition is written as
\begin{equation}
\begin{aligned}
\label{eq:def_S_in_app}
\begin{pmatrix}
\partial_+\phi^1\\
\partial_-\phi^2
\end{pmatrix}
=S(\theta)
\begin{pmatrix}
\partial_-\phi^1\\
\partial_+\phi^2
\end{pmatrix}
\end{aligned}
\end{equation}

Upon mode expanding $\phi^{1,2}$ in the holomorphic coordinates as in Eq.~\eqref{eq:boson-mode-exp}, we have
\begin{equation}
\begin{aligned}
\label{eq:def_S_in_app_2}
\sum_{n\in\mathbb{Z}}
\begin{pmatrix}
\bar{a}_n^1\bar{z}^{-n}\\
\bar{a}_n^2\bar{z}^{-n}
\end{pmatrix}
=S(\theta)
\sum_{n\in\mathbb{Z}}
\begin{pmatrix}
a_n^1{z}^{-n}\\
a_n^2{z}^{-n}
\end{pmatrix}
\end{aligned}
\end{equation}
As explained above, the boundary condition is imposed at $t=0$, where $\bar{z}=z^{-1}$, therefore we have
\begin{equation}\begin{aligned}
a^i_n-(S^{-1}_{ij})\bar{a}^j_{-n}=0
\end{aligned}\end{equation}
The solution of the $n \ne 0$ constraints is exactly the boundary state in Eq.~\eqref{eq:bd_state}. 

%We recall that the gluing condition can written as
%\begin{equation}\begin{aligned}
%\label{eq:def_M}
%\begin{pmatrix}
%\partial_t\phi^1\\
%\partial_x\phi^1
%\end{pmatrix}_{t=0^-}
%=M\begin{pmatrix}
%\partial_t\phi^2\\
%\partial_x\phi^2
%\end{pmatrix}_{t=0^+}
%\end{aligned}\end{equation}
%where the derivatives are evaluated at the appropriate limit. In this appendix, we shall take the definition $x^\pm=x\pm t$ and $\partial_\pm=\partial_x\pm\partial_t$. Therefore we establish the scattering matrix as
%\begin{equation}
%\begin{aligned}
%\label{eq:def_S_in_app}
%\begin{pmatrix}
%\partial_+\phi^1\\
%\partial_-\phi^2
%\end{pmatrix}
%=S(\theta)
%\begin{pmatrix}
%\partial_-\phi^1\\
%\partial_+\phi^2
%\end{pmatrix}
%\end{aligned}
%\end{equation}
%
%The folding sends $\phi^2(t)\rightarrow\phi^2(-t)$ for $t<0$, and the $\partial_t \phi^2$ term changes sign, we therefore have 
%\begin{equation}
%\begin{aligned}
%\label{eq:def_S_in_app_2}
%\begin{pmatrix}
%\partial_+\phi^1\\
%\partial_+\phi^2
%\end{pmatrix}
%=S(\theta)
%\begin{pmatrix}
%\partial_-\phi^1\\
%\partial_-\phi^2
%\end{pmatrix}
%\end{aligned}
%\end{equation}
%We note that the parameterization of S-matrix is \emph{different} from Eq.~\eqref{eq:def_S}, but this will not change our conclusions in the main text. Upon mode expanding $\phi^{1,2}$ in the holomorphic coordinates as in Eq.~\eqref{eq:boson-mode-exp}, we have
%\begin{equation}
%\begin{aligned}
%\label{eq:def_S_in_app_2}
%\sum_{n\in\mathbb{Z}}
%\begin{pmatrix}
%\bar{a}_n^1\bar{z}^{-n}\\
%\bar{a}_n^2\bar{z}^{-n}
%\end{pmatrix}
%=-S(\theta)
%\sum_{n\in\mathbb{Z}}
%\begin{pmatrix}
%a_n^1{z}^{-n}\\
%a_n^2{z}^{-n}
%\end{pmatrix}
%\end{aligned}
%\end{equation}
%\todo[inline]{Check the sign and parameterization of S}
%As explained above, the boundary condition is imposed at $t=0$, where $\bar{z}=z^{-1}$, therefore we have
%\begin{equation}\begin{aligned}
%a^i_n-(-S^{-1}_{ij})\bar{a}^j_{-n}=0
%\end{aligned}\end{equation}
%The solution of the $n \neq 0$ constraints is exactly the boundary state in Eq.~\eqref{eq:bd_state}, with an extra minus sign. Following our boundary state calculation in the main text, it is clear that this extra minus sign, which is the result of {\bf T}-duality, does not affect our conclusion.

We specialize to $S = S_1$ to solve the $ n = 0$ constraint. We introduce the compactification lattice and its dual\cite{affleck_quantum_2001,oshikawa_boundary_2010}
\begin{equation}
\label{eq:lattice}
\vec{M} = (m_1 2 \pi R_1, m_2 2\pi  R_2)^\top, \quad  \vec{M}^* = (\frac{n_1}{R_1}, \frac{n_2}{R_2})^\top
\end{equation}
to rewrite the zero mode part as 
\begin{equation}
  a_0^i - S^{-1} _{ij} \bar{a}_{0}^j = 0 \,\, \implies \,\, ( \vec{M} + \frac{1}{g}\vec{M}^* ) = S_1 ( -\vec{M} + \frac{1}{g}\vec{M}^* )
\end{equation}
which is basically the multi-component boson winding constraints given in \onlinecite{affleck_quantum_2001,oshikawa_boundary_2010}. The solution gives the interface parameter $\lambda$
\begin{equation}
\begin{aligned}
\label{eq:S_1_constraint}
\lambda = \tan\theta=\frac{n_2R_1}{n_1R_2}=-\frac{m_1R_1}{m_2R_2}
\end{aligned}
\end{equation}
and the conformal boundary state
\begin{equation}\begin{aligned}
\label{eq:S1bd-state}
g_{S_1}\sum_{S_1}e^{in_1\phi_0-im_1\bar{\phi}_0}|n_1,m_1\rangle|n_2,m_2\rangle
\end{aligned}\end{equation}
where $\sum_{S_1}$ is the summation consistent with the constraint in Eq.~\eqref{eq:S_1_constraint}. The g-factor can only be determined by Cardy condition\cite{cardy_boundary_2004}. Since it is not important for what follows, we shall not include the calculation here. 

Since $S = S_2$ is effectively $S_1$ on the dual boson, we can expect that it will end up in the same expression as in Eq.~\eqref{eq:S1bd-state}, but with a different constraints on the winding number
\begin{equation}
\vec{M} = \frac{\cot \theta}{g} 
\begin{bmatrix}
0 & -1\\
1 & 0 \\                                
\end{bmatrix}
\vec{M}^*
\end{equation}



\subsection{Winding Mode Contribution to the Partition Function}
\label{app_sub:winding_contribution}
We now calculate the winding mode part of the partition function as shown in Fig.~\ref{fig:fig_lambda_1_lambda_2}
\begin{equation}\begin{aligned}
Z=\langle S_j( \theta_2 )|e^{-\pi H}|S_i(\theta_1 )\rangle\qquad H=\frac{2\pi}{\beta}(L_0+\bar{L}_0)
\end{aligned}\end{equation}
With boundary state, we can simply replace $L_0$($\bar{L}_0$) with $a_0$($\bar{a}_0$). 

For the amplitude between the same boundary states $\lambda_1=\lambda_2$, we have the winding mode contribution as
\begin{equation}
\begin{aligned}
\label{eq:Z0}
Z_0 \le  \sum_{S_i} g_{\,\!_{S_i} }g_{\,\!_{S_j} } \exp\Big\{- \frac{4\pi}{\beta} 2 \pi g ( \frac{n_1}{ 4 \pi R_1 g} + \frac{m_1 R_1 }{ 2} )^2 \Big\}
\end{aligned}
\end{equation}
where the equality only is only taken when the two boundary states are identical. 

In the limit $\beta\rightarrow\infty$, Eq.~\eqref{eq:Z0} can be approximated by a simple two dimensional integral
\begin{equation}\begin{aligned}
Z_0\approx g_{\,\!_{S_i} }g_{\,\!_{S_j} }\beta\int dxdy\exp\Big\{-8 \pi^2 g ( \frac{x}{ 4 \pi R_1 g} + \frac{y R_1 }{ 2} )^2 \Big\}
\end{aligned}\end{equation}
The winding mode thus can contribute at most a $\ln\beta$ term to the free energy. Compared to the result in App.~\ref{app:gnd_dn_lambda}-\ref{app:lambda_12}, we conclude that the winding mode contribution will not present in the leading order of large $\beta$ limit. 


%%% Local Variables:
%%% TeX-master: "bCFT_paper"
%%% TeX-PDF-mode: t
%%% End:
