
In this section, we consider a lattice model with bosonic interface at the center\cite{peschel_exact_2012,calabrese_entanglement_2012}, which reduces to the one considered in Sec.~\ref{sec_sub:general_formulation} in the continuum limit\cite{sakai_entanglement_2008}. Therefore, it serves as a numerical tool to check our analytic results in Sec.~\ref{sec:analytic_numerics}. 

% Despite its simpleness, such discrete bosonic defect can be considered in more general quadratic bosonic\cite{PhysRevA.74.022329}, or fermionic\cite{PhysRevB.64.064412} systems. It also sheds light on the possible application of our CFT results in condensed matter experiments\cite{Rohringer2015,Trotzky2012}.

\todo[inline]{I think the reference cited here are irrelevant.}

We consider two harmonic chains with bosonic field $\phi_i$ and conjugate momentum $\pi_i$ on lattice site $i$. The left and right chains are connected between site $0$ and $1$ with the Hamiltonian
\begin{eqnarray}\begin{aligned}
H = \frac{1}{2} \sum_i \pi_i^2  +  \frac{1}{2} \sum_{i\ne 0 }  ( \phi_i - \phi_{i+1} )^2  +  \frac{1}{2} \begin{pmatrix}  \phi_0, \phi_1 \end{pmatrix}
\begin{bmatrix}
1 + \Sigma_{11}  & \Sigma_{12} \\
\Sigma_{21} &  1 + \Sigma_{22} \\
\end{bmatrix}
\begin{pmatrix}
\phi_0 \\
\phi_1 
\end{pmatrix}
\end{aligned}\end{eqnarray}
where the matrix $\Sigma$ parameterize two-site interaction. In fact $\Sigma$ gives a simple relation with the scattering matrix defined in Sec.~\ref{sec_sub:general_formulation}, and thus the Hamiltonian is the lattice regularization of the bosonic interface discussed there.

\todo[inline]{Mao: I suggest to relocate this part to an appendix, and maybe more details can be filled in, especially scale invariance condition imposed on the $\Sigma$. In the main text, just cite the relation between $\Sigma$ and $S$ and proceed}

To find the scattering matrix, we set up the plane wave scattering problem across the interface using ansatz for two semi-infinite chain (the use of $(n-1)$ in $\phi_n^B$ simplifies the calculation)
\begin{equation}
\label{eq:ansatz}
\phi_n
= \left\lbrace
  \begin{aligned}
    \phi_n^A &= A_{-} e^{i \omega t  - inka}  + A_{+} e^{i \omega t  + inka}  & \quad  n \le 0 \\
    \phi_n^B &= B_{-} e^{i \omega t  - i(n-1)ka}  + B_{+} e^{i \omega t  + i(n-1)ka} & \quad n \ge 1 \\
  \end{aligned} \right. 
 \quad 
\end{equation}
where $a$ is the lattice constant. The dispersion of the system is $\omega = \left|2\sin\frac{k}{2}\right|$ which implies the system is gapless. The interface will not break the criticality of the harmonic chain\cite{peschel_exact_2012}. Upon plugging ansatz and dispersion relation into the equation of motion, we have the following relation between the incoming and outgoing amplitudes
\begin{equation}
\label{eq:discrete_S}
\begin{pmatrix}
A_{-} \\
B_{+}\\
\end{pmatrix}
=-
\begin{bmatrix}
\Sigma_{11} + e^{-ika} & \Sigma_{12} \\
\Sigma_{21} & \Sigma_{22}  + e^{-ika}  \\
\end{bmatrix}^{-1}
\begin{bmatrix}
\Sigma_{11} +e^{ika} & \Sigma_{12}\\
\Sigma_{21} & \Sigma_{22} + e^{ika}
\end{bmatrix}
\begin{pmatrix}
A_{+}\\
B_{-}\\
\end{pmatrix}
\end{equation}
and the scattering matrix can be solved as
\begin{equation}
  S = \frac{-1}{ \det \Sigma  + e^{-ika} \text{tr} \Sigma   + e^{-2ika}}
\begin{bmatrix}
\det \Sigma+ \Sigma_{11} e^{-ika} + \Sigma_{22} e^{ika}+1  & -2i \sin ka \Sigma_{12}  \\
-2i \sin ka \Sigma_{21} &  \det \Sigma+ \Sigma_{11} e^{ika} + \Sigma_{22} e^{-ika}+1\\
\end{bmatrix}
\end{equation}
A necessary condition for the scattering matrix to be scale invariance is that the ratio $|S_{11}/S_{21}|$ is independent of $k$. This implies 
\begin{equation}
\label{eq:Sigma_condition}
\det \Sigma = -1, \, \, \text{tr} \Sigma = 0
\end{equation}
Therefore, we have the scale invariant S-matrix as
\begin{equation}
S = \frac{1}{1 - e^{-2ika } } ( -2i \sin k ) \Sigma
 = - e^{ika} \Sigma
\end{equation}
In the continuum limit where $a\rightarrow0$, the matrix $\Sigma$ can be parameterized as
\begin{equation}
\Sigma = -\lim_{a \rightarrow 0 } S = 
\begin{bmatrix}
\frac{\lambda^2- 1}{1 + \lambda^2} & \frac{-2\lambda }{1 + \lambda^2} \\
\frac{-2\lambda }{1 + \lambda^2} & \frac{1- \lambda^2}{1 + \lambda^2} \\
\end{bmatrix}
\end{equation}
in accordance to Eq.~\eqref{eq:Sigma_condition}. 

{\bf\color{red}
Say in the continuum limit, what happens

say in the long time, what happens? 

say the special case in the boundary
}

This is the matrix we will use for the numerics. The results will be presented in Sec.~\ref{sec:analytic_numerics}.


%%% Local Variables:
%%% TeX-master: "bCFT_paper"
%%% TeX-PDF-mode: t
%%% End:
