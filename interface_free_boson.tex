
In this appendix, we demonstrate how to realize conformal interface in a lattice harmonic chain defined in Sec.~\ref{sec_sub:free_boson_lattice},
\begin{equation}
\begin{aligned}
H =& \frac{1}{2} \sum_i \pi_i^2  +  \frac{1}{2} \sum_{i\ne 0 }  ( \phi_i - \phi_{i+1} )^2  \\
 &+ \frac{1}{2} \begin{pmatrix}  \phi_0, \phi_1 \end{pmatrix}
\begin{bmatrix}
1 + \Sigma_{11}  & \Sigma_{12} \\
\Sigma_{21} &  1 + \Sigma_{22} \\
\end{bmatrix}
\begin{pmatrix}
  \phi_0 \\
  \phi_1 
\end{pmatrix},
\end{aligned}
\end{equation}
where the matrix $\Sigma$ parameterizes the two-site interaction between site 0 and 1. We set up the plane wave scattering problem across the interface with the following ansatz (the use of $(n-1)$ in $\phi_n^B$ simplifies the calculation)
\begin{equation}
\label{eq:ansatz}
\phi_n
= \left\lbrace
  \begin{aligned}
	& A_{-} e^{i \omega t  - inka}  + A_{+} e^{i \omega t  + inka}  & \quad  n \le 0 \\
	& B_{-} e^{i \omega t  - i(n-1)ka}  + B_{+} e^{i \omega t  + i(n-1)ka} & \quad n \ge 1 ,\\
  \end{aligned} \right. 
 \quad 
\end{equation}
where $a$ is the lattice constant. The solution on both semi-infinite chains are gapless with the dispersion relation $\omega = \left|2\sin\frac{ka}{2}\right|$. The $S$ matrix connecting them can be found by relating the incoming and outgoing amplitudes

\begin{equation}
\label{eq:discrete_S}
\begin{aligned}
\begin{pmatrix}
A_{+} \\
B_{-}\\
\end{pmatrix}
=&-
\begin{bmatrix}
\Sigma_{11} +e^{ika} & \Sigma_{12}\\
\Sigma_{21} & \Sigma_{22} + e^{ika}
\end{bmatrix}^{-1} \\
&\begin{bmatrix}
\Sigma_{11} + e^{-ika} & \Sigma_{12} \\
\Sigma_{21} & \Sigma_{22}  + e^{-ika}  \\
\end{bmatrix}
\begin{pmatrix}
A_{-}\\
B_{+}\\
\end{pmatrix},
\end{aligned}
\end{equation}
and the explicit expression is
\begin{widetext}
\begin{equation}
  S = \frac{-1}{ \det \Sigma  + e^{-ika} \text{tr} \Sigma   + e^{-2ika}}
\begin{bmatrix}
\det \Sigma+ \Sigma_{11} e^{-ika} + \Sigma_{22} e^{ika}+1  & -2i \sin ka \Sigma_{12}  \\
-2i \sin ka \Sigma_{21} &  \det \Sigma+ \Sigma_{11} e^{ika} + \Sigma_{22} e^{-ika}+1\\
\end{bmatrix}.
\end{equation}
\end{widetext}

The reflection and transmission coefficients associated with this interface contained in the $S$ matrix and both of them have to be $k-$independent to form a conformal interface\cite{peschel_exact_2012}. A necessary condition is that $|S_{12}|$ must be $k$-independent. If $\Sigma_{12} = \Sigma_{21} = 0 $, we have 
\begin{equation}
  S_{11} = - \frac{\Sigma_{11} + e^{ika} }{\Sigma_{11} + e^{-ika}},
\end{equation}
which is not scale invariant. The only remaining possibility is
\begin{equation}
\label{eq:Sigma_condition}
\det \Sigma = -1, \, \, \text{tr} \Sigma = 0,
\end{equation}
which leads to a scale invariant S-matrix 
\begin{equation}
S = \frac{1}{1 - e^{-2ika } } ( -2i \sin ka ) \Sigma
 = - e^{ika} \Sigma.
\end{equation}
In this continuum limit where $a\rightarrow0$, the matrix $\Sigma$ can be parameterized as
\begin{equation}
\Sigma = -\lim_{a \rightarrow 0 } S = 
\begin{bmatrix}
\frac{\lambda^2- 1}{1 + \lambda^2} & \frac{-2\lambda }{1 + \lambda^2} \\
\frac{-2\lambda }{1 + \lambda^2} & \frac{1- \lambda^2}{1 + \lambda^2} \\
\end{bmatrix},
\end{equation}
where $\lambda\in\mathbb{R}$ is the parameter for $S_1(\theta)$, as introduced in Sec.~\ref{sec:notation}. 

We use this two-site interaction to model the $S_1(\theta)$ type conformal interface, as they give the same $S$ matrix in the continuum limit. Therefore, the large $t$ behavior of its Loschmidt echo should match with our field theoretic prediction. 

%%% Local Variables:
%%% TeX-master: "bCFT_paper"
%%% TeX-PDF-mode: t
%%% End:
