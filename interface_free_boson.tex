\todo[inline]{Mao: I suggest to relocate this part to an appendix, and maybe more details can be filled in, especially scale invariance condition imposed on the $\Sigma$. In the main text, just cite the relation between $\Sigma$ and $S$ and proceed}
In this appendix, we demonstrate how to realize scale invariant S-matrix in a lattice regularization. Following Sec.~\ref{sec_sub:free_boson_lattice}, we consider the following harmonic chain 
\begin{equation}
H = \frac{1}{2} \sum_i \left(\pi_i^2  + ( \phi_i - \phi_{i+1} )^2 \right) +  \frac{1}{2} \begin{pmatrix}  \phi_0, \phi_1 \end{pmatrix}
\Sigma
\begin{pmatrix}
\phi_0 \\
\phi_1 
\end{pmatrix}
\end{equation}
where the matrix $\Sigma$ parameterize two-site interaction between site 0 and 1. To find the scattering matrix, we set up the plane wave scattering problem across the interface with the following ansatz (the use of $(n-1)$ in $\phi_n^B$ simplifies the calculation)
\begin{equation}
\label{eq:ansatz}
\phi_n
= \left\lbrace
  \begin{aligned}
    \phi_n^A &= A_{-} e^{i \omega t  - inka}  + A_{+} e^{i \omega t  + inka}  & \quad  n \le 0 \\
    \phi_n^B &= B_{-} e^{i \omega t  - i(n-1)ka}  + B_{+} e^{i \omega t  + i(n-1)ka} & \quad n \ge 1 \\
  \end{aligned} \right. 
 \quad 
\end{equation}
where $a$ is the lattice constant. Upon solving the equation of motion far away from the interface, it is found that the system is gapless with the dispersion relation $\omega = \left|2\sin\frac{k}{2}\right|$. The interface will not break the criticality of the harmonic chain\cite{peschel_exact_2012}. After some algebra, the incoming and outgoing amplitudes are related via
\begin{widetext}
\begin{equation}
\label{eq:discrete_S}
\begin{pmatrix}
A_{+} \\
B_{-}\\
\end{pmatrix}
=-
\begin{bmatrix}
\Sigma_{11} +e^{ika} & \Sigma_{12}\\
\Sigma_{21} & \Sigma_{22} + e^{ika}
\end{bmatrix}^{-1}
\begin{bmatrix}
\Sigma_{11} + e^{-ika} & \Sigma_{12} \\
\Sigma_{21} & \Sigma_{22}  + e^{-ika}  \\
\end{bmatrix}
\begin{pmatrix}
A_{-}\\
B_{+}\\
\end{pmatrix}
\end{equation}
and the scattering matrix can be solved as
\begin{equation}
  S = \frac{-1}{ \det \Sigma  + e^{-ika} \text{tr} \Sigma   + e^{-2ika}}
\begin{bmatrix}
\det \Sigma+ \Sigma_{11} e^{-ika} + \Sigma_{22} e^{ika}+1  & -2i \sin ka \Sigma_{12}  \\
-2i \sin ka \Sigma_{21} &  \det \Sigma+ \Sigma_{11} e^{ika} + \Sigma_{22} e^{-ika}+1\\
\end{bmatrix}
\end{equation}
\end{widetext}
The S matrix elements are reflection and transmission coefficients. In order to have a scale invariant interface, they need to be $k-$independent. Therefore the necessary condition is that the ratio $|S_{11}/S_{21}|$ is independent of $k$, which implies 
\begin{equation}
\label{eq:Sigma_condition}
\det \Sigma = -1, \, \, \text{tr} \Sigma = 0
\end{equation}
Therefore, we have the scale invariant S-matrix as
\begin{equation}
S = \frac{1}{1 - e^{-2ika } } ( -2i \sin ka ) \Sigma
 = - e^{ika} \Sigma
\end{equation}
In the actual simulation, we evolved a large enough system for a long time, such that a field theoretic description can be applied. In this continuum limit where $a\rightarrow0$, the matrix $\Sigma$ can be parameterized as
\begin{equation}
\Sigma = -\lim_{a \rightarrow 0 } S = 
\begin{bmatrix}
\frac{\lambda^2- 1}{1 + \lambda^2} & \frac{-2\lambda }{1 + \lambda^2} \\
\frac{-2\lambda }{1 + \lambda^2} & \frac{1- \lambda^2}{1 + \lambda^2} \\
\end{bmatrix}
\end{equation}
where $\lambda\in\mathbb{R}$ is the parameter for $S_1(\theta)$, as emphasized in Sec.~\ref{sec:notation}. This simple relation between $\Sigma$ and the S-matrix provides a tool to check our analytic results. 

%%% Local Variables:
%%% TeX-master: "bCFT_paper"
%%% TeX-PDF-mode: t
%%% End:
