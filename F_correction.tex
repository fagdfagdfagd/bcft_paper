
In the course of deriving the free energy subject to various boundary conditions, we use conformal transformation to convert the spacetime diagram with slits to a cylinder diagram, where the boundary state calculation in App.~\ref{app:lambda_12} (and ground state energy calculation in App.~\ref{app:gnd_dn_lambda}) is applicable. However, the free energy is not invariant under the conformal transformation since the boundaries partially break the conformal symmetry. In this Appendix, we point out two corrections -- one from the outer boundary regulator and the other from the inhomogeneous Schwartzian term to get the correct exponent of the fidelity and Loschmidt echo. 

% outer boundary correction
It is discussed in Cardy and Peschel's work\cite{cardy_finite-size_1988} that boundary will contribute a logarithmic term in the free energy,
\begin{equation}
F = - \frac{c}{6} \left( \int_M  K(x) d^2x + \int_{\partial M} k_g ds \right)  \ln L , 
\end{equation}
where $M$ is a 2d smooth manifold, $K(x)$ is the Gaussian curvature, $k_g$ is the geodesic curvature of the boundary of the manifold and $L$ is the system's characteristic length. 

The boundary term was not previous noticed in the literature, but is actually important even in the simplest example of the disk free energy. Consider an annulus on flat space with inner radius $r_1$ and outer radius $r_2$. Its free energy is
\begin{equation}
F({\rm annulus}) = -  \frac{c}{6} \ln \frac{r_2}{r_1} .
\end{equation}
On the other hand the free energy of a disk of radius $r_2$ is
\begin{equation}
  F( {\rm disk} ) = - \frac{c}{6} \ln \frac{r_2}{a},
\end{equation}
where $a$ is the short distance regulator. The disk free energy is completely contributed by its outer boundary with other parts being conformal invariant. In fact, $K = 0, k_g = \frac{1}{r}$ for disk, and so 
\begin{equation}
F( {\rm disk}  ) = - \frac{c}{6}  \left( \int_{\partial M} k_g ds \right)  \ln \frac{r_2}{a}  = - \frac{c}{6}  \ln \frac{r_2}{a} , 
\end{equation}
where $a$ is the short distance cut-off. 

One can then interpret the annulus free energy as additive contributions from its outer and inner surfaces
\begin{equation}
F( {\rm annulus} ) = -  \frac{c}{6} \ln \frac{r_2}{a} +  \frac{c}{6} \ln \frac{r_1}{a} = - \frac{c}{6} \ln \frac{r_2}{r_1}.
\end{equation}
An annulus becomes a disk when its inner radius is of order $a$, and we can see that the contribution from the inner surface $\frac{c}{6} \ln \frac{r_1}{a}$ becomes negligible compared to the one from the outer surface.  

A similar outer surface logarithmic term also appears in the middle panel of Fig.~\ref{fig:fidel-map}. The conformal map from the $z$ plane to $\xi$ plane bring the strip (with the small blue semi-circle) to the upper half plane with the semi-circle around $z = 1$ extracted. This is in close analogy with the truncated corner calculation in Ref.~\onlinecite{cardy_finite-size_1988}. In order to evaluate this diagram, we manually add the large blue semi-circle as IR cut-off, at the price of introducing an additional contribution  $-\frac{c}{6} \ln \frac{r_2}{a}$ of free energy which should not be there. 

The same thing happened in Fig.~\ref{fig:H-tau_fold} with a slightly different mechanism. In the slit diagram (left panel in Fig.~\ref{fig:H-tau_fold}), the regulators all have radii that are at the order of the short distance cut-off. They will have negligible contributions to the free energy. However, in the new $\xi$ plane, we implicitly switch to a {\it new short distance regulator} such that only the blue semi-circle around $0$ contribute negligibly. The outer surface radius, despite being the image of a small semi-circle on $z$ plane, will contribute a $-\frac{c}{6} \ln \frac{r_2}{a}$ term on the $\xi$ plane that should not be there. 

Therefore in both cases we should compensate $\frac{c}{6} \ln \frac{r_2}{a}$. Using the cylinder parameters in App.~\ref{app:lambda_12}, the $\xi$ plane and $z$ plane free energy are related through
\begin{equation}
F_{z} = F_{\xi} + \frac{c}{6} \beta .
\end{equation}
for both the fidelity and Loschmidt echo. 

% staircase geometry correction
The annulus on the $\xi$ plane is called the staircase geometry in Ref.~\onlinecite{cardy_finite-size_1988} due to its evolution in angular direction. The traditional radial quantization however has radial direction to be the time. One can show that the Hamiltonian of the staircase and rectangle has a shift due to the Schwartzian\cite{cardy_finite-size_1988} of the conformal transform
\begin{equation}
H_{\xi} = H_{w} - \frac{c}{24\pi} \beta .
\end{equation}
After the evolution for $2\pi$ (in the folding picture, the evolution is only $\pi$ but there are two bosons), the difference in the free energy is
\begin{equation}
F_{\xi} = F_{w} - \frac{c}{12} \beta .
\end{equation}
Gathering the two terms, we obtain the missing correction $\frac{c}{12} \beta$ between the slit and cylinder diagram, 
\begin{equation}
F_{z} = F_w + \frac{c}{12}\beta.
\end{equation}


%%% Local Variables:
%%% TeX-master: "bCFT_paper"
%%% TeX-PDF-mode: t
%%% End:
