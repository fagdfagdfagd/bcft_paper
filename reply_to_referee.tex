\documentclass{article}
\usepackage{amsmath}
\usepackage{amssymb}
\usepackage[compact]{titlesec}
\usepackage{graphicx}
\usepackage{tikz}
\usetikzlibrary{calc,fadings,decorations.pathreplacing,shapes,shapes.multipart,arrows,shapes.misc,intersections,positioning}
\usepackage{wrapfig}
\usepackage{bm}
\usepackage[autostyle, english = american]{csquotes}
\MakeOuterQuote{"}
\graphicspath{{./}}

\newcommand{\cf}{\textit{cf.} }
\newcommand{\ie}{\textit{i.e.} } 
\newcommand{\eg}{\textit{e.g.} }
\newcommand{\vs}{\textit{vs.} } 
\newcommand{\etal}{\textit{et al.} }
\newcommand{\etc}{\textit{etc.} }
\newcommand{\rev}[1]{{\color{red}#1}}

\newcommand{\up}{\mathord{\uparrow}}
\newcommand{\dn}{\mathord{\downarrow}}
\newcommand{\I}{\mathbb{I}}

\renewcommand{\vec}[1]{\boldsymbol{\mathbf{#1}}}
\newcommand{\qed}[0]{$\blacksquare$}

\newcommand{\reply}[1]{{\color{black}#1}}

\textwidth = 6.5 in
\textheight = 9 in
\oddsidemargin = 0.0 in
\evensidemargin = 0.0 in
\topmargin = 0.0 in
\headheight = 0.0 in
\headsep = 0.0 in
\parskip = 0.2in
\parindent = 0.0in

% \definecolor{darkblue}{rgb}{0.1,0.2,0.6} \definecolor{darkred}{rgb}{0.8,0.1,0.2}
% \usepackage[colorlinks,citecolor=darkblue,linkcolor=darkred,urlcolor=darkblue]{hyperref}

\begin{document}
\large

{\bf Reply to the First Referee -- BF13677/Zhou}

\color{blue}

The authors investigate a quench problem where two critical bosonic chains are initially separated and then coupled by a conformal defect. This generalizes earlier treatments, where the chains were simply connected. They determine the fidelity (the negative logarithm of the overlap of the initial and the final ground state for chain lengths L) and the Loschmidt echo (the negative logarithmic overlap of the evolved state after some time t with the initial ground state). Since these overlaps decay as power laws, the logarithms vary as lnL and lnt, respectively. For a simple connection of the chains, the prefactors are known to be 1/8 and 1/4, respectively. For a defect, the authors find, that they can vary continuously with the defect transmission, depending on the boundary conditions at the chain ends and at the defect. This is derived analytically via a path- integral picture of the two quantities and conformal mappings. It is also verified by numerical calculations on oscillator chains, based on explicit expressions of the ground states in terms of boson operators from which the overlaps can be obtained as determinants.

Given that such varying exponents have a long history and were found also for the entanglement entropy in chains with defects, this is an interesting topic and the results are worth publishing. However, there are a couple of points to be clarified or changed.

\reply{We first acknowledge the effort of this respectful referee to read our manuscript thoroughly and provide constructive comments. We will address the issues below.}

(1) In order to obtain the correct result for the known case, the authors have to add a constant beta/12 in (21). However, the derivation in Appendix A is not right. The relation (A5) between the Hamiltonians in the staircase and in the strip differs by a sign from the one given in Ref. [32]. I think that the argument must be different. Usually, the universal terms in free energies are determined via appropriate non-conformal transformations which bring in the energy- momentum tensor. In this context one wonders about additional constants coming from the starting states in the path integral representation.

\begin{figure}[h]
\centering
\includegraphics[width=\columnwidth]{/images/fig_H-tau_fold.pdf}
\caption{Conformal transformation from the slit to cylinder.}
\label{fig:H-tau_fold}
\end{figure}

\reply{
We thank the referee for pointing out this sign error, in fact we used the $w \rightarrow \xi \rightarrow z$ order in the calculation as opposed to the $z \rightarrow \xi \rightarrow w$ order of the conformal transformation in the figure. We have fixed them.  

Nevertheless we hold our opinion of using these two corrections to account for the missing $\frac{1}{12}\beta$. Let us reiterate our arguments. 

The first correction comes from mapping the slit diagram to the annulus. Before doing this conformal transformation, the short distance regulator is small on the $z$ plane. Hence the contribution of the free energy coming from introducing the two blue lines are small. However, after the transformation, we implicitly choose another regulator which is small on the $\xi$ plane. But by the change of this short distance cutoff, the $\xi$ plane erroneously contribute a $- \frac{c}{6} \ln \frac{r_2}{r_1}$ term from the outer surface. The compensation $+\frac{c}{6} \ln \frac{r_2}{r_1} = \frac{c}{6}\beta$ is the correction we are talking about. 

The is the lesson we learned from the Cardy-Peschel paper, where there are two explicit examples of computing the free energies of a corner and a cone. Both objects should be regarded as infinitely extended without boundaries at certain radii. The authors computed the correct results by conformally mapping the geometry to whole plane (or upper half plane) and contrasted with the calculation of the truncated corner and cone with both short and long distance regulators. They found that the discrepancies of the truncated cone and corner w.r.t. to the correct results come from the outer boundary contribution. By subtracting this otherwise nonexistent contribution using their proposed geodesic curvature formula, they are able to recover the correct result. We believe that the mapping from the $z$ to $\xi$ plane result in the same situation here. 

The staircase geometry Hamiltonian and the plane Hamiltonian is differed by the Schwartzian term in the transformation rule of the stress tensor, which gives $-\frac{c}{12} \beta $ correction. By gathering the two terms, we do get the desired $\frac{c}{12}\beta$ correction. 

We request the referee to reconsider the presented arguments.
}

(2) I also find the whole presentation of the mapping unsatisfactory. In part, this has to with the fact that no lengths are indicated in the figures.

For the fidelity, one maps in Fig. 5 the initial strip to an annulus with inner and outer radii determined by the length of the strip which goes to infinity. Then one moves the point, where the boundary condition changes, to the origin, which means that, in the strip, it is moved to the far left and disappears there. I cannot see, how the calculation then proceeds. In [25], a different conformal map was used.

\reply{
We have added more descriptions in the text around Fig.5 and Fig.6 about the conformal mappings used there. One major change is Fig 5. We do not impose IR regulator(previously a vertical line at $x = W$)  on the left panel of the slit diagram, but do add the small semi-circle. After doing the conformal mapping to the center diagram, we end up with a truncated corner centered at $z = 1$. To evaluate this diagram, we add another semi-circle centered at $z = 1$ with radius $R$. This should avoid the problem the referee was worried about. 

(By the way, we add the semi-circle manually in the center of Fig. 5, which is in close analogy to the computation of the truncated cone and corner in the Cardy-Peschel paper. So to get the correct result, we should subtract its contribution to the free energy, which is exactly what we do in App. C.)

The annulus diagrams in both Fig. 5 and Fig. 6 are \rev{not} strictly concentric to higher order corrections of $\frac{\epsilon}{L}, \frac{\epsilon}{\tau}$. To resolve this problem, (and maybe the worry the referee previously had), we notice the following fact
\begin{itemize}
\item the non-conformal mapping from the non-concentric circles to the cylinder diagram exists: this can be doing by first mapping the non-concentric circles to concentric circles and then use the $\ln z$ map. 
\item the resulting width of the cylinder only depend on the cross ratio of the intersection points of the two circles w.r.t. the real axis.  \rev{still don't get why the map is non-conformal}
\end{itemize}
since the conformal mapping preserve the cross ratio, we can calculate the width of the cylinder in terms of the cross ratio in the non-concentric circles. The calculation we did use the first order approximation of the cross ratio, but is enough to get the leading order of the width of the cylinder correct. The higher order corrections are of order $\frac{\epsilon}{L}$ and $\frac{\epsilon}{\tau}$, which should not affect the power law scalings of the fidelity and echo. 

This line of argument is also included in the main text below Fig. 6. 
}

For the Loschmidt echo, the starting point is again a strip of width L, as shown in Fig. 4. However, in the left part of Fig. 6 this width no longer appears. There is only a small remark in the text that $L \gg \tau$. But L is of significance, for finite L the ratio t/L enters, and in Ref. [25] this case could actually be treated. 

\reply{We declared in the text (above Fig. 4) that we will take $L \gg \tau$ in all the echo computation. This keeps the problem with only one length scale. 

  We are aware that such geometry with both finite $t$ and $L$ can be mapped to the upper half plane by the Schwarz-Christoffel mappings, as demonstrated in Ref.[25]. But the conformal interface, the dashed line, will be inside the upper half plane and we don;t know how to calculate its OPE with the stress tensor. So in this paper, we only aim to understand the leading order behavior and its relation to the transmission coefficient. }


In the center part, there are two half-circles around the origin, which supposedly are the images of those on the left. But their radii (which determine the parameter beta) are not given anywhere. 

\reply{The radii for the left and center parts are indicated in the caption, and also discussed in the main text.}

Finally, the identification of the two semi-circles is physically quite strange. One wonders if it is really necessary for the calculations or the reasoning.

\reply{We take this identification because the boundary state computation is easier. 

In fact, since the distance between the blue lines ($\sim \ln L , \ln \tau$) on the $w$ plane in Fig.6 is much larger than the distance between the two dashed lines, one can view rotate this cylinder diagram and consider an evolution between the blue lines. This is the point of view taken by our alternative approach (an $S$ modular transformation to the figure in the boundary state calculation), the leading term of free energy is then the groundstate energy of the Hamiltonian given the boundary conditions on the dashed line, multiply by $\beta$, see Eq.(B8). 

If one impose different boundary conditions on the blue lines, the free energy then is not 
\begin{equation}
F = - \ln \text{tr}( e^{ - \beta H_{ab} } ) =  \beta  E_c
\end{equation}
but (for large $\beta$ as is the case here)
\begin{equation}
\begin{aligned}
F &= - \ln \langle \text{blue boundary state 1} | e^{ - \beta H_{ab}} | \text{blue boundary state 2} \rangle \\
&= \beta  E_c - \ln \langle \text{blue boundary state 1} | \text{ground state} \rangle \\
 &\quad - \ln \langle \text{blue boundary state 2} |\text{ ground state }\rangle                                                                                 
\end{aligned}
\end{equation}
where the additional terms can be viewed as Affleck-Ludwig boundary entropy that is subleading (independent of $\beta$) in the free energy. 

As far as the leading order (in beta) is concerned, the boundary conditions imposed on the blue line shouldn’t matter. We therefore take the simplest periodic boundary condition for the ease of computation. \rev{Do we want to mention this somewhere in the text?}}

(3) The calculation of the vacuum energy in equ.(E6) is not quite clear. Since the original expression diverges, a word about regularization would be in order. In the conversion of the sums into zeta functions (what does the index H mean ?) it seems to me that the n=0 terms are not handled correctly. The second function must be zeta(-1,-x). One could also imagine just giving a reference. 

\reply{
A clarification is indeed necessary. We here use the Hurwitz zeta function regularization and so H stands for Hurwitz. 

The final results, after a careful check is correct, however, the $x^{-s}$ term should be $-(-x)^{-s)}$ if the $n = 0$ term is handled properly. Thanks for pointing this out!

We labeled several minus signs to be red to remind future readers to be cautious. \rev{Do we want to keep these highlights?}}

I also think that Appendix E ("Alternative approach...") should come after the "actual" approach which is presented in Appendix F. There is also a typo in equ.(F9).

\reply{We have rearranged the order of appendixes as request and fix the typo in Eq.(A9), where the $\frac{1}{6}$ should be $-\frac{1}{6}$}

(4) The main result, namely the variation of the exponents in Equ. (27) and Figs. 8-10 as functions of the parameter theta is not discussed. The increase with theta must have some kind of physical interpretation and also the particular values at theta=0 and theta=pi/2.

\reply{We add two paragraphs in the discussion section about the main results. 

We first deal the three special points $\theta = 0, \frac{\pi}{4}, \frac{\pi}{2}$ and their special relations to the bcc operator ${\rm D} \rightarrow {\rm N}$, and then present an hand wave argument that the quadratic relation should come from the dimension of the vertex operators. 

In another paragraph, we use the free lattice boson model to argue why the exponent increase for larger $\theta$. 
}

(5) In all examples except the very last one in Fig.10, the outer boundaries of the chains remain the same and only the inner ones change. Therefore the notation for the change of boundary conditions introduced in (17) is unnecessary complicated. The formulation in (22) would be sufficient and much simpler.

(6) A number of equations expressing these changes of boundary conditions appear several times in identical form: (22),(37),(40); (24),(30) and Fig.7; (18),(26),(32),(38) and Fig.8; (28),(34) and Fig.9; (36),(39) and Fig.10. This is a redundancy which should be reduced.

\reply{Question (5) and (6) are about the notations of boundary conditions. 

We remove the redundant boundary condition c in the some of the equations as long as it is equal to a. The simplified notation like $\text{DN} \rightarrow \lambda$ is better to be perceived as a single boundary condition changing process and the associated bcc operator. 

Since the notation is shortened, we don'��t let them occupy a single equation and instead put them inside a sentence. We believe that such redundancy can be tolerated given that the length of the this notation is comparable to the citation phrase in Eq.(XX)��, and the time it saves to look up. }

(7) All References appear twice with different numbers! This must also be the reason, why the numbering in the text starts at 48. I am astonished that the authors did not notice that.

\reply{This is really embarrassing. We had 47 citations in our local version of manuscripts but 94 on the submitted. We find that this is because our submission script flatex inserted all the input tex files (since the submission is easier with a single tex file) as well as all bbl files, resulting in the duplication of references. 

But it is totally our fault not to check the number of references before submission. We have fixed it in this submission. }

(8) At no point could I see a remark on the analytical continuation from imaginary to real time. Instead, the notation changes back and forth between t and tau.

\reply{This point should be clarified in the introduction of the path-integral formalism(Sec. III. A.). All the path-integral spacetime diagrams are in the imaginary time. We only do analytic continuation to the final results of the imaginary time diagram.

We talk about the analytic continuation procedure between Eq.(16) and Fig.4 and demonstrate in Eq. (17). Thereafter we uniformly use $t$ towards the end of the main text. \rev{This seems not to be the case.. Do we want to change that?}}

(9) As the parameter theta determines the transmission through the defect, it would be useful to give the transmission coefficient below Equ.(4) explicitly.

\reply{We have added the transmission coefficients and add a short paragraph discussing the parameter lambda in different physical realizations.}

(10) As to the figures:

- If one wants to show the folding in Figs. 3 and 4, one should not move and rescale the strip on the right hand sides but leave it in place. Then the meaning of the length L is also clearer. Incidentally, L appears here for the first time in the paper, which should also be changed.

- In Fig. 5, the boundary conditions should also be indicated on the right side.

\reply{We have modified Fig. 3 and 4 according to the suggestions. 

We also define L to be the length for immediately after the definition of fidelity and above Fig. 4 for echo. 

We have dramatically change Fig. 5. 
}

- The legends in Figs. 7-10 are strange in that the analytical results are represented by a horizontal symbol (line), but the numerical ones by a vertical one.

\reply{The numerical results are represented by dots and error bar. The error bar is small and not clearly visible in these diagrams. We have modified the legend for Fig. 7-10 to make the vertical error bar clearer.}

- There is no (a) and (b) in Figs. 8 and 9, but in the caption these are referred to.

\reply{We have modified Fig. 8 and 9 accordingly.}

- The system sizes in the numerics given in the captions are unclear to me. What does 30k or 35k mean? In Fig. 7, it says k=1 immediately after that. Is this the same k? If not, what is it? As with L, I think that such numbers should also appear in the text. If one is talking about 35.000 sites, one wonders why such a large number was taken.

\reply{The k in the math mode font is the spring constant, while the $k$ after 30 and 35 means 1000. We didn't realize the conflict. We have changed 30k to 30000 and 35k to 35000. We take 35000 sites for reducing the finite size effect in Fig. 8 which is important for the theta value near zero. }

- Fig. 11 corresponds to the right piece of Fig. 6, but is rotated by 90°. This is confusing, in particular since the corresponding Fig. 12 is not rotated. The lambda line on the left has to be dotted. Also confusing is the use of x and L in (E1) below.

\reply{We didn't realize that the spatial coordinate $x$ and $x = \frac{\theta}{pi}$ have a conflict. We rotate the figure to use $y$ as coordinate and reserve $x = \frac{\theta}{pi}$. }
 
(11) In a number of places the English has to be corrected (missing articles or endings, missing or incorrect words).

\reply{We have proofread the paper before the resubmission and (hopefully) corrected those grammatical/spelling errors.}




{\bf \color{black} Report of the Second Referee -- BF13677/Zhou}

In the manuscript "Bipartite Fidelity and Loschmidt Echo of Bosonic Conformal Interface" the authors report on results concerning the finite volume decay exponent of the fidelity and long time exponent of the Loschmidt echo after joining two bosonic boundary CFTs. As they point out such exponents could potentially be observed in certain quantum wire experiments. The results appear to be correct and they are supported by independent numerical calculations coming from a harmonic lattice model.

I think the paper is interesting enough and sufficiently well written to be published in Physical Review B, but before acceptance I would like the authors to consider my remarks below and make the necessary modifications.

\reply{We thank the referee for the positive review and reply to the comments below.}

1) The introduction overestimates the results that are actually presented in the main text. In the fourth paragraph it is claimed that the paper studies joining two different CFTs, when in fact only a very special case is treated: joining two identical massless free boson theories, with arbitrary (but identical) boundary conditions though. It should be clarified in the introduction that we will only see results for this special case. In fact, paragraph 4 is not at all clear, it should be rewritten.

\reply{We agree that this paper only deals with 2d bosonic CFTs. However these CFTs can be different, e.g. having different compactification radii. The relation between lambda and the the different compactification radii are nevertheless hidden in the last appendix of the paper, which makes it hardly accessible to the readers. We also didn't explicitly point out the derivation (and physical interpretation) of this relation in one of the references we cited. 

On the other hand, the boundary conditions on the sites that later connected are different. For example, the notation DN means that one end of the chain has Dirichlet and the other end that connects to it has Neumann boundary conditions, see Fig. 3.

After connection, normally one should assume it's a periodic boundary condition (after folding) which gives transmission coefficient 1. However, in the text we consider more general conformal interface that can have arbitrary transmission coefficient between 0 and 1. Such interface can indeed be realized by connecting two boson theories with different compactification radii.

To clarify this issue, we have rewritten paragraph 4 in the introduction that made the following changes 
\begin{itemize}
\item We rephrase what we did as ``generalization to interface that is partially transmissive''.
\item We list the examples we talked about in this paper: the harmonic chain example(non-compact boson), the two different 1+1 bosonic CFT with different radii.
\item We also explicitly mention the the dependence of lambda w.r.t. the different compactification at the end of Sec. II. A
\end{itemize}
}

2) In this 4th paragraph of the introduction (also recurring in some places later) it is mentioned that the parameter lambda is controlled by the ratio of compactification radii. Is this so? This point is never discussed in the main text (or the appendices). Doesn't lambda describe the interface after joining the two CFTs and the compactification radii are the same in the two theories? At least in the case considered in the main text.

\reply{It is our fault that the relation of lambda and ratio of the compactification radii are not pointed out explicitly in the main text. 
We have made changes in paragraph 4 accordingly and also some texts in the end of Sec.II.A to discuss the generalization to compact bosons (with different radii)}

3) In section 3 after Eq. (16) twist fields are mentioned but these were never introduced before, therefore the statement "The tip of the slit is no longer a twist field..." doesn't seem to be self-contained.

\reply{We avoid the use the phrase ``twist field'' and change it to corner singularity and cite Cardy1988 for its treatment.}

4) After Eq. (17) it is stated that the boundary condition "c" is neglected. Isn't it taken to be "a" instead?

\reply{This statement is incorrect. It is taken to be a and changing it will influence the result even in the large L limit. So we did no��t neglect c� but taking it to be a��. We also take a simpler notation for this case after Eq. (17) }

5) Now we come to the main result of the paper, Eq. (23), that will yield all the exponents which will then be numerically verified. Despite this equation being the main results, its derivation is hidden away in an appendix. In fact all the calculations in this paper are demoted to be part of one of the numerous Appendix sections. I believe, that at least the main result should be given some space in the main text. I don't think this would compromise readability or accessibility of the results, in fact it think it would improve on it.

\reply{The derivation of Eq.(23) is contained in the original App. F, which is mainly the evaluation of partition function using a determinant identity with no physical arguments. In fact we believe moving App. F back to the main text will still make it isolated to the surrounding physical discussion.  However, in order to improve readability, we have rearranged the appendices and make explicit references in the main text for the derivation.

App. A is the calculation for the main analytic results. The reader shall only need this appendix to reproduce Eq.(23). 

App. B/C are supplements to analytic results. Other appendices are for the numerical models and calculational details. We have also mentioned these structural changes in the end of the introduction.}

6) The last sentence before Sec. 4 states that the analytical (incorrect) and numerical curves "merge" at theta=0. In fact, these two curves are quite different and clearly the analytical curve cannot reproduce the numerics at all, as expected. To me it seems like that the green curve coincides with the dots in Fig. 10 only by chance. I'm not sure what the relevance of this curve is or mentioning that at theta=0 the numerics and the (incorrect) analytics "merge." This merging is simply because theta=0 is the no-quench point, when the pre- and post-quench setup is the same.

\reply{We have remove the sentence of ``merging'' at theta = 0. That is as the referee said only a calibrating point. But we do want to reserve the green curve as we (subjectively) believe the collapse is not a coincidence. The difference $1/8$� is a magic number that is twice of the majorana operator dimension, maybe indicating the structure of the 3-point function (the bcc operators between each pair of  a, b, c). We didn't write this speculation for the lack of evidence but do want to reveal this coincidence to inspire the readers.}

7) The references appear in duplicate.

8) Additionally, there are typos throughout the text that should be corrected (e.g. like "the this"), I suggest the manuscript to be proofread carefully before resubmission. 



\end{document}
%%% Local Variables: 
%%% TeX-PDF-mode: t
%%% End:


%%% Local Variables: 
%%% TeX-PDF-mode: t
%%% End: 
