
We write the quadratic bosonic Hamiltonian in the following form
\begin{equation}
H = \frac{1}{2} (b^{\dagger}, -b)
M 
\begin{pmatrix}
b\\
b^{\dagger} 
\end{pmatrix} \qquad 
M = 
\begin{bmatrix}
A & -B^* \\
B & -A^* \\
\end{bmatrix}
\end{equation}
where $A$ is Hermitian and $B$ is symmetric. The use of $(b^{\dagger}, -b)$ instead of $(b^{\dagger}, b)$ is intentional to facilitate intensive symplectic calculation. The Bogoliubov transformation that diagonalizes the Hamiltonian
\begin{equation}
(b, b^{\dagger}) S = ( a, a^{\dagger} ) 
\end{equation}
should preserve the commutation relation
\begin{equation}
J \equiv \begin{bmatrix}
0 & \mathbb{I}\\
-\mathbb{I} & 0 
\end{bmatrix} =  
[
\begin{pmatrix}
a \\
a^{\dagger} 
\end{pmatrix}, 
\begin{pmatrix}
a \, a^{\dagger}
\end{pmatrix}]
=
S^{\top}
[
\begin{pmatrix}
b \\
b^{\dagger} 
\end{pmatrix}, 
\begin{pmatrix}
b \, b^{\dagger}
\end{pmatrix}]S
 = S^{\top} J S 
\end{equation}
and it is therefore in ${\rm Sp}( 2n, \mathbb{C} ) $. On the other hand, the requirement that $a^{\dagger}$ still be the complex conjugation of $a$ leads to the block structure of $S$
\begin{equation}
S = 
\begin{bmatrix}
u & v^*\\
v & u^*\\    
\end{bmatrix}
\end{equation}
In turns of the new bosonic variable $a$,
\begin{equation}
\begin{aligned}
H &= \frac{1}{2} ( b,  b^{\dagger} ) J^{-1} M 
\begin{pmatrix}
b\\
b^{\dagger} 
\end{pmatrix}
 = \frac{1}{2} ( a,  a^{\dagger} ) S^{-1} J^{-1} M (S^{\top})^{-1} 
\begin{pmatrix}
a\\
a^{\dagger} 
\end{pmatrix}\\
& = \frac{1}{2} ( a^{\dagger}, -a )  (S^{\top} M (S^{\top})^{-1} )
\begin{pmatrix}
a\\
a^{\dagger} 
\end{pmatrix}
\end{aligned}
\end{equation}
Out goal is to find such a $S \in {\rm Sp}( 2n, \mathbb{C} )$ that diagonalize $M$\footnote{This particular diagonal form is dictated by the symmetric $P M^* P^{-1} = -M, P S P^{-1} = S $. This is also evident when one realizes the diagonal matrix $\in \mathfrak{sp}(2n, \mathbb{C} )$}
\begin{equation}
M = (S^{\top})^{-1} 
  \begin{bmatrix}
    d & 0\\
    0 & -d \\
  \end{bmatrix} S^{\top}  
\end{equation}
By matching to the standard form symplectic Lie algebra $\begin{bmatrix} a & b \\ c & -a^{\top} \end{bmatrix}$ where $b$ and $c$ are symmetric, we immediately see that $M \in \mathfrak{sp}( 2n, \mathbb{C} ) $. The adjoint action of $S^{\top}$ conjugates $M$ to the maximal torus. 

In the real basis
\begin{equation}
\begin{bmatrix}
b \\
b^{\dagger} \\
\end{bmatrix}
=
C
\begin{bmatrix}
\phi \\
\pi \\
\end{bmatrix}
\end{equation}
the Hamiltonian becomes
\begin{equation}
H = \frac{1}{2} ( b, b^{\dagger}) J^{-1} M 
\begin{bmatrix}
b \\
b^{\dagger} 
\end{bmatrix}
= \frac{1}{2} ( \phi, \pi )
\mathcal{M} 
\begin{pmatrix}
\phi \\
\pi 
\end{pmatrix}
\end{equation}
where
\begin{equation}
\mathcal{M} = C^{\top} J^{-1} M C = C^{\dagger} 
\begin{bmatrix}
A & - B^* \\
-B & A^* \\                                             
\end{bmatrix}C =
\begin{bmatrix}
{\rm Re}(A - B ) & -{\rm Im}( A ) + {\rm Im}( B  )\\
{\rm Im}(A) + {\rm Im} B & {\rm Re}(A+ B ) 
\end{bmatrix} 
\end{equation}
is a real symmetric matrix based on the symmetry of $A$ and $B$. If we can diagonalize the matrix $\mathcal{M}$ by a real symplectic matrix $\mathcal{S} \in \mathfrak{sp}( 2n, \mathbb{R} ) $
\begin{equation}
\mathcal{M} = \mathcal{S}
\begin{bmatrix}
d & 0 \\
0 & d \\
\end{bmatrix} \mathcal{S}^{\top}
\end{equation}
then in the $a$ basis
\begin{equation}
\begin{bmatrix}
a \\
a^{\dagger}\\
\end{bmatrix}
= C \mathcal{S}^{\top}
\begin{bmatrix}
\phi \\
\pi 
\end{bmatrix}
\end{equation}
the Hamiltonian is diagonal,
\begin{equation}
\begin{aligned}
( \phi , \pi ) \mathcal{S}
\begin{bmatrix}
d & 0 \\
0 & d \\ 
\end{bmatrix} \mathcal{S}^{\top} 
\begin{pmatrix}
\phi \\
\pi 
\end{pmatrix}
 &= ( a, a^{\dagger} ) C^*
\begin{bmatrix}
d & 0 \\
0 & d \\ 
\end{bmatrix}
C^{\dagger}
\begin{pmatrix}
a \\
a^{\dagger} 
\end{pmatrix} 
=( a^{\dagger}, a )J 
\begin{bmatrix}
0 & d \\
d & 0 \\ 
\end{bmatrix}
\begin{pmatrix}
a \\
a^{\dagger} 
\end{pmatrix}\\
&=( a^{\dagger}, -a )
\begin{bmatrix}
d & 0 \\
0 & -d \\ 
\end{bmatrix}
\begin{pmatrix}
a \\
a^{\dagger} 
\end{pmatrix}
\end{aligned}
\end{equation}
Tracing back to the complex basis, we have
\begin{equation}
\begin{bmatrix}
d & 0 \\
0 & -d \\ 
\end{bmatrix}
 = J C^* \mathcal{S}^{-1} C^{\top} J^{-1} M C (\mathcal{S}^{\top})^{-1} C^{\dagger} 
\end{equation}
or equivalently
\begin{equation}
\label{eq:M-diag}
M = J C^* \mathcal{S} C^{\top} J^{-1}
\begin{bmatrix}
d & 0 \\
0 & -d \\
\end{bmatrix} C \mathcal{S}^{\top} C^{\dagger}
\end{equation}
One then identifies
\begin{equation}
C \mathcal{S}^{\top} C^{\dagger} = S^{\top} 
\end{equation}
Let's check this is the wanted $S^{\top}$: 
\begin{enumerate}
\item $S^{\top}$ so constructed has the correct block structure $[u, v^*; v, u^*]$. The computation to verify this is similar to the fermionic case where $U = C \mathcal{O} C^{\dagger} $ has the correct block structure for a real orthogonal matrix $\mathcal{O}$. 
\item $C \mathcal{S}^{\top} C^{\dagger}  \in {\rm Sp}(2n, \mathbb{C} )$. It is easily verified that $C^{\dagger} J C^* = iJ$ and $C J C^{\top} = - i J$, so
\begin{equation}
C\mathcal{S}^{\top} C^{\dagger} J (C\mathcal{S}^{\top} C^{\dagger})^{\top} = i C\mathcal{S}^{\top} J (C\mathcal{S}^{\top})^{\top} = i C J C^{\top} = J 
\end{equation}
\item Eq.~\eqref{eq:M-diag} is a similarity transformation. 
\begin{equation}
J C^* \mathcal{S} C^{\top} J^{-1} = J C^* J (\mathcal{S}^{\top})^{-1} J C^{\top} J = C(\mathcal{S}^{\top})^{-1}  C^{\dagger} 
\end{equation}
\end{enumerate}

Now we come back to the case when the symplectic diagonalization is possible. It is the Williamson theorem\footnote{\cite{simon_congruences_1999,pirandola_correlation_2009} and the Stackexchange post \cite{_linear_stexg} contains the same simple constructive proof. \cite{xiao_theory_2009} was inspired from solving the dynamics(which is a canonical transformation). Chapter 3 of \cite{gosson_symplectic_2006} contains a basis independent proof. Also see a complete classification of Williamson theorem in Appendix 6 of Arnold\cite{arnold_mathematical_1997}}: for a real symmetric and positive definite matrix $\mathcal{M}$, there exist $\mathcal{S} \in \mathfrak{sp}( 2n, \mathbb{R} )$, such that
\begin{equation}
\label{eq:M-diag2}
\mathcal{M}  = \mathcal{S}
\begin{bmatrix}
d & 0 \\
0 & d \\
\end{bmatrix} \mathcal{S}^{\top}
\end{equation}
where the diagonal matrix is positive and contains positive eigenvalues of $i J\mathcal{M}$. I here present a constructive proof[Appendix A of \cite{pirandola_correlation_2009}]. The general solution of Eq.~\eqref{eq:M-diag2} is 
\begin{equation}
 \mathcal{S} = \mathcal{M}^{\frac{1}{2}} O 
\begin{bmatrix}
d^{-\frac{1}{2}} & 0 \\
0 & d^{-\frac{1}{2}}\\
\end{bmatrix}
\end{equation}
where $O$ is an orthogonal matrix to be fixed. Since $\mathcal{S}$ is symplectic, we have
\begin{equation}
O^{\top} \mathcal{M}^{\frac{1}{2}} J \mathcal{M}^{\frac{1}{2}} O  = 
\begin{bmatrix}
d^{\frac{1}{2}} & 0 \\
0 & d^{\frac{1}{2}} \\
\end{bmatrix}
J
\begin{bmatrix}
d^{\frac{1}{2}} & 0 \\
0 & d^{\frac{1}{2}} \\
\end{bmatrix}
=
\begin{bmatrix}
0 & d\\
-d & 0 \\
\end{bmatrix}
\end{equation}
Hence matrix $O$ actually diagonalize the anti-symmetric $ \mathcal{M}^{\frac{1}{2}} J \mathcal{M}^{\frac{1}{2}}$. The eigenvalues of $ \mathcal{M}^{\frac{1}{2}} J \mathcal{M}^{\frac{1}{2}}$ are $\pm i d$, which are also the eigenvalues of $  J \mathcal{M}$. And since $\mathcal{S}$ uses $d^{-\frac{1}{2}}$, $d$ should be the positive eigenvalues of  $iJ \mathcal{M}$. 



%%% Local Variables:
%%% TeX-master: "Peschel_EE"
%%% TeX-PDF-mode: t
%%% End: