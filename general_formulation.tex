We consider a conformal interface connecting two potentially different CFTs. The interface is located at $x=0$ and characterized by the ``gluing condition''
\begin{eqnarray}\begin{aligned}
\label{Def. of M}
\begin{pmatrix}
\partial_x\phi\\
\partial_t\phi
\end{pmatrix}_{x=0^-}
=M\begin{pmatrix}
\partial_x\phi\\
\partial_t\phi
\end{pmatrix}_{x=0^+}
\end{aligned}\end{eqnarray}
The interface is now penetrable. Instead of requiring the normal components of the stress tensor to vanish\cite{cardy_conformal_1984}, one only needs the stress tensor to be continuous across the defect. From the explicit expression of the stress tensor $T^{xt}=-\partial_x\phi\partial_t\phi$, one can show that\cite{bachas_permeable_2002} $M$ is an element of the Lorentz group $O(1,1)$. In terms of the boost parametrization, we have
\begin{eqnarray}\begin{aligned}
M_1(\theta)=\pm
\begin{bmatrix}
\lambda^{-1} & 0 \\
0 & \lambda
\end{bmatrix}\quad
M_2(\theta)=\pm
\begin{bmatrix}
0 & \lambda  \\
\lambda^{-1} & 0 
\end{bmatrix}
\end{aligned}\end{eqnarray}
where $\lambda=\tan\theta$ for $\theta\in\left[-\frac{\pi}{2},\frac{\pi}{2}\right]$. 

We note the following special cases. For $\theta=0,\pm\pi/2$, $\lambda$ appears to be singular and the fields on either side of the defect do not communicate to each other. For example, $M_1(0)$ implies $\partial_x\phi(0^+)=\partial_t\phi(0^-)=0$. For the two CFTs, this corresponds to Dirichlet boundary condition on the left, and Neumann condition on the right. Hereafter we shall denote it as `DN'. %Similarly one can check that $M_1(\pm\pi/2),M_2(0),M_2(\pm\pi/2)$ correspond to ND, NN, DD respectively. 
For $\theta=\pm\pi/4$, on the other hand, characterizes a perfectly transmitting defect. For example, there is effectively no defect in the case $M(\pi/4)$. For the other three cases, although the field $\phi$ may pick up a phase across the defect, the two counter propagating waves are still fully transmitted.

It proves useful to rewrite Eq.~\eqref{Def. of M} in the ``light cone'' coordinate $x^\pm\equiv t\pm x$, where the physical meaning of $\theta$ is most transparent. For clarity, we denote the field on the negative (positive) real axis as $\phi^1$ ($\phi^2$). {\bf\color{red}Tianci: Add in the right panel in Fig.~\ref{fig:cut-and-join} the field content?}. After some algebra, we have
\begin{eqnarray}\begin{aligned}
\label{Def. of S}
\begin{pmatrix}
\partial_+\phi^1\\
\partial_-\phi^2
\end{pmatrix}
=S
\begin{pmatrix}
\partial_-\phi^1\\
\partial_+\phi^2
\end{pmatrix}
\end{aligned}\end{eqnarray}
where 
\begin{eqnarray}\begin{aligned}
S_1(\theta)=\begin{bmatrix}
\cos 2\theta & \sin 2\theta \\
\sin 2\theta & -\cos 2\theta
\end{bmatrix}\quad
S_2(\theta)=\begin{bmatrix}
-\cos 2\theta & \sin 2\theta \\
-\sin 2\theta & -\cos 2\theta
\end{bmatrix}
\end{aligned}\end{eqnarray}
The scattering matrix $S_1$ and $S_2$ relate the incoming and outgoing waves at the defect. For example, $S_1(0)$ implies that $\partial_\pm\phi^{1}=\partial_\pm\phi^{2}$ as discussed above. For other nontrivial values of $\theta$, the defect becomes partially-transmitting. The transmission and reflection coefficients, which are determined by $\theta$, can be read off from the scattering matrices.

{\bf\color{red}As we shall show explicitly in Sec.~\ref{sec: A Free Boson Lattice Model}, the scattering matrices are independent of the wavelengths, as required by the scale invariance.}
