We consider a conformal interface connecting two potentially different CFTs in $1+1$ dimensions. The interface is located at $x=0$ and characterized by the ``gluing condition''
\begin{eqnarray}\begin{aligned}
\label{eq:def_M}
\begin{pmatrix}
\partial_x\phi\\
\partial_t\phi
\end{pmatrix}_{x=0^-}
=M\begin{pmatrix}
\partial_x\phi\\
\partial_t\phi
\end{pmatrix}_{x=0^+}
\end{aligned}\end{eqnarray}
The interface is assumed to be penetrable. Instead of requiring the normal components of the stress tensor to vanish\cite{cardy_conformal_1984}, one only needs the stress tensor to be continuous across the defect. From the explicit expression of the stress tensor $T^{xt}=-\partial_x\phi\partial_t\phi$, one can show that\cite{bachas_permeable_2002} $M$ is an element of the Lorentz group $O(1,1)$. In terms of the boost parametrization, we have
\begin{eqnarray}\begin{aligned}
M_1(\theta)=\pm
\begin{bmatrix}
\lambda^{-1} & 0 \\
0 & \lambda
\end{bmatrix}\quad
M_2(\theta)=\pm
\begin{bmatrix}
0 & \lambda  \\
\lambda^{-1} & 0 
\end{bmatrix}
\end{aligned}\end{eqnarray}
where $\lambda=\tan\theta$ for $\theta\in\left[-\frac{\pi}{2},\frac{\pi}{2}\right]$. 

We note the following special cases. For $\theta=0,\pm\pi/2$, $\lambda$ (or $\lambda^{-1}$) appears to be singular and the fields on either side of the defect cannot transmit to the other side. For example, $M_1(0)$ implies $\partial_x\phi(0^+)=\partial_t\phi(0^-)=0$ which correspond to Dirichlet boundary condition on the left, and Neumann condition on the right. Hereafter we shall denote it as `DN'. Similarly one can check that $M_1(\pm\pi/2),M_2(0),M_2(\pm\pi/2)$ correspond to ND, NN, DD respectively. For $\theta=\pm\pi/4$, on the other hand, $M(\theta)$ characterizes a perfectly transmitting defect. For example, there is effectively no defect in the case $M(\pi/4)$. For the other three cases, although the field $\phi$ may pick up a phase across the defect, the two counter propagating waves are still fully transmitted. {\bf\color{red} shall we follow permeable wall paper to denote them as PP, etc? it is a bit confusing though..}

It proves useful to rewrite Eq.~\eqref{eq:def_M} in the ``light cone'' coordinate $x^\pm\equiv t\pm x$, where the physical meaning of $\theta$ is most transparent. For clarity, we denote the field on the negative (positive) real axis as $\phi^1$ ($\phi^2$). {\bf\color{red}Tianci: Add in the right panel in Fig.~\ref{fig:cut-and-join} the field content?}. After some algebra, we have
\begin{eqnarray}\begin{aligned}
\label{eq:def_S}
\begin{pmatrix}
\partial_+\phi^1\\
\partial_-\phi^2
\end{pmatrix}
=S
\begin{pmatrix}
\partial_-\phi^1\\
\partial_+\phi^2
\end{pmatrix}
\end{aligned}\end{eqnarray}
where 
\begin{eqnarray}\begin{aligned}
\label{eq:S1_S2}
S_1(\theta)=\begin{bmatrix}
\cos 2\theta & \sin 2\theta \\
\sin 2\theta & -\cos 2\theta
\end{bmatrix}\quad
S_2(\theta)=\begin{bmatrix}
-\cos 2\theta & \sin 2\theta \\
-\sin 2\theta & -\cos 2\theta
\end{bmatrix}
\end{aligned}\end{eqnarray}
The scattering matrix $S_1$ and $S_2$ relate the incoming and outgoing waves at the defect. For example, $S_1(0)$ implies that $\partial_\pm\phi^{1}=\pm\partial_\mp\phi^{2}$ as discussed above. For other nontrivial values of $\theta$, the defect becomes partially-transmitting. The transmission and reflection coefficients, which are determined by $\theta$, can be read off from the scattering matrices. As we shall show explicitly in Sec.~\ref{sec_sub:free_boson_lattice}, the scattering matrices are independent of the wavelengths, as required by the scale invariance.

The problem of conformal interface can be studied in the framework of boundary CFT\cite{cardy_boundary_2004,cardy_conformal_1984} after we fold the field $\phi^2$ to the left half plane on top of $\phi^1$\cite{oshikawa_boundary_2010}. The boundary at $x=0$ becomes impenetrable for the tensor theory, and there hosts a conformal boundary states for the bosons. The folding sends $\phi^2(x)\rightarrow\phi^2(-x)$ for $x<0$, therefore the gluing condition reads
\begin{eqnarray}\begin{aligned}
\partial_t(\cos\theta\phi^1-\sin\theta\phi^2)=0 \quad
\partial_x(\sin\theta\phi^1+\cos\theta\phi^2)=0 
\end{aligned}\end{eqnarray}
for the case $M=M_1(\theta)$. After performing explicit mode expansion for both $\phi^{1,2}$, {\bf\color{red}write down mode expansion, or simply write down the physical meanings of a's as follows?}we have
\begin{eqnarray}\begin{aligned}
\label{eq:rotation_a_basis}
\cos\theta a_n^1-\sin\theta a_n^2 &= +( \cos\theta\bar{a}_{-n}^1-\sin\theta \bar{a}_{-n}^2 ) \\
\sin\theta a_n^1+\cos\theta a_n^2 &= -( \sin\theta\bar{a}_{-n}^1+\cos\theta \bar{a}_{-n}^2 ) 
\end{aligned}\end{eqnarray}
where $n>0$ is an integer. $a_n^{1,2}$ ($\bar{a}_n^{1,2}$) annihilate the left (right)-moving modes of $\phi^{1,2}$, and $a_{-n}^{1,2}$ ($\bar{a}_{-n}^{1,2}$) are the corresponding creation operators. Eq.~\eqref{eq:rotation_a_basis} is essentially a rotation of basis, and the conformal boundary state reads\cite{oshikawa_boundary_2010}
\begin{equation}
\label{eq:bd_state}
\begin{aligned}
| B \rangle 
& =  \exp\Big\{ \sum_{n > 0 } \frac{1}{n}
\begin{pmatrix}
a_{-n}^1\\
a_{-n}^2\\                              
\end{pmatrix}
S_1
\begin{pmatrix}
\bar{a}_{-n}^1  \bar{a}_{-n}^2
\end{pmatrix} \Big\} |0\rangle
\end{aligned}
\end{equation}
where $S^1$ is precisely the scattering matrix in Eq.~\eqref{eq:def_S}. As one can explicitly check, Eq.~\eqref{eq:bd_state} satisfies the condition for the conformal boundary state\cite{oshikawa_boundary_2010}. The calculation for the case $M=M_2(\theta)$ is completely analogue. Eq.~\eqref{eq:bd_state} will be the starting point of calculating fidelity and Loschmidt echo in Sec.~\ref{sec:analytic_numerics}. In particular, 
\todo[inline]{say sth: connecting to what will follow etc. }