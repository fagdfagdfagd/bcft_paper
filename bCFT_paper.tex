\documentclass[preprint, prb]{revtex4-1}
\usepackage{amsmath}
\usepackage{amssymb}
\usepackage{graphicx}
\usepackage{enumitem}
\usepackage{bm}
\usepackage[font={footnotesize}]{caption}
\usepackage{todonotes}
\definecolor{darkblue}{rgb}{0.1,0.2,0.6} \definecolor{darkred}{rgb}{0.8,0.1,0.2}
\usepackage[colorlinks,citecolor=darkblue,linkcolor=darkred,urlcolor=darkblue]{hyperref}
\usepackage[all]{hypcap}

\renewcommand{\vec}[1]{\boldsymbol{\mathbf{#1}}}

\newcommand{\cf}{\textit{cf.} } 
\newcommand{\ie}{\textit{i.e.} } 
\newcommand{\eg}{\textit{e.g.} }
\newcommand{\vs}{\textit{vs.} } 
\newcommand{\etal}{\textit{et al.} }
\newcommand{\etc}{\textit{etc.} }

\begin{document}

\title{Bipartite Fidelity and Loschmidt echo of bosonic conformal interface}
 
\author{Tianci Zhou}
\email{tzhou13@illinois.edu}
\affiliation{University of Illinois, Department of Physics, 1110 W. Green St. Urbana, IL 61801 USA}

\author{Mao Lin}
\email{maolin2@illinois.edu}
\affiliation{University of Illinois, Department of Physics, 1110 W. Green St. Urbana, IL 61801 USA}

\date{\today}

\begin{abstract}
\end{abstract}

\maketitle

\section{Introduction}
% the last one to be finished.

\section{Bosonic Conformal Interface}
\subsection{General Formulation}
% M, S
% special case
\subsection{Physical Realization: Connecting Bosons of Different Compactification Radii}
% free boson + compact
\subsection{A Free Boson Lattice Model}
% cite Calabrese, introduce the model
% lattice model S matrix

\section{Bipartite Fidelity and Loschmidt Echo}
\subsection{Definition}
% definition
% general discussion

\section{Analytic and Numerical Results}
\subsection{Fidelity}
% analytic: conformal transformation
% numerical comparison 
\subsection{Loschmidt Echo}
% analytic: conformal transformation
% numerical comparison

\section{Discussion}


\begin{acknowledgments}
\todo[inline]{Thomas Faulkner, Xueda Wen, Shinsei Ryu, Romain Vasseur}
    TZ is supported by the National Science Foundation under grant number NSF-DMR-1306011.
    This work made use of the Illinois Campus Cluster, a computing resource that is operated by the
    Illinois Campus Cluster Program (ICCP) in conjunction with the National Center for
    Supercomputing Applications (NCSA) and which is supported by funds from the University of
    Illinois at Urbana-Champaign.
\end{acknowledgments}

\appendix
\section{Corrections to the Free Energy}
% two corrections:
% 1. staircase geometry
% 2. contribution from the boundary: cite Cardy 

\section{Numerical Computation of Bipartite Fidelity and Loschmidt Echo}
% basis transformation, Bogoliubov state, overlap(master theorem)
% Peschel_EE.pdf

\section{A Determinant Identity for the Boundary State Amplitude}
% copy notes app. C

\section{$\lambda_1 \rightarrow \lambda_2$ Boundary State Amplitude}
% copy notes app. B

\section{Alternative Approach to ${\rm DN} \rightarrow \lambda$ Amplitude}
% copy notes app. A


\bibliographystyle{plain}
\bibliography{bCFT}

\end{document}

%%% Local Variables: 
%%% TeX-PDF-mode: t
%%% End: 
