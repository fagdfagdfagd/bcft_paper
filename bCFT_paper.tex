\documentclass[reprint, prb]{revtex4-1}
\usepackage{amsmath}
\usepackage{amssymb}
\usepackage{graphicx}
\usepackage{enumitem}
\usepackage{bm}
%\usepackage[font={footnotesize}]{caption}
\usepackage[caption=false]{subfig}
\usepackage{todonotes}
\usepackage{comment}
\graphicspath{{./}{./images/}}
\definecolor{darkblue}{rgb}{0.1,0.2,0.6} \definecolor{darkred}{rgb}{0.8,0.1,0.2}
\usepackage[colorlinks,citecolor=darkblue,linkcolor=darkred,urlcolor=darkblue]{hyperref}
\usepackage[all]{hypcap}
\renewcommand{\vec}[1]{\boldsymbol{\mathbf{#1}}}

\newcommand{\cf}{\textit{cf.} } 
\newcommand{\ie}{\textit{i.e.} } 
\newcommand{\eg}{\textit{e.g.} }
\newcommand{\vs}{\textit{vs.} } 
\newcommand{\etal}{\textit{et al.} }
\newcommand{\etc}{\textit{etc.} }

\begin{document}

\title{Bipartite Fidelity and Loschmidt Echo of Bosonic Conformal Interface}
 
\author{Tianci Zhou}
\email{tzhou13@illinois.edu}
\affiliation{University of Illinois, Department of Physics, 1110 W. Green St. Urbana, IL 61801 USA}

\author{Mao Lin}
\email{maolin2@illinois.edu}
\affiliation{University of Illinois, Department of Physics, 1110 W. Green St. Urbana, IL 61801 USA}

\date{\today}

\begin{abstract}
\end{abstract}

\maketitle

\section{Introduction}


% quantum impurity problem,  boundary critical phenomena, Anderson orthogonality catastrophe
In (one-dimensional) quantum critical systems, the presence of physical boundary and isolated impurity weakly break the conformal symmetry. Simply put, they scatter and hence relate the otherwise independent modes and therefore demonstrate novel boundary critical phenomena\cite{cardy_boundary_2004}. Operators close to the boundary are interpreted as boundary condition changing operator\cite{oshikawa_boundary_1997,affleck_boundary_1997} in the boundary conformal field theory (CFT). Their correlation functions exhibit different critical exponent from the bulk\cite{cardy_conformal_1984}. One example is the "Anderson orthogonality catastrophe", where the core hole creates a potential that act as an impurity to the conduction band. The X-ray absorption rate will then have a power singularity of boundary exponent\cite{affleck_boundary_1997} at the resonance frequency. There are numerous impurity problem of this kind that has been studied in the last few decades, such as magnetic impurity in spin chain\cite{eggert_magnetic_1992}, boundary and impurity effects in Luttinger liquid\cite{fabrizio_interacting_1995}, entanglement of the defects\cite{peschel_entanglement_2005, igloi_entanglement_2009,calabrese_entanglement_2012} \etc.

% non-equilibrium dynamics, particle mediate the two parts, Cardy cut-and-join protocol, Vassur majarana
Recently, more attention has been paid to the non-equilibrium dynamics of the quantum impurity. The "cut-and-join" quench protocol is a popular framework to study the spreading of influence from the localized impurity (or boundary) cross the system. As shown in Fig.~\ref{fig:cut-and-join}, one first prepares the ground states of the critical chain A and B, and then connect them at $t = 0$ and evolve. Various quantities are used to detect the information in the quench process. For instance, Ref.~\onlinecite{calabrese_entanglement_2007, calabrese_quantum_2016} find a logarithmic increase of entanglement entropy in subsystem $A$, when both $A$ and $B$ are identical critical systems. And the authors ascribe the growth to the proliferation and propagation of quasi-particle excitation emitted at the joint. Ref.~\onlinecite{vasseur_universal_2014} take A to be a normal lead and B to be a topological superconductor in topological phase and calculate the Loschmidt echo (overlap of wavefunction, see below) of the evolution. In this mode, effectively it is the Majorana zero mode which mediates the left and right modes of the free fermion that are responsible for the universal power law decay of the Loschmidt echo with the boundary critical exponent. 

\begin{figure}[htb]
\centering
\includegraphics[width=\textwidth]{fig_cut_and_join.pdf}
\caption{Cut-and-join quench protocol. Left panel: prepare the ground states of two separated chains and joined them at $t = 0$, then do the time evolution with whole chain Hamiltonian. Right panel: spacetime diagram of the cut and join protocol. The solid line represents the boundaries of the two disconnected chains. It is totally reflective for the incident particles on both sides. The dashed line is the world line of the junction, which we will call interface. It could either be totally transparent or partially permeable, depending the types of theories of A and B.}
\label{fig:cut-and-join}
\end{figure}

% All of these are the same CFTs. we consider different CFTs, the mediating particle or primary field will be different, from the S matrix point of view.
In the path integral language, the "cut-and-join" protocol corresponds to a spacetime diagram as shown in Fig.~\ref{fig:cut-and-join}. The separating ground stats prepared before $t = 0$ are joined to form a new type of interface between them. Before the quench, the slit before $t = 0$ represents the boundary that are completely reflective to the injecting particles. The joining turns on the transmission from one side to the other. In the entanglement entropy and Loschmidt echo examples cited above\cite{calabrese_entanglement_2007, calabrese_quantum_2016, vasseur_universal_2014}, the two sides of the CFTs are the same (chiral fermion CFT in the case of \onlinecite{vasseur_universal_2014}) and the boundary becomes totally transparent after the joining. 

It would therefore be interesting to consider to connect two different CFTs in the "cut-and-join" protocol. The interface produced will interpolate between the total reflective and complete transparent ones. It can be realized as interface of two free compact Boson theories with different compactification radii\cite{bachas_permeable_2002}. There is also non-compact free boson/free fermion lattice realization, which was used to numerically and analytically\cite{peschel_exact_2012,sakai_entanglement_2008} study entanglement properties. In these models, there is a parameter $\lambda$ that is directly related to the transmission coefficient. In the compact Boson case, it is controlled by the ratio of the compactification radii and in the free lattice Boson case by ratio of masses. We expect it to be tunable in the a realistic experimental setting.

% fidelity and Loschmidt echo are quantities that extract boundary critical exponents, may reveal the nature of the mediating particle 
Fidelity and Loschmidt echo are the two quantities we compute to extract information from the quench process. Fidelity is the (square of) the overlap of the ground state wavefunction of the Hamiltonian before and after the quench. There is no dynamic information in fidelity, nevertheless its finite size effect also reveal the boundary critical exponent that maybe be diagnostic in Loschmidt echo computation. The Loschmidt echo is (square of) the overlap of the wavefunction before the quench and the one evolved at time $t$. The Loschmidt echo decays with a power law for the lack of length scale in the $t \rightarrow \infty$ limit. Its exponent has been calculated for various geometries and combinations of normal boundary conditions of the same CFT in \onlinecite{stephan_logarithmic_2013,stephan_local_2011}. What we do in this paper is to extend the analysis to the aforementioned parametric interface of (possibly) different CFTs. 

% structure of the paper.
\todo[inline]{structure of the paper}

%%% Local Variables:
%%% TeX-master: "bCFT_paper"
%%% TeX-PDF-mode: t
%%% End:


\section{Bosonic Conformal Interface}
\subsection{General Formulation}
\label{sec_sub:general_formulation}
We consider a conformal interface connecting two potentially different CFTs. The interface is located at $x=0$ and characterized by the ``gluing condition''
\begin{eqnarray}\begin{aligned}
\label{Def. of M}
\begin{pmatrix}
\partial_x\phi\\
\partial_t\phi
\end{pmatrix}_{x=0^-}
=M\begin{pmatrix}
\partial_x\phi\\
\partial_t\phi
\end{pmatrix}_{x=0^+}
\end{aligned}\end{eqnarray}
The interface is now penetrable. Instead of requiring the normal components of the stress tensor to vanish\cite{cardy_conformal_1984}, one only needs the stress tensor to be continuous across the defect. From the explicit expression of the stress tensor $T^{xt}=-\partial_x\phi\partial_t\phi$, one can show that\cite{bachas_permeable_2002} $M$ is an element of the Lorentz group $O(1,1)$. In terms of the boost parametrization, we have
\begin{eqnarray}\begin{aligned}
M_1(\theta)=\pm
\begin{bmatrix}
\lambda^{-1} & 0 \\
0 & \lambda
\end{bmatrix}\quad
M_2(\theta)=\pm
\begin{bmatrix}
0 & \lambda  \\
\lambda^{-1} & 0 
\end{bmatrix}
\end{aligned}\end{eqnarray}
where $\lambda=\tan\theta$ for $\theta\in\left[-\frac{\pi}{2},\frac{\pi}{2}\right]$. 

We note the following special cases. For $\theta=0,\pm\pi/2$, $\lambda$ appears to be singular and the fields on either side of the defect do not communicate to each other. For example, $M_1(0)$ implies $\partial_x\phi(0^+)=\partial_t\phi(0^-)=0$. For the two CFTs, this corresponds to Dirichlet boundary condition on the left, and Neumann condition on the right. Hereafter we shall denote it as `DN'. %Similarly one can check that $M_1(\pm\pi/2),M_2(0),M_2(\pm\pi/2)$ correspond to ND, NN, DD respectively. 
For $\theta=\pm\pi/4$, on the other hand, characterizes a perfectly transmitting defect. For example, there is effectively no defect in the case $M(\pi/4)$. For the other three cases, although the field $\phi$ may pick up a phase across the defect, the two counter propagating waves are still fully transmitted.

It proves useful to rewrite Eq.~\eqref{Def. of M} in the ``light cone'' coordinate $x^\pm\equiv t\pm x$, where the physical meaning of $\theta$ is most transparent. For clarity, we denote the field on the negative (positive) real axis as $\phi^1$ ($\phi^2$). {\bf\color{red}Tianci: Add in the right panel in Fig.~\ref{fig:cut-and-join} the field content?}. After some algebra, we have
\begin{eqnarray}\begin{aligned}
\label{Def. of S}
\begin{pmatrix}
\partial_+\phi^1\\
\partial_-\phi^2
\end{pmatrix}
=S
\begin{pmatrix}
\partial_-\phi^1\\
\partial_+\phi^2
\end{pmatrix}
\end{aligned}\end{eqnarray}
where 
\begin{eqnarray}\begin{aligned}
S_1(\theta)=\begin{bmatrix}
\cos 2\theta & \sin 2\theta \\
\sin 2\theta & -\cos 2\theta
\end{bmatrix}\quad
S_2(\theta)=\begin{bmatrix}
-\cos 2\theta & \sin 2\theta \\
-\sin 2\theta & -\cos 2\theta
\end{bmatrix}
\end{aligned}\end{eqnarray}
The scattering matrix $S_1$ and $S_2$ relate the incoming and outgoing waves at the defect. For example, $S_1(0)$ implies that $\partial_\pm\phi^{1}=\partial_\pm\phi^{2}$ as discussed above. For other nontrivial values of $\theta$, the defect becomes partially-transmitting. The transmission and reflection coefficients, which are determined by $\theta$, can be read off from the scattering matrices.

{\bf\color{red}As we shall show explicitly in Sec.~\ref{sec: A Free Boson Lattice Model}, the scattering matrices are independent of the wavelengths, as required by the scale invariance.}

% M, S
% special case
\subsection{A Free Boson Lattice Model}
\label{sec_sub:free_boson_lattice}
% cite Calabrese, introduce the model
% lattice model S matrix

In this section, we consider a lattice model with bosonic interface at the center\cite{peschel_exact_2012,calabrese_entanglement_2012}, which reduces to the one considered in Sec.~\ref{sec_sub:general_formulation} in the continuum limit\cite{sakai_entanglement_2008}. Therefore, it serves as a numerical tool to check our analytic results in Sec.~\ref{sec:analytic_numerics}. 


We consider two harmonic chains with bosonic field $\phi_i$ and conjugate momentum $\pi_i$ on lattice site $i$. The left and right chains are connected between site $0$ and $1$ with the Hamiltonian
\begin{equation}
H = \frac{1}{2} \sum_i \left(\pi_i^2  + ( \phi_i - \phi_{i+1} )^2 \right) +  \frac{1}{2} \begin{pmatrix}  \phi_0, \phi_1 \end{pmatrix}
\Sigma
\begin{pmatrix}
\phi_0 \\
\phi_1 
\end{pmatrix}
\end{equation}
%\begin{eqnarray}\begin{aligned}
%H = \frac{1}{2} \sum_i \pi_i^2  +  \frac{1}{2} \sum_{i\ne 0 }  ( \phi_i - \phi_{i+1} )^2  +  \frac{1}{2} \begin{pmatrix}  \phi_0, \phi_1 \end{pmatrix}
%\begin{bmatrix}
%1 + \Sigma_{11}  & \Sigma_{12} \\
%\Sigma_{21} &  1 + \Sigma_{22} \\
%\end{bmatrix}
%\begin{pmatrix}
%\phi_0 \\
%\phi_1 
%\end{pmatrix}
%\end{aligned}\end{eqnarray}
where the $2\times2$ matrix $\Sigma$ parameterizes two-site interaction. In fact $\Sigma$ gives a simple relation with the scattering matrix defined in Sec.~\ref{sec_sub:general_formulation}, and thus the Hamiltonian is the lattice regularization of the bosonic interface discussed there. We perform the scattering analysis in App.~\ref{app:interface_free_boson}, and the scale invariant S-matrix reads
\begin{equation}
S = \frac{1}{1 - e^{-2ika } } ( -2i \sin ka ) \Sigma
 = - e^{ika} \Sigma
\end{equation}
where $a$ is the lattice constant. In the continuum limit $a\rightarrow0$, the matrix $\Sigma$ can be parameterized as
\begin{equation}
\Sigma = -\lim_{a \rightarrow 0 } S = 
\begin{bmatrix}
\frac{\lambda^2- 1}{1 + \lambda^2} & \frac{-2\lambda }{1 + \lambda^2} \\
\frac{-2\lambda }{1 + \lambda^2} & \frac{1- \lambda^2}{1 + \lambda^2} \\
\end{bmatrix}
\end{equation}

{\bf\color{red}

say in the long time, what happens? 

say the special case in the boundary
}

This is the matrix we will use for the numerics. The results will be presented in Sec.~\ref{sec:analytic_numerics}.


%%% Local Variables:
%%% TeX-master: "bCFT_paper"
%%% TeX-PDF-mode: t
%%% End:


\section{Bipartite Fidelity and Loschmidt Echo}
\label{sec:analytic_numerics}
\subsection{Definition}

Bipartite fidelity is the overlap of the groundstates of two Hamiltonians. In the path integral language, the groundstates can be produced by imaginary time evolution. If we take the horizontal axis as the Euclidean time, the fidelity can be diagrammatically represented as in Fig.~\ref{fig:fidel}. Treating it as a partition function, we analyze its finite size dependence on the total chain size $L$. The logarithmic fidelity should then follow the general discussion of Cardy and Peschel\cite{cardy_finite-size_1988} of the finite size dependence of free energy and give a term proportional to $\ln L$
\begin{equation}
\mathcal{F} = - \ln \langle \phi_1 |\phi_2 \rangle^2 = - 2 \ln Z  \sim  \ln L 
\end{equation}
The proportionality constant is the exponent characterizing the interface. 

\begin{figure}[h]
\centering
\includegraphics[width=\textwidth]{fig_fidel.pdf}
\caption{Fidelity of connecting two CFTs. The horizontal axis the imaginary time. Evolution on the two infinitely extended side produces the groundstate of the disconnected and connected chain Hamiltonians. The right diagram is the result of folding the lower part of the diagram up, so that all the boundary are now boundary states.}
\label{fig:fidel}
\end{figure}

\todo[inline]{echo diagram folding}

The Loschmidt echo is defined to be defined as the overlap of the ground state and the quenched states, 
\begin{equation}
\mathcal{L} = |\langle \phi_{\rm gnd}  | e^{-i Ht } | \phi_{\rm gnd} \rangle|^2
\end{equation}
We have drawn the path integral definition of this quantity in Fig.~\ref{fig:cut-and-join}; for convenience of the latter conformal mapping, we turn the diagram by a right angle and take the horizontal axis as time. The slits represents disconnected boundary conditions like Dirichlet and Neumann, the dashed line represent partially transmitting boundary condition parameterized by $\lambda$. We may as well fold the lower half plane up and everything on the real axis should be regarded as boundary states. Again by Cardy and Peschel\cite{cardy_finite-size_1988}, the logarithm of the Loschmidt echo is the free energy and should have logarithmic dependence on the characterizing length scale in the problem. In other words
\begin{equation}
\mathcal{L} = - \ln |\langle \phi_{\rm gnd}  | e^{-i Ht } | \phi_{\rm gnd} \rangle|^2 \propto \ln \tau 
\end{equation}



%%% Local Variables:
%%% TeX-master: "bCFT_paper"
%%% TeX-PDF-mode: t
%%% End:

\subsection{Notation of Boundary Conditions}
\label{sec:notation}

We use chemical reaction style to represent the change of boundary conditions. Taking the example of the echo diagrams in Fig.~\ref{fig:echo}, there are three boundary conditions $a,b,c$ in the folding picture, which represents the status on two ends of the chain before and after the quench. The choice of a uniform $c$ boundary condition on the far end of the chain is to isolate the effect coming from the bcc on the conformal interface. The process $a + c \rightarrow b + c$ represents the change of boundary condition from the combination $a$/$c$ on the two ends to $b$/ $c$ after the quench. Since each letter can take a general conformal interface defined by the $S$ matrix in Eq.~\eqref{eq:S1_S2}, we denote it as
\begin{equation}
S_a( \theta_a ) + S_c( \theta_c) \rightarrow S_b( \theta_b )  + S_c( \theta_c ) 
\end{equation}
Analytic computation can be performed when we neglect the boundary condition $c$, assuming $L \gg \tau$. To remove the bcc operator from $c$ to $a$ at infinity, we take $a = c$ in the numerical calculation. 

In the ``cut-and-join" protocol we considered, $a$ should be one of `DD', `DN', `ND', `NN', $b$ is taken to be $S_1( \theta )$ or $S_2( \theta )$. The physical situation of connecting two compact bosons and our lattice model corresponds to the choice of $S_1( \theta)$, we also reserve the notation $\lambda$ for this type of boundary condition. For instance, the notation for the process presented in Fig.~\ref{fig:echo} is
\begin{equation}
{\rm DN }  + {\rm DN} \rightarrow \lambda + { \rm DN} 
\end{equation}
Another interesting case is to take $a$ or $c$ to be completely transmitting interface, i.e. $S_2( \frac{\pi}{4} )$. This corresponds to the traditional periodic boundary condition and we use symbol 'P' to denote it. 


%%% Local Variables:
%%% TeX-master: "bCFT_paper"
%%% TeX-PDF-mode: t
%%% End:

\subsection{Analytic Evaluation}
\label{sec_sub:analy_eval}

In this subsection, we relate the free energy to the amplitudes between boundary states and present the analytic results. 

We notice that there is only one apparent length scale in these diagrams -- the finite size $L$ for fidelity and imaginary time $\tau$ for the Loschmidt echo. These are the characteristic size of the corners at the tip of the slits. Regulators are necessary in keeping track of the scale dependence, otherwise a dilation transformation can rescale both $L$ and $\tau$ to $1$ and drop those scales. The introduction of regulators is also physically sensible when considering the lattice realization of the systems. 

We thus add small semi-circles around the points where the bcc operators reside, and then apply a series of conformal mappings. 

For the fidelity case, the regulators as well as the conformal maps are depicted in Fig.~\ref{fig:fidel-map}. 
\begin{figure}[h]
\centering
\includegraphics[width=\columnwidth]{fig_fidel-map.pdf}
\caption{Mapping from a strip to the upper half plane $\xi  = \exp( \frac{\pi z}{L} ) $. The two black dots represent possible locations of boundary condition changing (bcc) operators. The dot inside the blue semi-circle has coordinate $\xi = 1$, which is the image of the point connecting $a$ and $b$ boundaries. The other dot $\xi = 0$ corresponds to the connection between $a$ and $c$ boundaries at $- \infty$. To evaluate the diagram, we add the outer blue semi-circle centered at $\xi = 1$ with radius $R_{\xi}$ to be the IR cut-off and map it to the cylinder with $w = \ln(\xi - 1)$}
\label{fig:fidel-map}
\end{figure}
We add a small blue semi-circle to the folded strip in Fig.~\ref{fig:fidel} as the UV regulator and map it to the upper half plane using $\xi  = \exp( \frac{\pi z}{L} )$. Then both $\xi = 0$ and $1$ can host bcc operators. We assume $a = c$ such that the only bcc operator on the real axis is the one enclosed by the blue semi-circle around $\xi = 1$. In order to evaluate this diagram, we add another semi-circle centered around $\xi = 1$ with radius $R_{\xi}$ (this will introduce a correction as explained in App.~\ref{app:F_correction}), and map it to a cylinder of height $\pi$ on the right by $w = \ln ( \xi- 1)$. Finally the cylinder diagram can be viewed as an imaginary time path integral amplitude between the boundary states $b$ and $a$
\begin{equation}
\label{eq:partition_fun}
Z_{ab} = \langle a | e^{-\pi H } |b \rangle.
\end{equation}
The two end points of the $\epsilon$ radius semi-circle on the $z$ plane are mapped to
\begin{equation}
\exp( \pm \pi \frac{\epsilon}{ L}  ) \sim 1 \pm \pi \frac{\epsilon}{L} .
\end{equation}
The bigger blue semi-circle intersects the real axis at $1 \pm R_{\xi}$ and so the width of the cylinder is 
\begin{equation}
\label{eq:fidel_cyd_width}
\ln R_{\xi} - \ln \frac{\pi \epsilon}{L} = \ln L + \text{constant}.
\end{equation}

The Loschmidt echo can be evaluated in the same way. Again, we introduce two semi-circles (blue in Fig.~\ref{fig:H-tau_fold}) as regulators and then perform the conformal transformation shown in Fig.~\ref{fig:H-tau_fold}. From $z$ plane to the $\xi$ plane, we use $\xi = \frac{z}{\tau - z}$ to map the two slits to half of an annulus, which is the same as the fidelity case. With one more conformal mapping $w = \ln \xi$, the diagram again becomes the cylinder partition function between the two boundary states. 

The height of the cylinder is still $\pi$. In the $\xi$ plane, the coordinates of the two end points of the small semi-circle are $\frac{\pm \epsilon}{ \tau \mp \epsilon} \sim \frac{\pm \epsilon}{ \tau }$, while those of the larger semi-circle are $\mp \frac{\tau \pm \epsilon}{\epsilon} \sim \frac{\mp \tau}{\epsilon}$. Hence the width of the cylinder is
\begin{equation}
\label{eq:echo_cyd_width}
\ln \frac{\tau}{\epsilon} - \ln \frac{\epsilon}{\tau } = 2 \ln \tau + \text{constant}.  
\end{equation}

\begin{figure}[htb]
\centering
\includegraphics[width=\columnwidth]{fig_H-tau_fold}
\caption{The dashed (solid) lines are gluing (completely reflective) boundary conditions. Red arrows are the directions of Hamiltonian flow that propagates the dashed line boundary state to the solid line boundary state. Left: Diagram of the Loschmidt echo that reduces to a partition function with imaginary time in the horizontal direction. The blue semi-circles of radius $\epsilon$ are the UV regulators and they are identified as periodic boundaries in the direction perpendicular to the red arrow (equal time slice). Middle: Image of the map $\xi = \frac{z}{\tau - z}$. The two semi-circles have radii $({\tau}/{\epsilon})^{\pm1}$ respectively.  Right: Image of $w = \ln \xi$. It is a cylinder by identifying the blue lines and the standard radial quantization procedure can be applied. }
\label{fig:H-tau_fold}
\end{figure}


One subtlety of the above description is that the two semi-circles in the center diagrams of Fig.~\ref{fig:fidel-map} and Fig.~\ref{fig:H-tau_fold} are not precisely concentric. This can be resolved by the following observation. There exists a conformal map $\zeta(\xi)$ that maps the non-concentric circles to the two standard concentric circles of radii $1$ and $R$ ($R>1$) on the $\zeta$ plane\cite{brown_complex_2009}. Then the logarithmic map $w = \ln \zeta$ produces a cylinder of width $\ln R$. In our case, since the height of the cylinder is always $\pi$, the width of the cylinder is a conformal invariant that only depends on the cross ratio of the half annulus. The four intersection points of two standard concentric circles on $\zeta$ plane are $(\pm 1,0)$ and $(\pm R,0)$, whose cross ratio is
\begin{equation}
\eta = \frac{(1 + R)^2}{(1 - R)^2}. 
\end{equation}
Hence the width of the cylinder is $\ln \frac{\sqrt{ \eta } - 1}{\sqrt{ \eta} + 1}$. Since conformal transformation preserves the cross ratio, the result is the same if we use the cross ratio of the slightly non-concentric diagrams of Fig.~\ref{fig:fidel-map} and Fig.~\ref{fig:H-tau_fold}. The calculation in Eq.~\eqref{eq:fidel_cyd_width} and Eq.~\eqref{eq:echo_cyd_width} equivalently use the leading order approximation to $\eta$ in the respective geometries and thus get the leading order term in the width of the cylinders. The slight deviation to the precise concentric geometry will only bring in $\frac{\epsilon}{L}, \frac{\epsilon}{\tau}$ corrections to $\eta$ and the width parameter, so will not affect the fidelity and echo exponents. 

For the rest of this section, we should denote the width of the cylinder as $\beta$. After obtaining the partition function on it, we should set $\beta = 2 \ln L$ or $ 4 \ln \tau$ because the fidelity and Loschmidt echo are both square of the amplitudes.

The actual boundary conditions on the blue lines, which are the regulators in Fig.~\ref{fig:fidel-map} and Fig.~\ref{fig:H-tau_fold}, are not important in the leading order. Taking Fig.~\ref{fig:fidel-map} for example, rather than using Eq.~\eqref{eq:partition_fun}, we can alternatively view the right panel as the amplitude between the two blue boundary states $|1\rangle$ and $|2\rangle$
\begin{equation}
Z_{ab} =  \langle 1 | e^{ - \beta H_{ab}} | 2 \rangle  , 
\end{equation}
where $H_{ab}$ is the Hamiltonian with boundary condition $a$ and $b$. Since $\beta$ is taken to be large, we expect the imaginary time evolution (which is horizontal in this case) to project out only the groundstate $|0_{ab}\rangle $. Hence the free energy is
\begin{equation}
\begin{aligned}
F &=  - \ln Z_{ab}  \sim - \ln \langle 1 |0_{ab} \rangle \langle   0_{ab}  |e^{ - \beta H_{ab}}|0_{ab} \rangle \langle 0_{ab} | 2 \rangle \\
&= \beta  E_c - \ln \langle 1| 0_{ab} \rangle  - \ln \langle 0_{ab}  |2 \rangle,
\end{aligned}
\end{equation}
where $E_c$ is the groundstate/Casimir energy of $H_{ab}$. We see that different choices of the boundary conditions only change the term independent of $\beta$. Thus in the leading order we can choose any boundary conditions. The one we pick is the simplest one: the periodic boundary condition that identifies the two blue lines. 

With these simplifications, we now set up the partition function calculation of the general process $S_a( \theta_1 ) \rightarrow S_b( \theta_2)$. We define a set of bosonic operators related to the $a^i_n$s in Eq.~\eqref{eq:di_mode_expansion} through
\begin{equation}
\begin{aligned}
b^i_n &= \frac{a^i_n}{\sqrt{n}} \quad (b^i_n)^{\dagger} &= \frac{a^i_{-n}}{\sqrt{n}} \\
\bar{b}^i_n &= \frac{\bar{a}^i_n}{\sqrt{n}} \quad (\bar{b}^i_n)^{\dagger} &= \frac{\bar{a}^i_{-n}}{\sqrt{n}} \\
\end{aligned}
\end{equation}
for $n > 0 , i = 1, 2$, and group them compactly with the vector notation
\begin{equation}
\begin{aligned}
\vec{b}_i &= ( b^i_1, b^i_2, \cdots )^\top, \qquad \,\,\quad \vec{\bar{b}}_i = ( \bar{b}^i_1, \bar{b}^i_2, \cdots )^\top\\
\vec{b}^\dagger_i &= ( (b^i_1)^\dagger, (b^i_2)^\dagger, \cdots )^\top, \quad\vec{\bar{b}}^{\dagger}_i = ( (\bar{b}^i_1)^\dagger, (\bar{b}^i_2)^\dagger, \cdots )^\top.
\end{aligned}
\end{equation}
The boundary state in Eq.~\eqref{eq:bd_state} is then 
\begin{equation}
\exp\Big\{  (\vec{b}_1^{\dagger} \vec{b}_2^{\dagger} ) R_a( \theta )   
\begin{pmatrix}
  \vec{\bar{b}}_1^{\dagger}\\
  \vec{\bar{b}}_2^{\dagger}
\end{pmatrix}\Big\}  |0  \rangle ,
\end{equation}
where $R_a( \theta ) = -S_a \otimes \mathbb{I}$. Using even a lazier notation $\vec{b} = ( \vec{b}_1, \vec{b}_2 )^\top$, we have
\begin{equation}
\label{eq:bd_state_matrix}
|a \rangle = \exp\Big\{  \vec{b}^{\dagger} R_a( \theta )    \vec{\bar{b}}^{\dagger} \Big\} |0 \rangle .
\end{equation}
The matrix notation here should be understood as a bilinear expression. For example, $\vec{b}^{\dagger} R \bar{\vec{b}}^{\dagger}$ actually means $\sum_{ij}b^\dagger_iR_{ij}\bar{b}_j^\dagger$ where the dagger does not transpose the vector.

The Hamiltonian of the folding picture has the mode expansion in terms of the $b_n$ (with periodic boundary conditions)
\begin{equation}
\begin{aligned}
  H &= \frac{2\pi}{\beta} (L_0 + \bar{L}_0) =  \frac{4\pi}{\beta}  L_0 \\
  &= \frac{4\pi}{\beta}\sum_{\substack{n > 0\\ i=1,2} }  n (b^i_n)^{\dagger} b^i_n \\
  &=  \frac{1}{\pi}(\vec{b}_1^{\dagger} \vec{b}_2^{\dagger} ) (\mathbb{I}_2  \otimes M)
\begin{pmatrix}
  \vec{{b}_1}\\
  \vec{{b}_2}
\end{pmatrix}\\
 &= \frac{1}{\pi} \vec{b}^{\dagger}  (\mathbb{I}_2  \otimes M)  \vec{b} ,
\end{aligned}
\end{equation}
where $L_0+ \bar{L}_0$ are the dilation operator in CFT and we have used the condition $L_0 = \bar{L}_0$ when restricted to the space of the boundary states. The infinite dimensional matrix $M$ is
\begin{equation}
M =  \frac{4\pi^2}{\beta} \text{diag}( 1, 2, \cdots ).
\end{equation}
The partition function in Eq.~\eqref{eq:partition_fun} becomes
\begin{equation}
\label{eq:Zab-bd}
\begin{aligned}
Z_{ab} =& \langle b | e^{-\pi H} | a \rangle \\
=& \langle 0 | \exp\Big\{  \vec{b} R_b( \theta )    \vec{\bar{b}} \Big\}  \exp\Big\{ -\vec{b}^{\dagger}  (\mathbb{I}_2  \otimes M)  \vec{b} \Big\} \\
&\exp\Big\{  \vec{b}^{\dagger} R_a( \theta )    \vec{\bar{b}}^{\dagger} \Big\} | 0\rangle .
\end{aligned}
\end{equation}

In App.~\ref{app:lambda_12}, we obtained the leading order term in the free energy associated with Eq.~\eqref{eq:Zab-bd}. This expression is also obtained by an alternative Casimir energy calculation in App.~\ref{app:gnd_dn_lambda} for one set of the boundary conditions. A na\"ive application of the result however will lead to an apparent contradiction. One notable example is that when $a = b  = {\rm P}$, the free energy given by App.~\ref{app:lambda_12} is $- \frac{1}{12}\beta$, which should actually be {\it zero} because this is the (regularized) free energy on a plane without any interface. Physically this corresponds to the situation that the boundary condition does not change after joining the two chains. Hence the Loschmidt echo will stay at $1$ and the free energy is $0$. This motivates a shift to the free energy
\begin{equation}
\mathcal{F} = - \ln Z_{ab} ( \beta ) + \frac{1}{12} \beta ,
\end{equation}
where $\frac{1}{12}\beta$ is the value of $ \ln Z_{ab} ( \beta )$ when $a = b = {\rm P}$. A more careful inspection in App.~\ref{app:F_correction} shows the origin of the shift: part of it comes from the outer semi-circles in the middle panel of Fig.~\ref{fig:fidel-map} and Fig.~\ref{fig:H-tau_fold}, and another part comes from the non-homogeneous term in the conformal transformation of the stress tensor from annulus to cylinder. 

After incorporating this shift, for the process ($c$ is assumed to be the same as $a$)
\begin{equation}
\label{eq:S_i_S_j}
S_i( \theta_1 ) \rightarrow S_j( \theta_2 ) ,
\end{equation}
the free energy is
\begin{equation}
\mathcal{F}( \beta )  = 
\left\lbrace
\begin{aligned}
  &\frac{1}{2}(|x| - x^2 )\beta  \quad &i = j \\
  &\frac{1}{16}\beta   \quad &i \ne j ,  \\
\end{aligned} \right. \quad x = \frac{\theta_2 - \theta_1}{\pi} .
\end{equation}
We can then set $\beta = 2 \ln L$ and $ 4 \ln t$ (after analytic continuing to real time) to get the fidelity and echo exponent. 

As analyzed in Sec.~\ref{sec:analytic_numerics}, $S_2( \theta)$ interpolates between DD and NN, $S_1( \theta )$ interpolates between DN and ND. In the region accessible to the numerical calculation in the lattice model, we choose the process ${\rm DD} \rightarrow  \lambda$ to verify
\begin{equation}
\label{eq:result_DDDD}
\mathcal{F} = 
\left\lbrace
\begin{aligned}
\frac{1}{8}\ln L  &\quad\text{fidelity}  \\
\frac{1}{4}\ln t   &\quad \text{echo} .  \\
\end{aligned} \right.  
\end{equation}
The same results have already been obtained for $\lambda = {\rm P}$\cite{stephan_logarithmic_2013,stephan_local_2011,vasseur_universal_2014,vasseur_crossover_2013,kennes_universal_2014}. Another process $\rm{DN} \rightarrow \lambda$ {\iffalse \color{red} in Eq.~\eqref{eq:DNDN}\fi} is used to verify
\begin{equation}
\label{eq:result_DNDN}
\mathcal{F} = 
\left\lbrace
\begin{aligned}
 (x - x^2 )\ln L   &  \quad {\rm fidelity} \\
 2(x - x^2 ) \ln t  & \quad \text{ Loschmidt echo} , \\
\end{aligned} \right. 
\end{equation}
where $\lambda = \tan \theta$ and $x = \frac{\theta}{\pi}$. 

We also use a more artificial process ${\rm P} \rightarrow \lambda$ to check the shift of the curve
\begin{equation}
\label{eq:periodic-case}
\mathcal{F} = 
\left\lbrace
\begin{aligned}
  \Big(|x-\frac{1}{4}| - (x-\frac{1}{4})^2 \Big)\ln L   &  \quad {\rm fidelity} \\
  2\Big(|x-\frac{1}{4}| - (x-\frac{1}{4})^2 \Big) \ln t  & \quad \text{ Loschmidt echo} .\\
\end{aligned} \right. 
\end{equation}



%%% Local Variables:
%%% TeX-master: "bCFT_paper"
%%% TeX-PDF-mode: t
%%% End:

\subsection{Numerical Results and Comparison}
In order to check our analytic results, we reconsider the lattice model introduced in Sec.~\ref{sec_sub:free_boson_lattice}. In particular, we calculate the Loschmidt echo and bipartite fidelity for various joining boundary conditions determined by $S_1(\theta)$ or $S_2(\theta)$. We refer the reader to App.~\ref{app:comp_fid_echo} for the technical details. 

\begin{figure}[h]
\includegraphics[width=1\columnwidth]{DDDD_fit.pdf}
\caption{The slope of the free energy for Loschmidt echo in the process $\text{DD}+\text{DD}\rightarrow\lambda+\text{DD}$, with gluing condition $S_1(\theta)$ where $\theta\in(0,\frac{\pi}{2})$. We work with total system size $N=30k$ sites, with parameters $m=10^{-8}$,$k=1$. The lattice constant is set to unity. The blue dots are the numerical results and the red line is the analytical result. As predicted, the slopes are equal for different values of $\theta$. Inset: An example of Loschmidt echo with $\theta=0.02\pi$ shown in log scale. The dashline denotes the power law with $t^{-0.25}$. The curve fitting method is described in the main text.}
\label{fig:DDDD}
\end{figure}

We start with the process $\text{DD}+\text{DD}\rightarrow\lambda+\text{DD}$ and the numerical simulation is performed for $\theta=0.02n\pi$, $n=1,...,25$. The results for Loschmidt echo is shown in Fig.~\ref{fig:DDDD}, and numerical parameters are given in the caption. The green dots correspond to numerical results for the slope of free energy, and the solid line shows the analytical result for echo in Eq.~\eqref{eq:result_DDDD}. The inset shows the double logarithmic plot of Loschmidt echo versus time for the case $\theta=0.02\pi$. The dashline denotes the expected power law decay $\mathcal{L}(t)\sim t^{-0.25}$. For the curve fitting, we randomly pick a data point near $t=10$ and fit the lower half of the data. This is repeated five times and we calculate the mean and standard deviation for the exponents. With the data processing method described, the error bar in the main figure is invisible on the data points. We have performed the identical simulation for the process $\text{NN}+\text{NN}\rightarrow\lambda+\text{NN}$ and obtained identical result shown in Fig.~\ref{fig:DDDD} (the initial decay of Loschmidt is different but has the same exponent at larger time). We note one caveat for this case with Neumann boundary conditions on both sides of the chain. Because of the zero mode, it is expected (indeed we saw) that the numerical simulation may not be stable. The problem can be avoided by adding a small mass regulator $m=10^{-8}$. Similar numerical analysis can be applied to calculate the fidelity which matches the result in Eq.~\eqref{eq:result_DDDD}. We shall not show the result here. 

Next we analyze the process $\text{DN}+\text{DN}\rightarrow\lambda+\text{DN}$ in which the joining boundary condition is determined by $S_1(\theta)$. There is one subtly which is that $S_1(0)$ corresponds to `DN' and the process reduces to $\text{DN}+\text{DN}\rightarrow\text{DN}+\text{DN}$. The zero mode will cause the numerical simulation unstable near $\theta=0$, and it turns out that the mass regulator cannot resolve this problem. Rather, we shall shift one of the `DN' slightly and consider the following process
\begin{eqnarray}\begin{aligned}
S_1(0)+S_1(\delta\theta)\rightarrow S_1(\theta)+S_1(\delta\theta)
\end{aligned}\end{eqnarray}
where we take $\delta\theta=0.003\pi$ and $\theta=0.02n\pi$, $n=1,...,25$. We shall discuss the introduction of extra bcc at $-\infty$ due to this shift in Sec.~\ref{sec:disc}. We show the numerical result for Loschmidt echo in Fig.~\ref{fig:DDNN}(a). The slope of the free energy follows a quadratic relation as predicted in Eq.~\eqref{eq:result_DNDN}. The deviation from the analytic result (without the shift) near $\theta=0$ has been greatly suppressed because of $\delta\theta$ introduced manually. The numerical results near $\theta=0.5\pi$ will match the analytical ones more precisely if we were to set $\delta\theta=0$. 

Next we perform the numerical simulation for $\text{DN}+\text{DN}\rightarrow\lambda+\text{DN}$ with $\theta\in(0,\frac{\pi}{2})$.  We recall that  

{\color{red}say why echo bad, say we shift it that is why dot below curve slightly near pi, say there is extra bcc, say why fidelity better, say to confirm we do fidelity as shown in Fig.~\ref{fig:DDNN}(b) and it is good, confirm our conjecture. }

We further consider the case $\text{P}+\text{P}\rightarrow\lambda+\text{P}$, and the numerical results for Loschmidt echo and fidelity are shown in Fig.~\ref{fig:PPPP}. 


\begin{figure}
  \centering
\includegraphics[width=1\columnwidth]{DDNN_fit.pdf}
\includegraphics[width=1\columnwidth]{DN_DN2tan.pdf}
    \caption{The slope of the free energy for Loschmidt echo (a) and bipartite fidelity (b) in the process $\text{DN}+\text{DN}\rightarrow\lambda+\text{DN}$. The total system size is $N=30k$ sites with the same parameters in Fig.~\ref{fig:DDDD}. The numerical value of slopes follow a quadratic relation as predicted in Eq.~\eqref{eq:result_DNDN}. For the deviation from the analytic results in (a), see the discussion in the main text. Inset in (a): From the top to bottom, we show the power law decay of Loschmidt echo with $\theta=0.02\pi, 0.12\pi,0.24\pi$ and $0.48\pi$. We use the same curve fitting method as described for Fig.~\ref{fig:DDDD}.}
      \label{fig:DDNN}
    \todo[inline]{Tianci: make the plots have the same style?}
\end{figure}


\begin{figure}
  \centering
  \includegraphics[width=1\columnwidth]{p_p2tan.pdf}
    \includegraphics[width=1\columnwidth]{p_p2tan.pdf}
    \caption{daf}
    \label{fig:PPPP}
    \todo[inline]{Mao: Put the PPPP for echo; Tianci: make the plots have the same style?}
\end{figure}



\section{Discussion}
\label{sec:disc}

% relation between fidelity and echo

In all the computation we have done, the fidelity results can be converted to that of the Loschmidt echo by the recipe $ \ln L \rightarrow 2 \ln \tau$. The numerical factor $2$ comes from the fact that the Loschmidt echo has two slit tips. Other than that, we see that they probe the same finite size effect of the free energy associated with the new interfaces. Our purpose of computing the (simpler) fidelity is diagnostic and so in the following we will be mainly discuss the echo properties. 

% analytic results type 1 -> type 2, c/4; 
In the Sec.~\ref{sec_sub:analy_eval}, we have presented the analytic results for the general process (assuming the far end boundary condition $c$ is the same as prior-quench interface $a$) 
\begin{equation}
 S_i( \theta_1 ) \rightarrow S_j( \theta_2 )
\end{equation}
We find that if the conformal interfaces are of different types, i.e. $i \ne j$, the echo is always $\frac{1}{4} \ln t$, regardless of the theta angle. The two types of conformal interface do not talk to each other because they impose on different fields. If we treat $S_1$ as a combination of Dirichlet and Neumann boundary conditions imposed on the rotated $\phi$ fields as in Eq.~\eqref{eq:rotation_a_basis}, then $S_2$ one of them on the duel field of $\phi$. In the derivation of the $M$ matrix, these two corresponds to the parts of the Lorentz group and can't be connected by taking singular value of $\lambda$. It is then reasonable to find an universal echo between them. The special value of $DD \rightarrow P$ also agrees with the existing general CFT result with transparent interface\cite{stephan_logarithmic_2013,stephan_local_2011,vasseur_universal_2014,vasseur_crossover_2013,kennes_universal_2014}. 


% x dependence, DN -> ND agrees with existing result;
More interesting case is when the boundary conditions are of the same type, which we verified the quadratic angle dependence numerically for the process 
\begin{equation}
{\rm DN} + {\rm DN} \rightarrow \lambda + {\rm DN} 
\end{equation}
which gives a series of new bcc operator dimensions of the $c = 2$ boson boundary CFT. Unfortunately, our numerical curve of the Loschmidt echo is not as perfect as the fidelity calculation. This is very likely due to the ambiguity of the free boson zero mode, as moving slight way from the point remedy the situation. And numerical study also shows that the outer boundary condition, which in the large system size limit should not impact the system, does change the scaling dimensional in a way that is not captured by our analytic computation. Generally speaking, the boundary condition on the far end may introduce additional bcc operator and thus change the free energy. It would be interesting to have a CFT calculation to reproduce the better numerical result of
\begin{equation}
{\rm ND} + {\rm DN} \rightarrow \lambda + {\rm ND} 
\end{equation}

Nevertheless the special case of ${\rm DN} \rightarrow {\rm P}$ agrees with the Ref.~\onlinecite{kennes_universal_2014,stephan_logarithmic_2013}, where the difference of this and the $DD \rightarrow {\rm P}$ one  $\frac{3}{8}- \frac{1}{4} = \frac{1}{16} $ is interpreted as the bcc operator dimension between D and N. The picture is complete in this full ${\rm DN}$ to $\lambda $ transition. 

% connecting Luttinger liquid(quantum wire) using votage to control the boundary condition. X-ray edge singularity. examples include Taylor santo's topological phases. Hope to have experimental setup. 

Aside from free lattice boson model for the numerical purposes, these set of boundary conditions can be realized in connecting two compact bosons. There are already numerous theoretical and experimental work on the boundary conditions of Luttinger liquid\cite{schmeltzer_zero_1999,anfuso_luttinger_2003,voit_bounded_2000,fabrizio_interacting_1995,egger_applying_1998}, which is the universal compact boson theory of the (Bosonized) one-dimensional electron gas\cite{giamarchi_quantum_2015}. For example, gate voltage \cite{egger_applying_1998} may be used to twist the left and right modes of the boson to create a boundary condition interpolating between the normal open and fixed boundary conditions. The interface studied in this paper is a generalization which (in the folding picture) twist the two independent Bosons (two left modes plus two right modes) on the two sides. A X-ray edge singularity experiment in a quantum wire system, which uses ion to switch on and off the boson interfaces should be plausible. 


%%% Local Variables:
%%% TeX-master: "bCFT_paper"
%%% TeX-PDF-mode: t
%%% End:



\section{Conclusion}
\label{sec:conclusion}

In this paper, we analyze a class of boson conformal interfaces by computing the fidelity and Loschmidt echo. 

We begin by classifying the boundary states by two types of $S$ matrices, where the conventional DD, NN boundary conditions belong to $S_1$ while DN and $P$ belongs to $S_2$. A harmonic chain model allows us to realize part of these boundary conditions, including the partially-transimitive ones on a concrete setting. 

The study of the Loschmidt echo are a generalization of the conventional "cut-and-join" protocol. Its power law decay exponent is related to the bcc operator scaling dimension that now can depend on the tune-able transmission coefficient. Analytic computation shows that the exponent is $\frac{1}{4}$ when the change of boundary condition is made between different types of $S$ matrices. While the exponent will depend on $\theta$ (scattering angle in the $S$ matrix) as a quadratic relation when the change is made between the same types of $S$ matrices of different $\theta$.

These two features are tested in three typical processes in the numerical calculation. After using suitable regulators for the zero-mode problem, the numerical results agree with the analytic calculation within error. Although irrelevant to the non-equilibrium dynamics, the fidelity calculation is used as a diagnostic tool and shows better agreement, providing more confidence about the analytic results. 

We proposed that the Loschmidt echo exponent in principle should be realizable in connecting two quantum wire with an X-ray type experiment. 



%%% Local Variables:
%%% TeX-master: "bCFT_paper"
%%% TeX-PDF-mode: t
%%% End:



\begin{acknowledgments}
\todo[inline]{Thomas Faulkner, Xueda Wen, Shinsei Ryu, Romain Vasseur}
    TZ is supported by the National Science Foundation under grant number NSF-DMR-1306011.
    This work made use of the Illinois Campus Cluster, a computing resource that is operated by the
    Illinois Campus Cluster Program (ICCP) in conjunction with the National Center for
    Supercomputing Applications (NCSA) and which is supported by funds from the University of
    Illinois at Urbana-Champaign.
\end{acknowledgments}

\appendix
\section{Corrections to the Free Energy}
% two corrections:
% 1. staircase geometry
% 2. contribution from the boundary: cite Cardy 
\label{app:F_correction}

\todo[inline]{cite main text figure for conformal transformation}
In the course of derivation the free energy subject to various boundary conditions, we use conformal transformation to convert the space time diagram with slits to a cylinder diagram, where the boundary state (and ground state energy calculation in App.~\ref{app:gnd_dn_lambda}) is applicable. However, free energy, especially its constant part, is not invariant under the conformal transformation, because the boundaries partially break the conformal symmetry. In this Appendix, we point out two corrections -- one from the outer boundary and the other from the inhomogeneous Schwartzian term -- to get the correct exponent of the fidelity and Loschmidt echo. 

% outer boundary correction
It is discussed in the pioneering work\cite{cardy_finite-size_1988} by Cardy and Peschel that boundary will contribute logarithmic term, which manifest itself as the geodesic curvature integral in the trace anomaly. Here we demonstrate this principle using the disk free energy example in Ref.~\cite{cardy_finite-size_1988}. Consider an annulus on flat space with inner radius $r_1$ and outer radius $r_2$. Its free energy is
\begin{equation}
F({\rm annulus}) = -  \frac{c}{6} \ln \frac{r_2}{r_1} 
\end{equation}
On the other hand the free energy of a disk of radius $r_2$ is
\begin{equation}
  F( {\rm disk} ) = - \frac{c}{6} \ln \frac{r_2}{a}
\end{equation}
where $a$ is the short distance regulator. The disk free energy is completely contributed by its outer boundary and so the one can interpret the annulus free energy as the additive contribution from its outer and inner surfaces (which is also consistent with the geodesic curvature integral calculation),
\begin{equation}
F( {\rm annulus} ) = -  \frac{c}{6} \ln \frac{r_2}{a} +  \frac{c}{6} \ln \frac{r_1}{a} = - \frac{c}{6} \ln \frac{r_2}{r_1}
\end{equation}
An annulus becomes a disk when its inner radius is of order $a$, and we can see that the contribution from the inner surface $\frac{c}{6} \ln \frac{r_1}{a}$ becomes negligible. 

\todo[inline]{citing figure}
We now scrutinize the conformation transformation from the slit diagram to the annulus. In the slit diagram, the regulators all have radius that is order of the short distance cut-off. They will have negligible contributions to the free energy. However, after doing the conformal transformation (passive), the outer surface will contribute $-\frac{c}{6} \ln \frac{r_2}{r_1}$ which should not be there. Using the parameters in the free energy calculation in other appendices, we have 
\begin{equation}
F_{\xi } = F_{z} + \frac{c}{6} \beta 
\end{equation}

% staircase geometry correction
The annulus is called staircase geometry Ref.~\onlinecite{cardy_finite-size_1988} due to its angular direction of time evolution. The traditional radial quantization however has radial direction to be the time direction. One can go through a series of conformal maps, one can show that that the Hamiltonian of the staircase and rectangle has a shift due to the Schwartzian
\begin{equation}
H_{z} = H_{\xi} - \frac{c}{24\pi} \beta 
\end{equation}
After the evolution for $2\pi$, the difference in the free energy is
\begin{equation}
F_w = F_{\xi} - \frac{c}{12} \beta 
\end{equation}
Gathering the two together, we obtain the missing $\frac{c}{12} \beta$ between the echo diagram and the Casimir energy, 
\begin{equation}
F_{w} = F_z + {\color{red} \frac{c}{12}\beta }
\end{equation}
\begin{figure}[h]
\centering
\includegraphics[width=\textwidth]{fig_uncon_rq.pdf}
\caption{Unconventional radial quantization on annulus(torus). Left panel: unlike the conventional radial quantization, we view the dashed line as a state and to time evolution in the radial direction. The blue circles are regarded as boundaries conditions rather than asymptotic states. Middle panel: $\xi = \ln w$ plane. Right panel: $z = \exp(i \frac{2\pi}{\beta} \xi)$ plane. We treat those dashed lines as asymptotic states and perform standard radial quantization here.}
\label{fig:uncon_rq}
\end{figure}

\todo[inline]{There are two series of conformal transformation, one is the main text, the other is the staircase geometry. Both use $w, z, \xi$, I will fix the notation after the main text figure is complete. Maybe detailed computation of the Schwartzian term is necessary, because of the sign. }

%%% Local Variables:
%%% TeX-master: "bCFT_paper"
%%% TeX-PDF-mode: t
%%% End:


\section{Scale Invariant Interface in Free Bosonic Lattice}
\label{app:interface_free_boson}
\todo[inline]{Mao: I suggest to relocate this part to an appendix, and maybe more details can be filled in, especially scale invariance condition imposed on the $\Sigma$. In the main text, just cite the relation between $\Sigma$ and $S$ and proceed}
In this appendix, we demonstrate how to realize scale invariant S-matrix in a lattice regularization. Following Sec.~\ref{sec_sub:free_boson_lattice}, we consider the following harmonic chain 
\begin{equation}
H = \frac{1}{2} \sum_i \left(\pi_i^2  + ( \phi_i - \phi_{i+1} )^2 \right) +  \frac{1}{2} \begin{pmatrix}  \phi_0, \phi_1 \end{pmatrix}
\Sigma
\begin{pmatrix}
\phi_0 \\
\phi_1 
\end{pmatrix}
\end{equation}
where the matrix $\Sigma$ parameterize two-site interaction between site 0 and 1. To find the scattering matrix, we set up the plane wave scattering problem across the interface with the following ansatz (the use of $(n-1)$ in $\phi_n^B$ simplifies the calculation)
\begin{equation}
\label{eq:ansatz}
\phi_n
= \left\lbrace
  \begin{aligned}
    \phi_n^A &= A_{-} e^{i \omega t  - inka}  + A_{+} e^{i \omega t  + inka}  & \quad  n \le 0 \\
    \phi_n^B &= B_{-} e^{i \omega t  - i(n-1)ka}  + B_{+} e^{i \omega t  + i(n-1)ka} & \quad n \ge 1 \\
  \end{aligned} \right. 
 \quad 
\end{equation}
where $a$ is the lattice constant. Upon solving the equation of motion far away from the interface, it is found that the system is gapless with the dispersion relation $\omega = \left|2\sin\frac{k}{2}\right|$. The interface will not break the criticality of the harmonic chain\cite{peschel_exact_2012}. After some algebra, the incoming and outgoing amplitudes are related via
\begin{widetext}
\begin{equation}
\label{eq:discrete_S}
\begin{pmatrix}
A_{+} \\
B_{-}\\
\end{pmatrix}
=-
\begin{bmatrix}
\Sigma_{11} +e^{ika} & \Sigma_{12}\\
\Sigma_{21} & \Sigma_{22} + e^{ika}
\end{bmatrix}^{-1}
\begin{bmatrix}
\Sigma_{11} + e^{-ika} & \Sigma_{12} \\
\Sigma_{21} & \Sigma_{22}  + e^{-ika}  \\
\end{bmatrix}
\begin{pmatrix}
A_{-}\\
B_{+}\\
\end{pmatrix}
\end{equation}
and the scattering matrix can be solved as
\begin{equation}
  S = \frac{-1}{ \det \Sigma  + e^{-ika} \text{tr} \Sigma   + e^{-2ika}}
\begin{bmatrix}
\det \Sigma+ \Sigma_{11} e^{-ika} + \Sigma_{22} e^{ika}+1  & -2i \sin ka \Sigma_{12}  \\
-2i \sin ka \Sigma_{21} &  \det \Sigma+ \Sigma_{11} e^{ika} + \Sigma_{22} e^{-ika}+1\\
\end{bmatrix}
\end{equation}
\end{widetext}
The S matrix elements are reflection and transmission coefficients. In order to have a scale invariant interface, they need to be $k-$independent. Therefore the necessary condition is that the ratio $|S_{11}/S_{21}|$ is independent of $k$, which implies 
\begin{equation}
\label{eq:Sigma_condition}
\det \Sigma = -1, \, \, \text{tr} \Sigma = 0
\end{equation}
Therefore, we have the scale invariant S-matrix as
\begin{equation}
S = \frac{1}{1 - e^{-2ika } } ( -2i \sin ka ) \Sigma
 = - e^{ika} \Sigma
\end{equation}
In the actual simulation, we evolved a large enough system for a long time, such that a field theoretic description can be applied. In this continuum limit where $a\rightarrow0$, the matrix $\Sigma$ can be parameterized as
\begin{equation}
\Sigma = -\lim_{a \rightarrow 0 } S = 
\begin{bmatrix}
\frac{\lambda^2- 1}{1 + \lambda^2} & \frac{-2\lambda }{1 + \lambda^2} \\
\frac{-2\lambda }{1 + \lambda^2} & \frac{1- \lambda^2}{1 + \lambda^2} \\
\end{bmatrix}
\end{equation}
where $\lambda\in\mathbb{R}$ is the parameter for $S_1(\theta)$, as emphasized in Sec.~\ref{sec:notation}. This simple relation between $\Sigma$ and the S-matrix provides a tool to check our analytic results. 

%%% Local Variables:
%%% TeX-master: "bCFT_paper"
%%% TeX-PDF-mode: t
%%% End:


\section{Numerical Computation of Bipartite Fidelity and Loschmidt Echo}
% basis transformation, Bogoliubov state, overlap(master theorem)
% Peschel_EE.pdf
\label{app:comp_fid_echo}

In this appendix, we provide technical details about the numerical calculation of the bipartite fidelity and Loschmidt echo.  Our strategy takes advantage of the symplectic structure of the bosonic Bogoliubov transformation and explicitly construct the "BCS" like ground state. With slight modification\cite{blaizot_quantum_1986}, one can work out its fermionic version and apply to quadratic fermion models for example in Ref.~\onlinecite{vasseur_universal_2014,stephan_local_2011}.

During the course of derivation in this and other appendices, we will repeatedly use the combinatorial identity called McMahon Master theorem
\begin{equation}
\label{eq:bosonic_McMahon}
\langle0|\exp\left\{\frac{1}{2}b_iX_{ij}b_j\right\}\exp\left\{\frac{1}{2}b^\dagger_iY_{ij}b^\dagger_j\right\}|0\rangle=\text{det}^{-\frac{1}{2}}(1-XY),
\end{equation}
for symmetric matrix $X$ and $Y$ and set of independent bosonic creation operators $b_i^{\dagger}$. One can prove it for (simultaneously) diagonalizable matrices and then claim its legitimacy for its combinatorial nature. 

\subsection{Boson Bogoliubov transformation}
\label{app_sub:boson_BdG}
We consider the following quadratic bosonic Hamiltonian
\begin{equation}
\begin{aligned}
\label{eq:quadratic_boson_H}
\hat{H} &= \frac{1}{2} (\vec{b}^{\dagger}, -\vec{b}) M 
\begin{pmatrix}
\vec{b}\\
\vec{b}^{\dagger} 
\end{pmatrix},  \qquad 
M = 
\begin{bmatrix}
A & -B^* \\
B & -A^* \\
\end{bmatrix},
\end{aligned}
\end{equation}
where ${\bf b}\equiv(b_{1},...,b_{n})^T$ is a vector of bosonic annihilation operators. The matrix $M$ consists of a $n\times n$ Hermitian block $A$ which plays the role of single particle Hamiltonian in the fermionic case and symmetric block $B$ of the pairing interaction. 

We want to do a Bogoliubov transformation, which uses a $2n \times 2n$ matrix $S$ to define a diagonal basis 
\begin{equation}\begin{aligned}
\label{eq:def_a}
({\bf a} , {\bf a}^\dagger)\equiv({\bf b} , {\bf b}^\dagger)S
\end{aligned}\end{equation}
of the Hamiltonian. The transformation is canonical, meaning that it preserves the commutation relation 
\begin{equation}
\begin{aligned}
\label{eq:preserve_commutator}
J&\equiv\begin{bmatrix}
0 & \mathbb{I}\\
-\mathbb{I} & 0
\end{bmatrix}
=[
\begin{pmatrix}
{\bf a} \\
{\bf a}^\dagger
\end{pmatrix},
\begin{pmatrix}
{\bf a} & {\bf a}^\dagger
\end{pmatrix}]
=S^T[
\begin{pmatrix}
{\bf b} \\
{\bf b}^\dagger
\end{pmatrix},
\begin{pmatrix}
{\bf b} & {\bf b}^\dagger
\end{pmatrix}]S \\
&=S^T JS,
\end{aligned}
\end{equation}
where we have used the compact notation of the sort $([\vec{b}, \vec{b}^{\dagger}])_{ij} =  [b_i, b_j^\dagger]$ to denote the commutator matrix. The appearance of $J$ makes the symplectic nature of the problem manifest and we find $S$ is in the symplectic group ${\rm Sp}( 2n, \mathbb{C} ) $\cite{blaizot_quantum_1986,fulton_representation_2004}. Furthermore, the requirement that $a^\dagger$ is a complex conjugation of $a$ leads to the block structure of $S$
\begin{equation}
\label{eq:block_S}
S=
\begin{bmatrix}
u & v^*\\
v & u^*
\end{bmatrix}.
\end{equation}
And the blocks are constrained by the symplectic property
\begin{eqnarray}
  u^\dagger u-v^\dagger v&=\mathbb{I},\label{eq:constraint_1}\\
  u^T u-v^T v&=0\label{eq:constraint_2}.  
\end{eqnarray}
With these conditions, the Hamiltonian in basis $a$ becomes (the use of $({\bf b}^\dagger , -{\bf b})$ rather than $({\bf b}^\dagger , {\bf b})$ can be appreciated in this step)
\begin{equation}
H = \frac{1}{2} ( a^{\dagger}, -a )  (S^{\top} M (S^{\top})^{-1} )
\begin{pmatrix}
a\\
a^{\dagger} 
\end{pmatrix}.
\end{equation}
Quiet unusually, the diagonalization is performed by the symplectic group element.

To proceed, we introduce the real basis 
\begin{equation}\begin{aligned}
\label{eq:real_basis}
\begin{pmatrix}
{\bf b}\\
{\bf b}^\dagger
\end{pmatrix}
=C\begin{pmatrix}
\phi\\
\pi
\end{pmatrix}
=\frac{1}{\sqrt{2}}\begin{bmatrix}
1 & i \\
1 & -i
\end{bmatrix}\begin{pmatrix}
\phi\\
\pi
\end{pmatrix},
\end{aligned}\end{equation}
in which the Hamiltonian reads
\begin{equation}\begin{aligned}
\hat{H}
&=\frac{1}{2}
\begin{pmatrix}
\phi & \pi
\end{pmatrix}
\begin{bmatrix}
\text{Re}(A-B^*) & -\text{Im}(A)+\text{Im}(B)\\
\text{Im}(A)+\text{Im}(B)& \text{Re}(A+B) \\
\end{bmatrix}
\begin{pmatrix}
\phi\\
\pi
\end{pmatrix} \\
&=\frac{1}{2}
\begin{pmatrix}
\phi & \pi
\end{pmatrix}
\mathcal{M}
\begin{pmatrix}
\phi\\
\pi
\end{pmatrix}.
\end{aligned}\end{equation}
It is not hard to check that $\mathcal{M}$ is real and symmetric. 

The general solution diagonalization problem is hard\cite{arnold_mathematical_2010}, however the positive definite $\mathcal{M}$ (and hence $M$) can be solved by Williamson's theorem\cite{arnold_mathematical_2010,xiao_theory_2009,pirandola_correlation_2009,gosson_symplectic_2006}, which states the existence, uniqueness (up to reordering of eigenvalues) and explicit construction of the matrix $\mathcal{S}\in {\rm Sp}(2n \mathbb{R}) $ such that 
\begin{equation}\begin{aligned}
\mathcal{M}=\mathcal{S}
\begin{bmatrix}
\, d \, & \\
 & \, d\, \\
\end{bmatrix}
\mathcal{S}^T,
\end{aligned}\end{equation}
where the diagonal matrix $d$ are positive eigenvalues of $iJ\mathcal{M}$. After some algebra, we have
\begin{equation}\begin{aligned}
\label{eq:diagonalization_M}
M=J(C^{-1})^T\mathcal{S}C^TJ^{-1}
\begin{bmatrix}
d\\
&-d
\end{bmatrix}
C\mathcal{S}^TC^{-1}.
\end{aligned}\end{equation}
One can show that,
\begin{equation*}\begin{aligned}
S\equiv C\mathcal{S}^TC^{-1}
\end{aligned}\end{equation*}
is the required symplectic matrix in the complex basis. 

We will not elaborate on Williamson's theorem and its proof (see proofs in Ref.~\onlinecite{xiao_theory_2009,pirandola_correlation_2009,gosson_symplectic_2006} and also a recent application in entanglement entropy context\cite{coser_contour_2017}). Instead we will show in App.~\ref{app_sub:harmonic_chain} that for the problem of harmonic chain we are interested in, the diagonalization can be easily done without using the general recipe in the Williamson theorem. 


\subsection{Groundstate in $b$ Basis}

Suppose we have obtained the required matrix $S$, the ground state will be the vacuum of the annihilation operators defined in Eq.~\eqref{eq:def_a} and in $b$ basis it satisfies
\begin{equation}\begin{aligned}
\label{eq:a_vacuum_condition}
(b_iu_{ij}+b_i^\dagger v_{ij})|0\rangle_{{\bf a}}=0.
\end{aligned}\end{equation}
If the matrix $u$ is invertible, then we can introduce a matrix $T=vu^{-1}$ to rewrite Eq.~\eqref{eq:a_vacuum_condition} as
\begin{equation}\begin{aligned}
(b_i+b_j^\dagger T_{ji})|0\rangle_{{\bf a}}=0. 
\end{aligned}\end{equation}
The constraint Eq.~\eqref{eq:constraint_2} on the blocks of $u$ and $v$ (followed by the symplectic constraint of $S$) implies that $T$ is a symmetric matrix. With the observation of 
\begin{equation}\begin{aligned}
\exp\left\{-\frac{1}{2}b_j^\dagger T_{jk}b_k^\dagger\right\}b_i\exp\left\{\frac{1}{2}b_j^\dagger T_{jk}b_k^\dagger\right\}=b_i+T_{ij}b^\dagger_j,
\end{aligned}\end{equation}
we solve the groundstate
\begin{equation}\begin{aligned}
\label{eq:boson_BCS_gnd}
|0\rangle_{{\bf a}}&=\text{det}^{\frac{1}{4}}(1-T^\dagger T)\exp\left\{-\frac{1}{2}b_j^\dagger T_{jk}b_k^\dagger\right\}|0\rangle_{\bf b},\\
\end{aligned}\end{equation}
where the normalization is given by the McMahon Master theorem Eq.~\ref{eq:bosonic_McMahon}. Apply constraint in Eq.~\eqref{eq:constraint_1}, it simplifies to the top left corner of the symplectic matrix
\begin{equation}
\text{det}^{\frac{1}{4}}(1-T^\dagger T) =|\text{det}(u)|^{-\frac{1}{2}}.
\end{equation}

Eq.~\eqref{eq:boson_BCS_gnd} takes a similar form as the superconducting ground state, with the pairing wavefunction $T_{ij}$ determined by the Bogoliubov transformation. In the next section, we will see that the normalization factor gives the fidelity and Loschmidt echo.

\subsection{Boson fidelity} 
\label{app_sub:boson_fidelity}

Fidelity is defined as the (squared) overlap of groundstates of two different bosonic Hamiltonians. 

We start with a quadratic bosonic Hamiltonian $\hat{H}_0$ in ${\bf b}$ basis, as in Eq.~\eqref{eq:quadratic_boson_H}. From the discussion in App.~\ref{app_sub:boson_BdG}, we are able to diagonalize it in ${\bf a}$ basis for positive definite $M$. At $t=0$, the Hamiltonian becomes $\hat{H}_1$, which is still written in ${\bf b}$ basis, but is diagonalized in a new basis ${\bf c}$. The corresponding Bogoliubov transformations read
\begin{equation}\begin{aligned}
\label{eq:two_BdG}
({\bf b} , {\bf b}^\dagger)S_0=({\bf a} , {\bf a}^\dagger),\quad
({\bf b} , {\bf b}^\dagger)S_1=({\bf c} , {\bf c}^\dagger),
\end{aligned}\end{equation}
and so
\begin{equation}\begin{aligned}
\label{eq:S0invS}
({\bf a} , {\bf a}^\dagger)=({\bf c} , {\bf c}^\dagger)\left(S_0^{-1}S_1\right)^{-1}.
\end{aligned}\end{equation}
One realizes that Eq.~\eqref{eq:S0invS} is another Bogoliubov transformation and so the corresponding matrix has the block structure
\begin{equation}
S_1^{-1}S_0=\begin{bmatrix}
u_1 & v_1^*\\
v_1 & u_1^*
\end{bmatrix}.
\end{equation}
Thus $|0\rangle_{\bf c}$ is related to the $|0\rangle_{{\bf a}}$ in the same way as in Eq.~\eqref{eq:boson_BCS_gnd}. Their overlap is therefore given by the normalization factor
\begin{equation}\begin{aligned}
|{}_{\bf a}\langle0|0\rangle_{\bf c}|^2=\frac{\Big|{}_{\bf a}\langle 0 | \exp( -\frac{1}{2} a_j^{\dagger} T^{jk} a_k^{\dagger} )|0   \rangle_{\bf a} \Big|^2}{|\text{det}(u_1)|} = \frac{1}{|\text{det}(u_1)|}.
\end{aligned}\end{equation}

\subsection{Boson Loschmidt echo}
\label{app_sub:boson_Loschmidt_echo}

The Loschmidt echo is defined as the (squared) overlap of the evolved state
\begin{equation}\begin{aligned}
|0\rangle_{{\bf a}(t)}\equiv e^{-i\hat{H}_1t}|0\rangle_{\bf a},
\end{aligned}\end{equation}
with $|0\rangle_{\bf a}$ the ground state of the Hamiltonian $\hat{H}_0$ before the quench. We introduce a dynamical basis
\begin{equation}\begin{aligned}
a_i(t)&=e^{-i\hat{H}_1t}a_ie^{i\hat{H}_1t},
\end{aligned}\end{equation}
which annihilate the evolved state at time $t$: $a_i(t)|0\rangle_{{\bf a}(t)}=0$. Upon using the diagonal basis $\hat{H}_1=\sum_iE_ic_i^\dagger c_i$, the Bogoliubov transformation at time $t$ can be represented as a chain of symplectic transformation
\begin{equation}\begin{aligned}
\label{eq:BdG_transformation_echo}
&({\bf a}(t),{\bf a}^\dagger(t))=e^{-iHt}
({\bf a} , {\bf a}^\dagger)e^{iHt} \\
=&({\bf a},{\bf a}^\dagger)S_0^{-1}S_1\text{diag}(e^{iEt},e^{-iEt})S_1^{-1}S_0.
\end{aligned}\end{equation}
It is evident that the evolved state $|0\rangle_{{\bf a}(t)}$ is related to the $|0\rangle_{{\bf a}}$ in the same way as in Eq.~\eqref{eq:boson_BCS_gnd}. The overlap, as we have seen in the fidelity case, is the normalization factor of the "BCS" ground state. It is related to the top left block of the Bogoliubov transformation in Eq.~\eqref{eq:BdG_transformation_echo},
\begin{equation}\begin{aligned}
\label{eq:echo_t}
\mathcal{L}(t)=|_{\bf a}\langle0|0\rangle_{{\bf a}(t)}|^2=|\text{det}(u_1^\dagger e^{iEt}u_1-v_1^\dagger e^{-iEt}v_1)|^{-1}.
\end{aligned}\end{equation}

\subsection{Harmonic chain} 
\label{app_sub:harmonic_chain}

In this subsection, we explicitly construct the matrix $\mathcal{S}$ for the case of harmonic chain introduced in Sec.~\ref{sec_sub:free_boson_lattice}. In the basis defined in Eq.~\eqref{eq:real_basis}, the Hamiltonian for 1D harmonic chain reads
\begin{equation}\begin{aligned}
\hat{H}
=\frac{1}{2}
\begin{pmatrix}
\phi & \pi
\end{pmatrix}
\mathcal{M}
\begin{pmatrix}
\phi\\
\pi
\end{pmatrix}
=\frac{1}{2}
\begin{pmatrix}
\phi & \pi
\end{pmatrix}
\begin{bmatrix}
\mathcal{V} \\
& {\mathbb{ I}}
\end{bmatrix}
\begin{pmatrix}
\phi\\
\pi
\end{pmatrix},
\end{aligned}\end{equation}
where $\mathcal{V}$ is real symmetric matrix that can be diagonalized as $\mathcal{V}=\mathcal{O}D^2\mathcal{O}^T$. The matrix $\mathcal{V}$ depends on the boundary condition, but positive definiteness is the only requirement here. 

The matrix $\mathcal{S}$ that diagonalize $\mathcal{M}$,
\begin{equation}
\begin{aligned}
\mathcal{M}=\mathcal{S}
\begin{bmatrix}
D \\ 
& D
\end{bmatrix}
\mathcal{S}^T
\end{aligned}
\end{equation}
is given by the following real symplectic matrix
\begin{equation}\begin{aligned}
\mathcal{S}\equiv
\begin{bmatrix}
\mathcal{O}D^{1/2} \\
& \mathcal{O}D^{-1/2}
\end{bmatrix}.
\end{aligned}\end{equation}


Thus the Hamiltonian is diagonalized as
\begin{equation}\begin{aligned}
M=S^{-1}
\begin{bmatrix}
D \\
& -D
\end{bmatrix}
S,
\end{aligned}\end{equation}
where the Bogoliubov transformation take the desired block form
\begin{equation}\begin{aligned}
S&=C\mathcal{S}C^{-1}
=\begin{bmatrix}
O(D^{1/2}+D^{-1/2}) & O(D^{1/2}-D^{-1/2}) \\
O(D^{1/2}-D^{-1/2}) & O(D^{1/2}+D^{-1/2}) 
\end{bmatrix}.
\end{aligned}\end{equation}


%%% Local Variables:
%%% TeX-master: "bCFT_paper"
%%% TeX-PDF-mode: t
%%% End:


\section{A Determinant Identity for the Boundary State Amplitude}
\label{app:pf_of_id}
% copy notes app. C
% flatex input: [pf_of_id.tex]

In this appendix, we provide more details for calculating the amplitude $Z_{ab}$ in Sec.{\bf\color{red}where}. We started to prove the following identity for a real symmetric matrix $M$
\begin{equation}
\label{1st id in app.pf_of_id}
\exp\Big\{- \vec{b}^{\dagger} M \vec{b}  \Big\} \exp \Big\{ \vec{b}^{\dagger} R \bar{\vec{b}}^{\dagger}  \Big\}  = \exp \Big\{ \vec{b}^{\dagger} e^{-M}  R \bar{\vec{b}}^{\dagger}  \Big\} \exp\Big\{- \vec{b}^{\dagger} M \vec{b}  \Big\} 
\end{equation}
where ${\bf b}$ and $\bar{\bf b}$ are vectors of bosonic operators. {\bf\color{red}As emphasized in the main text}, the matrix notation, for example, $\vec{b}^{\dagger} R \bar{\vec{b}}^{\dagger}$ should really mean $\sum_{ij}b^\dagger_iR_{ij}\bar{b}_j^\dagger$ where the dagger does not transpose the vector.  

To prove Eq.(\ref{1st id in app.pf_of_id}), we first consider the special case where $R={\bf I}$. We diagonalize $M = O^{T} \Lambda O $ and rotate the two sets of boson operators to the diagonal basis
\begin{equation}
  \vec{b}^{\dagger}  M \vec{b} = \vec{d}^{\dagger} \Lambda \vec{d}  \quad \vec{d} = O \vec{b} \quad \vec{\bar{d}}^\dagger = O^T \vec{\bar{b}}^\dagger
\end{equation}
where we understood $\vec{\bar{b}}^\dagger$ is a column vector and independent to $\vec{{b}}^\dagger$. Thus the whole expression can be written as
\begin{equation}
\begin{aligned}
  \exp&\Big\{- \vec{b}^{\dagger} M \vec{b}  \Big\} \exp \Big\{ \vec{b}^{\dagger} \bar{\vec{b}}^\dagger  \Big\}  =  
  \exp\Big\{- \vec{d}^{\dagger} \Lambda \vec{d}  \Big\} \exp \Big\{   \vec{d}^{\dagger} \bar{\vec{d}}^\dagger  \Big\} \\
& = \prod_i  \exp\Big\{- \lambda_i d_i^{\dagger} d_i  \Big\} \exp \Big\{  d_i^{\dagger} \bar{d}_i ^{\dagger}  \Big\}
\end{aligned}
\end{equation}
We recall for $ [X, Y] = sY $, 
\begin{equation}
  e^X e^{Y} = e^{\exp (s ) Y} e^{X}
\end{equation}
which is a solvable case of Baker-Campbell-Hausdorff formula. Upon taking $X = -\lambda_i d_i^{\dagger} d_i$, $Y = d_i^{\dagger} \bar{d}^{\dagger}_i$, we have
\begin{equation}
\label{lambda commutator}
[- \lambda_i d_i^{\dagger} d_i, d_i ^{\dagger} \bar{d}_i^{\dagger}] =  - \lambda_i  d_i ^{\dagger} \bar{d}_i^{\dagger} 
\end{equation}
and so $s = - \lambda_i$ for each $\lambda_i$. This enable us to commute those exponentials
\begin{equation}
\begin{aligned}
 \exp\Big\{- \vec{b}^{\dagger} M \vec{b}  \Big\} \exp \Big\{  \vec{b}^{\dagger} \bar{\vec{b}}^\dagger  \Big\}   &= \prod_i \exp \Big\{ e^{- \lambda_i }  d^{\dagger}_i \bar{d}^{\dagger}_i  \Big\}  \exp \Big\{-\lambda_i  d^{\dagger}_i d_i  \Big\} \\
 & = \exp \Big\{ \vec{b}^{\dagger} e^{-M}  \bar{\vec{b}}^\dagger  \Big\} \exp\Big\{- \vec{b}^{\dagger} M \vec{b}  \Big\} 
\end{aligned}
\end{equation}
For the general case where $R \ne I$, we take $\vec{\bar{d}}^* = O^T R \vec{\bar{b}}^*$. This will not change the commutation relation of ${\bf d}$, and the role of $\bar{b}$ is decorative in Eq.(\ref{lambda commutator}). Hence the rest of the proof follows the same way. \hfill$\blacksquare$

{\bf \color{red}Mao: use the format "Eq.\verb|~\eqref|{}" for equation, "Sec.\verb|~\ref|{}" for section. }

A direct consequence of Eq.(\ref{1st id in app.pf_of_id}) is the following
\begin{equation}
\label{2nd id in app.pf_of_id}
Z_{ab} = \langle 0 | \exp\Big\{ \vec{b} R_a^* \vec{\bar{b}}\Big\} \exp\Big\{ - \vec{b}^{\dagger} M  \vec{b} \Big\}   \exp\Big\{  \vec{b}^{\dagger} R_b  \vec{\bar{b}}^{\dagger}\Big\}  |0  \rangle  = \frac{1}{\det( 1- R_a^{\dagger} e^{-M} R_b )} 
\end{equation}
where $|0\rangle$ is the vacuum for ${\bf b}$ and $\bar{\bf b}$. Using the identity we have just proved, 
\begin{equation}
Z_{ab} =   \langle 0 | \exp\Big\{ \vec{b} R_a^* \vec{\bar{b}}\Big\}  \exp \Big\{ \vec{b}^{\dagger} e^{-M}  R_b \bar{\vec{b}}^{\dagger}  \Big\}  |0 \rangle 
\end{equation}
A direct application of the MacMahon master theorem
\begin{equation}
  \langle 0 | \exp \Big\{ \vec{b}_1 X \vec{b}_2 \Big\}  \exp \Big\{ \vec{b}^{\dagger}_1 Y \vec{b}^{\dagger}_2 \Big\}|0  \rangle 
 = \frac{1}{\det(1 - X^T Y )}
\end{equation}
proves Eq.(\ref{2nd id in app.pf_of_id}). \hfill$\blacksquare$


%%% Local Variables:
%%% TeX-master: "bCFT_paper"
%%% TeX-PDF-mode: t
%%% End:


\section{Alternative Approach to ${\rm DN} \rightarrow \lambda$ Amplitude}
\label{app:gnd_dn_lambda}
% copy notes app. A
% flatex input: [gnd_dn_lambda.tex]

\begin{figure}[h]
\centering
\includegraphics{fig_gnd_dn_lambda.pdf}
\caption{Partition function of Hamiltonian with DN and $\lambda$ boundary conditions. The width of the strip is $\pi$ due to folding. We unfold the cylinder and the new stripe have $N$ and $D$ boundary conditions on the left and right plus a $\lambda$ junction in the middle. }
\label{Fig in gnd_dn_lambda}
\end{figure}

In this appendix, we would like to calculate the amplitude for the setup shown in Fig.~\ref{Fig in gnd_dn_lambda}. In particular, the unfolded configuration has D/N boundary conditions at $x = \pm \frac{L}{2}$ and linking boundary condition at $x = 0$. The general solutions can be written as
\begin{equation}
\label{Normalized f in gnd_dn_lambda}
f(k, x) = 
\left\lbrace
\begin{aligned}
  A_1 e^{i kt} \cos(kx +\frac{1}{2}kL ) &  \quad x < 0  \\
  A_2 e^{ikt}  \sin(kx - \frac{1}{2}kL ) & \quad x > 0   \\
\end{aligned} \right. 
\end{equation}
As demonstrated in the {\bf\color{red}main text}, if we denote $f(k,x<0)\equiv\phi_1$ and $f(k,x>0)\equiv\phi_2$, the boundary condition at the junction reads
\begin{eqnarray}\begin{aligned}
\frac{\partial_x \phi_1}{ \partial_t \phi_1} = \lambda^2 \frac{\partial_x \phi_2}{ \partial_t \phi_2} = \tan^2 \theta\frac{\partial_x \phi_2}{ \partial_t \phi_2} \quad \theta \in [0,\frac{\pi}{2} ]  
\end{aligned}\end{eqnarray}
which implies
\begin{equation}
\label{Momentum in gnd_dn_lambda}
k = \frac{2\pi}{L}( n \pm \frac{\theta}{\pi} )  \quad n\in{\bf Z}
\end{equation}
It is evident that the momentum $k$ is shifted from integer multiple of $2\pi/L$ due to the $\lambda$ boundary condition in the middle. 
\begin{comment}
Thus, the normalized eigenfunctions are
\begin{equation}
f_n(x) = \sqrt{\frac{2}{L}}
\left\lbrace
\begin{aligned}
  \cos(kx +\frac{1}{2}kL ) &  \quad x < 0  \\
  \pm \sin(kx - \frac{1}{2}kL ) & \quad x > 0   \\
\end{aligned} \right. 
\qquad 
k = \frac{2\pi}{L}( n \pm  \frac{\theta}{\pi} )  \quad n \in \mathbb{Z} 
\end{equation}
Expand the field $\phi = \sum_n \phi_n f_n(x) $, the action and Hamiltonian becomes
\begin{equation}
  S = \frac{g}{2} \int dt \, \sum_{n \in \mathbb{Z} }\left(  \dot{\phi}^2_n + k^2 \phi_n^2 \right) \implies\quad   g \dot{\phi}_n  = \pi_n \quad \implies H =  
\frac{1}{2g}\sum_{n \in \mathbb{Z} } \pi_n^2 + ( kg )^2  \phi_n^2 
\end{equation}
\end{comment}

It is clear that the normalized eigenfunctions in Eq.~\eqref{Normalized f in gnd_dn_lambda} serves as orthonormal basis in mode expansion. Following the same procedure in \onlinecite{di_francesco_conformal_1997}, we have the Hamiltonian as
\begin{equation}
\label{H in gnd_dn_lambda}
H = \frac{1}{2} \sum_{n \in \mathbb{Z} } |k|  (a^{\dagger}_n a_n + \frac{1}{2} )
\end{equation}
where the momentum $k$ is defined in Eq.~\eqref{Momentum in gnd_dn_lambda}, and the creation and annihilation operators defined as usual
\begin{equation}
\begin{aligned}
a_n = \frac{1}{\sqrt{2}} ( \sqrt{ |k|g} \phi_n + \frac{i }{\sqrt{|k|g} }\pi_n  ) \\
a^{\dagger}_n = \frac{1}{\sqrt{2}} ( \sqrt{ |k|g} \phi_n - \frac{i }{\sqrt{|k|g} }\pi_n  ) \\
\end{aligned}
\end{equation}

The Casimir energy is the vacuum energy brought by the finite size of the setup. From Eq.~\eqref{H in gnd_dn_lambda} and define $x\equiv\theta/\pi$ in Eq.~\eqref{Momentum in gnd_dn_lambda}, we have
\begin{equation}
\begin{aligned}
E_c &= \frac{1}{4} \sum_{n \in \mathbb{Z}} | k| = \frac{\pi}{2L} ( \sum_{n \in \mathbb{Z}}  | n + x | + \sum_{n \in \mathbb{Z}}  | n - x |  ) \\  
&= \frac{\pi}{L} \Big[\sum_{n \ge 0 } ( n + x )^{-s} + \sum_{n \ge 0 }  ( n - x)^{-s}  +   x^{-s} \Big]\Big|_{s = -1} \\
&= \frac{\pi}{L} \left[ \zeta_{\rm H}( -1, x ) + \zeta_{\rm H}( -1, x ) +  x \right] \\
&= \frac{1}{2} ( - x^2 + x - \frac{1}{6})
\end{aligned}
\end{equation}
where we used the fact that the unfolded geometry has length $L=2\pi$. Thus the free energy reads
\begin{equation}
F = \beta E_c = - \frac{\beta}{2} B_2( x) 
\end{equation}
which agrees with the boundary state calculation in Eq.{\bf\color{red}where}



\section{General Boundary State Amplitude}
\label{app:lambda_12}
% copy notes app. B

\begin{figure}[h]
\centering
\includegraphics{fig_lambda_1_lambda_2.pdf}
\caption{Partition function between two boundary states of $S_i( \theta_1)$ and that of $S_j( \theta_2 )$}
\label{fig:fig_lambda_1_lambda_2}
\end{figure}

In this appendix, we calculate the amplitude between general boundary states (\cf  Eq.~\eqref{eq:bd_state})
\begin{equation}
\exp\Big\{  \vec{b}^{\dagger} S_a( \theta )   \vec{\bar{b}}^{\dagger}\Big\}  |0  \rangle ,
\end{equation}
where we have used the vector notations for the set of bosons. The $S$ matrices therein can be any one of the two types in Eq.~\eqref{eq:S1_S2}. 

Let the cylinder has height $2\pi$ and width $\beta \gg 1 $, the partition function between the two boundary states is
\begin{equation}
Z_{ab} =  \langle a | e^{ - 2\pi H } |b \rangle ,
\end{equation}
where the Hamiltonian is
\begin{equation}
H = \frac{2\pi}{\beta} (L_0 + \bar{L}_0) =  \frac{4\pi}{\beta}  L_0 = \frac{4\pi}{\beta}\sum_{n > 0 }  n b_n^{\dagger} b_n. 
\end{equation}
We have used the condition $L_n = \bar{L}_n$ valid for the boundary state. The partition function then becomes
\begin{eqnarray}\begin{aligned}
Z_{ab} &= \langle 0 | e^{ \vec{b} S_a^* \vec{\bar{b}} } e^{ - \vec{b}^{\dagger} \mathbb{I}_2 \otimes M  \vec{b} } e^{  \vec{b}^{\dagger} S_b  \vec{\bar{b}}^{\dagger} }  |0  \rangle  \\
&= \frac{1}{\det ( 1 - S^{\dagger} _a  e^{- \mathbb{I}_2 \otimes M} S_b) } ,
%Z_{ab} &= \langle 0 | \exp\Big\{ \vec{b} S_a^* \vec{\bar{b}}\Big\} \exp\Big\{ - \vec{b}^{\dagger} \mathbb{I}_2 \otimes M  \vec{b} \Big\}   \exp\Big\{  \vec{b}^{\dagger} S_b  \vec{\bar{b}}^{\dagger}\Big\}  |0  \rangle  \\
%&= \frac{1}{\det ( 1 - S^{\dagger} _a  e^{- \mathbb{I}_2 \otimes M} S_b) }
\end{aligned}\end{eqnarray}
where
\begin{equation}
M =  \frac{4\pi^2}{\beta} \text{diag}( 1, 2, \cdots ), \quad  \mathbb{I}_2 = \text{diag}( 1, 1) .
\end{equation}
Using the fact that $|\det( S_a S_b^{\dagger})|  = 1$, free energy becomes
\begin{equation}
F = - \ln |Z_{ab}| = \ln |\det ( S_a S_b^{\dagger} - e^{- \mathbb{I}_2 \otimes M} )| .
\end{equation}
There are two cases to be considered, and we only take out the leading order term in $\beta$. 
\begin{itemize}
\item {\it case 1: }$S_1( \theta_1 ) \rightarrow S_2 ( \theta_2 ) $, the free energy is
\begin{equation}
\begin{aligned}
F & = \ln |\det ( S_1( \theta_1 )  S_2^{\dagger}( \theta_2 )  - e^{- \mathbb{I}_2 \otimes M} )| \\
  & = \ln \left| \det
\begin{bmatrix}
-\cos 2 \Delta \theta \mathbb{I} - e^{-M}   & -\sin 2 \Delta \theta \mathbb{I}\\
- \sin 2\Delta \theta \mathbb{I}  &   \cos 2 \Delta \theta \mathbb{I} - e^{-M} \\ 
\end{bmatrix} \right| \\
& = \sum_i \ln [ 1 -  e^{- 2 \lambda_i }  ] \\
& = \frac{\beta}{4\pi^2} \int_0^{\infty} dx \ln [ 1 - e^{-2x} ]  = - \frac{1}{48 }\beta .
\end{aligned}
\end{equation}
\item {\it case 2:} $S_i( \theta_1 ) \rightarrow S_i( \theta_2 )$, where $i = 1 $ or $ 2$, 
\begin{equation}
\begin{aligned}
F & = \ln \det 
\begin{bmatrix}
\cos 2 \Delta \theta \mathbb{I} - e^{-M}   & \sin 2 \Delta \theta \mathbb{I}\\
- \sin 2\Delta \theta \mathbb{I}  &   \cos 2 \Delta \theta \mathbb{I} - e^{-M} \\ 
\end{bmatrix} \\
& = \sum_i \ln [ 1 - 2 \cos 2 \Delta \theta e^{- \lambda_i } + e^{- 2 \lambda_i }  ] \\
& = \frac{\beta}{4\pi^2} \int_0^{\infty} dx \ln [ 1 - 2 \cos 2 \Delta \theta e^{-x} + e^{-2x} ] ,
\end{aligned}
\end{equation}
where $\Delta \theta = \theta_2 - \theta_1$. This integral is an even function of $\Delta \theta$ and the $\Delta \theta > 0$ case reduce to the polylog and Bernoulli polynomial
\begin{equation}
\begin{aligned}
  F &= \frac{\beta}{4\pi^2} \left[ - \text{Li}_2 ( e^{2i |\Delta \theta|} ) - \text{Li}_2 ( e^{- 2i |\Delta \theta|} ) \right] \\
  & = \frac{\beta}{4\pi^2}  \left[ - 2\pi^2 B_2 (x) \right] \\
  &= - \frac{\beta}{2} B_2( |x| )  = \frac{\beta}{2} (| x| - x^2 - \frac{2}{6} ),
\end{aligned}
\end{equation}
where $x = \frac{\Delta \theta}{ \pi}$. 
\end{itemize}




%%% Local Variables:
%%% TeX-master: "bCFT_paper"
%%% TeX-PDF-mode: t
%%% End:


\section{Winding Modes of Compact Bosons}
% free boson + compact
\label{app:compact_diff_boson}
\todo[inline]{Remember to mention in the main text, winding sectors have no contribution}
In this appendix, we address the issue of winding modes of compact bosons. Up to now we have exclusively worked with the oscillator modes of free bosons. 
%We have introduced the general formulation of bosonic conformal interface in Sec.~\ref{sec_sub:general_formulation}, and calculated various partition functions from the boundary state approach in App.~\ref{app:gnd_dn_lambda}-\ref{app:lambda_12}. We have not talk about the winding mode for compact bosons. 
Here, we shall show that winding modes for compactified bosons will have no contribution to the fidelity or Loschmidt echo, at least in the leading order. Therefore, the result presented in the main text are ready applied to the case where two compactified bosons of different radii are connected by a conformal interface. \todo[inline]{say intuition why not matter, mention application to Laughlin, etc.}

We recall the gluing condition in Eq.~\eqref{eq:def_M} is a constraint from conformal invariance\cite{blumenhagen_introduction_2009,bachas_permeable_2002}, which provides no constraint for the winding modes. Indeed, the winding mode is a global property of the boson which is insensitive to the local variations. Thus one needs a more general treatment for the compactified bosons. This is summarized for the well known cases of Dirichlet and Neumann boundary conditions in App.~\ref{app_sub:compact_DN_boundary} and generalized to gluing condition in App.~\ref{app_sub:compact_gluing_boundary}. We demonstrate that winding mode does not contribute to the free energy through boundary state calculation in App.~\ref{app_sub:winding_contribution}. Throughout this appendix, the term ``conformal boundary state'' refers to its winding sector exclusively.

%\subsection{Free Boson with Different Compactification Radii}

\subsection{Winding Mode with Dirichlet and Neumann Boundary Conditions}
\label{app_sub:compact_DN_boundary}

We consider the following boson configuration
\begin{eqnarray}\begin{aligned}
\phi(x+L,t)=\phi(x,t)+2\pi mR
\end{aligned}\end{eqnarray}
where $m\in\mathbb{Z}$ is the winding number of the field. Since the field is compactified on a circle, its momentum $\pi_0$ takes discrete eigenvalues $\pi_0\equiv\frac{n}{R}$ with $n\in\mathbb{Z}$. Thus the mode expansion reads
\begin{eqnarray}\begin{aligned}
\label{eq:mode_expand_compact}
\phi(x,t)&=\phi_0-\frac{2\pi Rm}{L}x+\frac{\pi_0}{gL}t+\frac{i}{\sqrt{4\pi g}}\sum_{n\neq0}\frac{1}{n}(a_ne^{2\pi i n(x-t)/L}+\bar{a}_{n}e^{2\pi i n(x+t)/L})\\ 
&={\phi}_0-\frac{2\pi Rm}{L}x+\frac{n}{gRL}t+\frac{i}{\sqrt{4\pi g}}\sum_{n\neq0}\frac{1}{n}(a_ne^{2\pi i n(x-t)/L}+\bar{a}_{n}e^{2\pi i n(x+t)/L})
\end{aligned}\end{eqnarray}
Further, the holomorphic coordinates are introduced as
\begin{eqnarray}\begin{aligned}
z=\exp\left(2\pi i\frac{t-x}{L}\right)\qquad\bar{z}=\exp\left(2\pi i\frac{t+x}{L}\right)
\end{aligned}\end{eqnarray}
which leads to 
\todo[inline]{Take care of the format for long equation later, for two columns}
\begin{eqnarray}\begin{aligned}
\label{eq:phi_in_z}
\phi(z,\bar{z})=\phi_0-i\left(\frac{n}{4\pi gR}+\frac{mR}{2}\right)\ln z-i\left(\frac{n}{4\pi gR}-\frac{mR}{2}\right)\ln \bar{z}+\frac{i}{\sqrt{4\pi g}}\sum_{n\neq0}\frac{1}{n}(a_nz^{n}+\bar{a}_{n}\bar{z}^{-n})
\end{aligned}\end{eqnarray}
The key observation is that if one defines
\begin{eqnarray}\begin{aligned}
a_0=\sqrt{4\pi g}\left(\frac{n}{4\pi gR}+\frac{mR}{2}\right)\qquad\bar{a}_0=\sqrt{4\pi g}\left(\frac{n}{4\pi gR}-\frac{mR}{2}\right)\frac{mR}{2}
\end{aligned}\end{eqnarray}
then the holomorphic and anti-holomorphic derivatives can be written compactly as
\begin{eqnarray}\begin{aligned}
i\partial\phi(z)=\frac{1}{\sqrt{4\pi g}}\sum_{n\in\mathbb{Z}}a_{n}z^{-n-1}\qquad i\bar{\partial}\phi(\bar{z})=\frac{1}{\sqrt{4\pi g}}\sum_{n\in\mathbb{Z}}\bar{a}_{n}\bar{z}^{-n-1}
\end{aligned}\end{eqnarray}
\todo[inline]{I will think about how to say this part}
{\bf\color{red}
The Virasoro generators takes the same form as before
\begin{eqnarray}\begin{aligned}
L_n=\frac{1}{2}\sum_{m\in\mathbb{Z}}:a_{n-m}a_m:\qquad n\in\mathbb{Z}
\end{aligned}\end{eqnarray}
In particular for the conformal boundary state $|B\rangle$, the condition
\begin{eqnarray}\begin{aligned}
(L_n-\bar{L}_{-n})|B\rangle=0
\end{aligned}\end{eqnarray}
reduces to 
\begin{eqnarray}\begin{aligned}
\label{eq:DN_a}
\text{Dirichlet}&:a_0-\bar{a}_0=0\rightarrow m=0 \\
\text{Neumann}&:a_0+\bar{a}_0=0\rightarrow n=0
\end{aligned}\end{eqnarray}
}

The conformal boundary state for Dirichlet and Neumann are respectively
\begin{eqnarray}\begin{aligned}
\label{eq:DN_state}
|{\rm D} \rangle  &= g_{\,\!_\text{D}}  \prod_{n > 0 } \exp(\frac{1}{k} a_k^{\dagger} \bar{a}_{-k}^{\dagger} )| \phi_0 \rangle \quad | \phi_0 \rangle = \sum_{n \in \mathbb{Z}} e^{-i \frac{n}{R}\phi_0 } | n, m = 0 \rangle \\
| {\rm N} \rangle &= g_{\,\!_\text{N}} \prod_{n > 0 } \exp(\frac{1}{k} a_k^{\dagger} \bar{a}_{-k}^{\dagger} )| \bar{\phi}_0 \rangle \quad | \bar{\phi}_0 \rangle = \sum_{n \in \mathbb{Z}} e^{i m \bar{\phi}_0 } | n = 0, m \rangle 
\end{aligned}\end{eqnarray}
with the desired boundary conditions
\begin{eqnarray}\begin{aligned}
\phi(x,t=0)|{\rm D} \rangle=\phi_0|{\rm D} \rangle\qquad \partial_t \phi( x, t = 0 ) | {\rm N } \rangle = 0
\end{aligned}\end{eqnarray}

\subsection{Winding mode with Gluing Condition}
\label{app_sub:compact_gluing_boundary}
We recall that the conformal boundary condition can be written as
\begin{eqnarray}\begin{aligned}
\label{eq:def_S_in_app}
\begin{pmatrix}
\partial_+\phi^1\\
\partial_-\phi^2
\end{pmatrix}
=S(\theta)
\begin{pmatrix}
\partial_-\phi^1\\
\partial_+\phi^2
\end{pmatrix}
\end{aligned}\end{eqnarray}
where the scattering matrices are given in Eq.~\eqref{eq:S1_S2}. We shall specialize to the case $S=S_1$ first. The folding sends $\phi^2(x)\rightarrow\phi^2(-x)$ for $x<0$, and the $\partial_x\phi^2$ term changes sign, we therefore have 
\begin{eqnarray}\begin{aligned}
\label{eq:def_S_in_app_2}
\begin{pmatrix}
\partial_+\phi^1\\
\partial_+\phi^2
\end{pmatrix}
=S_1(\theta)
\begin{pmatrix}
\partial_-\phi^1\\
\partial_-\phi^2
\end{pmatrix}
\end{aligned}\end{eqnarray}
Upon mode expanding $\phi^{1,2}$ in the holomorphic coordinates as in Eq.~\eqref{eq:phi_in_z}, we have
\todo[inline]{Check this one}
\begin{eqnarray}\begin{aligned}
\label{eq:def_S_in_app_2}
\sum_{n\in\mathbb{Z}}\begin{pmatrix}
\bar{a}_n^1\bar{z}^{-n}\\
\bar{a}_n^2\bar{z}^{-n}
\end{pmatrix}
=S_1(\theta)
\sum_{n\in\mathbb{Z}}
\begin{pmatrix}
a_n^1{z}^{-n}\\
a_n^2{z}^{-n}
\end{pmatrix}
\end{aligned}\end{eqnarray}
{\bf\color{red}The boundary condition is imposed at $t=0$, where $\bar{z}=z^{-1}$}, therefore we have
\begin{eqnarray}\begin{aligned}
a^i_n-(S^{-1}_{ij})\bar{a}^j_{-n}=0
\end{aligned}\end{eqnarray}
This is clearly a generalization of Dirichlet and Neumann boundary conditions in Eq.~\eqref{eq:DN_a}. We therefore expect that the conformal boundary state is a generalization of Eq.~\eqref{eq:DN_state}, subject to the more general constraint 
\todo[inline]{check this one in the note}
\begin{eqnarray}\begin{aligned}
\label{eq:S_1_constraint}
\tan\theta=\frac{n_2R_1}{n_1R_2}=-\frac{m_1R_1}{m_2R_2}
\end{aligned}\end{eqnarray}
Thus we conclude that the conformal boundary state reads
\begin{eqnarray}\begin{aligned}
g_{S_1}\sum_{S_1}e^{in_1\phi_0-im_1\bar{\phi}_0}|n_1,m_1\rangle|n_2,m_2\rangle
\end{aligned}\end{eqnarray}
where $\sum_{S_1}$ is the summation consistent with the constraint in Eq.~\eqref{eq:S_1_constraint}. The g-factor can only be determined by Cardy condition\cite{oshikawa_boundary_2010}. Since it is not important for what follows, we shall not repeat the calculation here. For the case of $S=S_2$, it is evident that the same calculation can be carried out, and one will arrive at the similar boundary state.

\subsection{Winding Mode Contribution to the Partition Function}
\label{app_sub:winding_contribution}
We are ready to demonstrate that the winding mode will have no contribution to the leading order of $\beta$. We shall reconsider the partition function as shown in Fig.~\ref{fig:fig_lambda_1_lambda_2}
\begin{eqnarray}\begin{aligned}
Z=\langle\lambda_2|e^{-\pi H}|\lambda_1\rangle\qquad H=\frac{2\pi}{\beta}(L_0+\bar{L}_0)
\end{aligned}\end{eqnarray}
For the winding mode contribution, we simply replace $L_0$($\bar{L}_0$) with $a_0$($\bar{a}_0$). For the case of same boundary state $\lambda_1=\lambda_2$, we have the winding mode contribution as
\begin{eqnarray}\begin{aligned}
\label{eq:Z0}
Z_0 = \sum_{S_1 } g_{\,\!_{\lambda_1} }g_{\,\!_{\lambda_2} } \exp\Big\{- \frac{4\pi}{\beta} 2 \pi g ( \frac{n_1}{ 4 \pi R_1 g} + \frac{m_1 R_1 }{ 2} )^2 \Big\}
\end{aligned}\end{eqnarray}
In the $\beta\rightarrow\infty$ limit, Eq.~\eqref{eq:Z0} can be approximated by a simple two dimensional integral
\begin{eqnarray}\begin{aligned}
Z_0\approx g_{\,\!_{\lambda_1} }g_{\,\!_{\lambda_2} } (\sqrt{\beta})^2\int dxdy\exp\Big\{-8 \pi^2 g ( \frac{x}{ 4 \pi R_1 g} + \frac{y R_1 }{ 2} )^2 \Big\}\propto\beta
\end{aligned}\end{eqnarray}
Thus the winding mode will contribute at most a $\ln\beta$ term to the free energy. Compared to the result in App.~\ref{app:gnd_dn_lambda}-\ref{app:lambda_12}, we conclude that the winding mode contribution is negligible in the large $\beta$ limit. 

For the case $\lambda_1=\lambda_2$, the partition function will acquire non-zero contribution from the least common divisor of $(n_1,m_1)$ and $(n_2,m_2)$. The same conclusion still holds in the limit $\beta\rightarrow\infty$. 


\bibliographystyle{plain}
\bibliography{bCFT}

\end{document}

%%% Local Variables: 
%%% TeX-PDF-mode: t
%%% End: 
