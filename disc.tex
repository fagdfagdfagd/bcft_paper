
% relation between fidelity and echo

In the computations we have done, the results of fidelity can be converted to that of Loschmidt echo by the replacement recipe $ \ln L \rightarrow 2 \ln \tau$. The numerical factor $2$ comes from the fact that the Loschmidt echo has two slit tips. Other than that, we see that they probe the same finite size effect of the free energy associated with the new interfaces. Our purpose of computing the (simpler) fidelity is diagnostic and so in the following we will mainly discuss the echo properties. 

% analytic results type 1 -> type 2, c/4; 
In the Sec.~\ref{sec_sub:analy_eval}, we have presented the analytic results for the general process $S_i( \theta_1 ) \rightarrow S_j( \theta_2 )$ {\iffalse {\color{red}described in Eq.~\eqref{eq:S_i_S_j}}\fi} (assuming the far end boundary condition $c$ is the same as prior-quench interface $a$). 

We find that if the conformal interfaces are of different types, i.e. $i \ne j$, the (long time) free energy is always $\frac{1}{4} \ln t$, regardless of the theta angles. The two types of conformal interface do not talk to each other because they impose on different fields. If we treat $S_1$ as a combination of Dirichlet and Neumann boundary conditions imposed on the rotated $\phi$ fields as in Eq.~\eqref{eq:rotation_a_basis}, then $S_2$ imposes one of them on the duel field of $\phi$. In the derivation of the $M$ matrix, these two correspond to the parts of the Lorentz group and can not be connected even by taking singular value of $\lambda$. It is then reasonable to find a universal echo between them. The special value of ${\rm DD} \rightarrow P$ also agrees with the existing general CFT result of completely transparent interface\cite{stephan_logarithmic_2013,stephan_local_2011,vasseur_universal_2014,vasseur_crossover_2013,kennes_universal_2014}. 

% x dependence, DN -> ND agrees with existing result;

For the more interesting case where the boundary condition are of the same type, we have verified the quadratic angle dependence numerically for ${\rm DN} \rightarrow \lambda$ {\iffalse \color{red}the process in Eq.~\eqref{eq:DNDN}.\fi}. We can first understand the values of several special points on this curve. 
\begin{itemize}
\item $\theta = 0$: This is where the boundary condition does change before and after the quench, so the Loschmidt echo stays at $1$ and hence the exponent is $0$.
\item $\theta = \frac{\pi}{2}$: This is the process ${\rm DN}\rightarrow {\rm  ND}$. The chain is still disconnected after the change of the boundary condition. We can thus view the problem of changing the boundary conditions for two independent changes in Fig.~\ref{fig:echo}, one from D to N and the other from N to D. The Loschmidt echo can then be viewed as the product of the boundary two point correlation functions of the associated bcc operators $\phi_{\rm DN}$ and $\phi_{\rm ND}$, whose dimension is $\Delta = \frac{1}{16}$. From this we can get the exponent to be $\frac{1}{2}$
\begin{equation}
\begin{aligned}
  \quad \mathcal{L}(\tau) &\sim |\langle \phi_{{\rm D N}}(0) \phi_{{\rm ND}}(\tau)   \rangle |^2 
|\langle \phi_{{\rm ND}}(0)  \phi_{{\rm DN}}(\tau)   \rangle| ^2\\
& \sim \frac{1}{|\tau|^{8\Delta}} = \frac{1}{|\tau|^{\frac{1}{2}}},
\end{aligned}
\end{equation}
which agrees with our results. 
\item $\theta = \frac{\pi}{4}$: This is the process ${\rm DN}\rightarrow {\rm P}$. The exponent $\frac{3}{8}$ agrees with the Ref.~\onlinecite{kennes_universal_2014,stephan_logarithmic_2013}, where the difference with the exponent $\frac{1}{4}$ of ${\rm DD} \rightarrow {\rm P}$ is interpreted as twice the dimension of bcc operator dimension ($\frac{1}{16}$) that transforms D to N.
\end{itemize}
In general, the result gives the full spectrum of operator dimensions in the ${\rm DN}$ to $\lambda $ transition. In the bosonic theory we consider, the primary fields are vertex operators, which is exponential of the field $\exp( \nu \phi )$. Depending on the convention, the dimension of the vertex operator is a numerical constant times $\frac{\nu^2}{8\pi}$. So if $\nu$ linearly depend on $\theta$ (or $x$), then we will end up with a quadratic relation whose expression can already be fixed by the three special points we work out above. In a rational boson theory, the number of primary fields is finite. It requires further exploration to identify these bcc operators with the existing primary fields and their physical significance. 

On the other hand, in our lattice boson model, $\theta$ parameterizes the bond interaction between the boundary sites of the chains. Consequently it characterizes the strength of the local perturbation to the Hamiltonian: smaller $\theta$ means the $\Sigma$ matrix is closer to the original, thus a smaller perturbation and vise versa. Therefore, we expect that a larger perturbation will result in a faster decay of the Loschmidt echo, which is reflected by the monotonically increasing decay exponent (the absolute value of the exponent) of the result.

Our numerical study also shows that the far end boundary condition $c$, which in the large system size limit should not impact the system, {\it does change} the scaling dimension in a way that is not captured by our analytic computation. The reason is that the boundary condition on the far end may introduce additional bcc operators and thus change the free energy. It would be interesting to have a CFT calculation that reproduces the better numerical result in Fig.~\ref{fig:PDN_fit} {\color{red}for the process in Eq.~\eqref{eq:DNP_rand()}}
% \begin{equation}
% {\rm DN} + {\rm P} \rightarrow \lambda + {\rm P} 
% \end{equation}

% connecting Luttinger liquid(quantum wire) using votage to control the boundary condition. X-ray edge singularity. examples include Taylor santo's topological phases. Hope to have experimental setup. 

This set of boundary conditions can be realized by connecting two compact bosons. There are already numerous theoretical and experimental works on the boundary conditions of a Luttinger liquid\cite{schmeltzer_zero_1999,anfuso_luttinger_2003,voit_bounded_2000,fabrizio_interacting_1995,egger_applying_1998}, which is the universal compact boson theory of the (bosonized) one-dimensional electron gas\cite{giamarchi_quantum_2015}. For example, gate voltage \cite{egger_applying_1998} may be used to twist the left and right modes of the boson to create a boundary condition interpolating between the normal open and fixed boundary conditions. The interface studied in this paper is a generalization which (in the folding picture) twists the two independent bosons (two left modes plus two right modes) on the two sides. An X-ray edge singularity experiment in a quantum wire system, which uses ions to switch on and off the boson interfaces should be plausible to detect the exponent found in this paper. 


%%% Local Variables:
%%% TeX-master: "bCFT_paper"
%%% TeX-PDF-mode: t
%%% End:
