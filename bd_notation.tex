
We use chemical reaction style to represent the change of boundary conditions. Taking the example of the echo diagrams in Fig.~\ref{fig:echo}, there are three boundary conditions $a,b,c$ in the folding picture, which represents the status on two ends of the chain before and after the quench. The choice of a uniform $c$ boundary condition on the far end of the chain is to isolate the effect coming from the bcc on the conformal interface. The process $a + c \rightarrow b + c$ represents the change of boundary condition from the combination $a$/$c$ on the two ends to $b$/ $c$ after the quench. Since each letter can take a general conformal interface defined by the $S$ matrix in Eq.~\eqref{eq:S1_S2}, we denote it as
\begin{equation}
S_a( \theta_a ) + S_c( \theta_c) \rightarrow S_b( \theta_b )  + S_c( \theta_c ) 
\end{equation}
Analytic computation can be performed when we neglect the boundary condition $c$, assuming $L \gg \tau$. To remove the bcc operator from $c$ to $a$ at infinity, we take $a = c$ in the numerical calculation. 

In the ``cut-and-join" protocol we considered, $a$ should be one of `DD', `DN', `ND', `NN', $b$ is taken to be $S_1( \theta )$ or $S_2( \theta )$. The physical situation of connecting two compact bosons and our lattice model corresponds to the choice of $S_1( \theta)$, we also reserve the notation $\lambda$ for this type of boundary condition. For instance, the notation for the process presented in Fig.~\ref{fig:echo} is
\begin{equation}
{\rm DN }  + {\rm DN} \rightarrow \lambda + { \rm DN} 
\end{equation}
Another interesting case is to take $a$ or $c$ to be completely transmitting interface, i.e. $S_2( \frac{\pi}{4} )$. This corresponds to the traditional periodic boundary condition and we use symbol 'P' to denote it. 


%%% Local Variables:
%%% TeX-master: "bCFT_paper"
%%% TeX-PDF-mode: t
%%% End:
