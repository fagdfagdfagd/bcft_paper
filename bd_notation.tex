
We use chemical reaction style to represent the change of boundary conditions. Taking the example of the echo diagrams in Fig.~\ref{fig:echo}, there are three boundary conditions $a,b,c$ in the folding picture, which represents the status of the two ends of the chain before and after the quench. The choice of a uniform $c$ boundary condition on the far end of the chain is to isolate the effect coming from the bcc on the $ab$ interface. The process $a + c \rightarrow b + c$ represents the change of boundary condition from the combination $a$/$c$ prior to the quench to $b$/$c$ after the quench. Since each letter can take a general conformal interface defined by the $S$ matrix in Eq.~\eqref{eq:S1_S2}, we denote it as
\begin{equation}
\label{eq:Full_notation_rand()}
S_a( \theta_a ) + S_c( \theta_c) \rightarrow S_b( \theta_b )  + S_c( \theta_c ) .
\end{equation}
In most cases of the following, we will consider taking $a = c$ to remove the bcc operator from $a$ to $c$ at infinity. And we will use the shorthand notation
\begin{equation}
S_a( \theta_a ) \rightarrow S_b( \theta_b )
\end{equation}
to remind ourselves that we are isolating the boundary condition change only on the joint of the two chains. 

In the ``cut-and-join" protocol we considered, $a$ should be one of `DD', `DN', `ND', `NN', $b$ is taken to be $S_1( \theta )$ or $S_2( \theta )$. The physical situation of connecting two compact bosons (and our numerical simulation) corresponds to the choice of $S_1( \theta)$, and we reserve the notation $\lambda$ for this type of the boundary condition. For instance, the notation for the process presented in Fig.~\ref{fig:echo} is
\begin{equation}
\label{eq:DNDN}
 {\rm DN} \rightarrow \lambda.
\end{equation}
Another interesting case is to take $a$ or $c$ to be a completely transmitting interface, i.e. $S_2( \frac{\pi}{4} )$. This $S$-matrix corresponds to the traditional periodic boundary condition and we use symbol 'P' to denote it. 


%%% Local Variables:
%%% TeX-master: "bCFT_paper"
%%% TeX-PDF-mode: t
%%% End:
