
In this appendix, we provide more details for calculating the amplitude $Z_{ab}$ in Sec.~\ref{sec:analytic_numerics}. We start to prove the following identity for a real symmetric matrix $M$
\begin{equation}
\label{eq:first_id_app_pf_of_id}
e^{- \vec{b}^{\dagger} M \vec{b} } e^{ \vec{b}^{\dagger} R \bar{\vec{b}}^{\dagger} }  = e^{ \vec{b}^{\dagger} e^{-M}  R \bar{\vec{b}}^{\dagger} } e^{- \vec{b}^{\dagger} M \vec{b} } ,
%\exp\Big\{- \vec{b}^{\dagger} M \vec{b}  \Big\} \exp \Big\{ \vec{b}^{\dagger} R \bar{\vec{b}}^{\dagger}  \Big\}  = \exp \Big\{ \vec{b}^{\dagger} e^{-M}  R \bar{\vec{b}}^{\dagger}  \Big\} \exp\Big\{- \vec{b}^{\dagger} M \vec{b}  \Big\} 
\end{equation}
where ${\bf b}$ and $\bar{\bf b}$ are vectors of bosonic operators. The matrix notation here should be understood as a bilinear expression as explained below Eq.~\eqref{eq:bd_state_matrix}.  

To prove Eq.~\eqref{eq:first_id_app_pf_of_id}, we first consider the special case where $R=\mathbb{I}$. We diagonalize $M = O^{T} \Lambda O $ and rotate the two sets of boson operators to the diagonal basis
\begin{equation}
  \vec{b}^{\dagger}  M \vec{b} = \vec{d}^{\dagger} \Lambda \vec{d}  \quad \vec{d} = O \vec{b} \quad \vec{\bar{d}}^\dagger = O^T \vec{\bar{b}}^\dagger,
\end{equation}
where we understand $\vec{\bar{b}}^\dagger$ as a column vector independent of $\vec{{b}}^\dagger$. Thus the whole expression can be written as
\begin{equation}
\begin{aligned}
  e^{- \vec{b}^{\dagger} M \vec{b} } e^{ \vec{b}^{\dagger} \bar{\vec{b}}^\dagger }  =  
  e^{- \vec{d}^{\dagger} \Lambda \vec{d} } e^{   \vec{d}^{\dagger} \bar{\vec{d}}^\dagger } = \prod_i  e^{- \lambda_i d_i^{\dagger} d_i } e^{  d_i^{\dagger} \bar{d}_i ^{\dagger} }.
%  \exp&\Big\{- \vec{b}^{\dagger} M \vec{b}  \Big\} \exp \Big\{ \vec{b}^{\dagger} \bar{\vec{b}}^\dagger  \Big\}  =  
%  \exp\Big\{- \vec{d}^{\dagger} \Lambda \vec{d}  \Big\} \exp \Big\{   \vec{d}^{\dagger} \bar{\vec{d}}^\dagger  \Big\} \\
%& = \prod_i  \exp\Big\{- \lambda_i d_i^{\dagger} d_i  \Big\} \exp \Big\{  d_i^{\dagger} \bar{d}_i ^{\dagger}  \Big\}
\end{aligned}
\end{equation}
We recall for $ [X, Y] = sY $, 
\begin{equation}
  e^X e^{Y} = e^{\exp (s ) Y} e^{X},
\end{equation}
which is a solvable case of the Baker-Campbell-Hausdorff formula. Upon taking $X = -\lambda_i d_i^{\dagger} d_i$, $Y = d_i^{\dagger} \bar{d}^{\dagger}_i$, we have
\begin{equation}
\label{eq:lambda_commutator}
[- \lambda_i d_i^{\dagger} d_i, d_i ^{\dagger} \bar{d}_i^{\dagger}] =  - \lambda_i  d_i ^{\dagger} \bar{d}_i^{\dagger} ,
\end{equation}
and so $s = - \lambda_i$ for each $\lambda_i$. This enables us to commute those exponentials
\begin{equation}
\begin{aligned}
e^{- \vec{b}^{\dagger} M \vec{b} } e^{  \vec{b}^{\dagger} \bar{\vec{b}}^\dagger }   &= \prod_i e^{ e^{- \lambda_i }  d^{\dagger}_i \bar{d}^{\dagger}_i }  e^{-\lambda_i  d^{\dagger}_i d_i }  = e^{ \vec{b}^{\dagger} e^{-M}  \bar{\vec{b}}^\dagger} e^{- \vec{b}^{\dagger} M \vec{b} }. 
% \exp\Big\{- \vec{b}^{\dagger} M \vec{b}  \Big\} \exp \Big\{  \vec{b}^{\dagger} \bar{\vec{b}}^\dagger  \Big\}   &= \prod_i \exp \Big\{ e^{- \lambda_i }  d^{\dagger}_i \bar{d}^{\dagger}_i  \Big\}  \exp \Big\{-\lambda_i  d^{\dagger}_i d_i  \Big\} \\
 %& = \exp \Big\{ \vec{b}^{\dagger} e^{-M}  \bar{\vec{b}}^\dagger  \Big\} \exp\Big\{- \vec{b}^{\dagger} M \vec{b}  \Big\} 
\end{aligned}
\end{equation}
For the general case where $R \neq\mathbb{I}$, we take $\vec{\bar{d}}^* = O^T R \vec{\bar{b}}^*$. This will not change the commutation relation of ${\bf d}$, and the role of $\bar{b}$ is decorative in Eq.~\eqref{eq:lambda_commutator}. Hence the rest of the proof follows the same way. \hfill$\blacksquare$

A direct consequence of Eq.~\eqref{eq:first_id_app_pf_of_id} is the following
\begin{equation}
\label{eq:second_id_in_app_pf_of_id}
Z_{ab} = \langle 0 | e^{ \vec{b} R_a^* \vec{\bar{b}} } e^{ - \vec{b}^{\dagger} M  \vec{b} } e^{  \vec{b}^{\dagger} R_b  \vec{\bar{b}}^{\dagger} }  |0  \rangle  = \frac{1}{\det( 1- R_a^{\dagger} e^{-M} R_b )} ,
%Z_{ab} = \langle 0 | \exp\Big\{ \vec{b} R_a^* \vec{\bar{b}}\Big\} \exp\Big\{ - \vec{b}^{\dagger} M  \vec{b} \Big\}   \exp\Big\{  \vec{b}^{\dagger} R_b  \vec{\bar{b}}^{\dagger}\Big\}  |0  \rangle  = \frac{1}{\det( 1- R_a^{\dagger} e^{-M} R_b )} 
\end{equation}
where $|0\rangle$ is the vacuum for ${\bf b}$ and $\bar{\bf b}$. 

One can use the identity in Eq.~\eqref{eq:first_id_app_pf_of_id} to reduce $Z_{ab}$ to 
\begin{equation}
Z_{ab} =   \langle 0 | \exp\Big\{ \vec{b} R_a^* \vec{\bar{b}}\Big\}  \exp \Big\{ \vec{b}^{\dagger} e^{-M}  R_b \bar{\vec{b}}^{\dagger}  \Big\}  |0 \rangle, 
\end{equation}
then a direct application of the MacMahon master theorem
\begin{equation}
  \langle 0 | \exp \Big\{ \vec{b}_1 X \vec{b}_2 \Big\}  \exp \Big\{ \vec{b}^{\dagger}_1 Y \vec{b}^{\dagger}_2 \Big\}|0  \rangle 
 = \frac{1}{\det(1 - X^T Y )}
\end{equation}
proves Eq.~\eqref{eq:second_id_in_app_pf_of_id}. \hfill$\blacksquare$


%%% Local Variables:
%%% TeX-master: "bCFT_paper"
%%% TeX-PDF-mode: t
%%% End:
